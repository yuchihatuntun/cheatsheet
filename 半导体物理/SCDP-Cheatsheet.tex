\documentclass[10pt, a4paper, landscape]{article}

% -------------------------------------------------
% 宏包引入
% -------------------------------------------------
\usepackage[fontset=mac]{ctex}       % 中文支持
\usepackage{multicol}   % 多分栏
\usepackage{calc}
\usepackage{ifthen}
\usepackage[landscape]{geometry} % 页面设置
\usepackage{amsmath,amsthm,amsfonts,amssymb} % 数学公式
\usepackage{color,graphicx,overpic} % 颜色与图片
\usepackage{hyperref}   % 超链接
\usepackage{enumitem}   % 列表环境控制
\usepackage{titlesec}   % 标题控制
\usepackage{bm}         % 加粗数学符号
\usepackage{xcolor}
\usepackage{tikz}       % 绘图
\usetikzlibrary{decorations.pathreplacing, positioning} % 加载brace装饰库
\setCJKmainfont{PingFang SC}
\setCJKsansfont{PingFang SC}
\setCJKmonofont{PingFang SC}

% -------------------------------------------------
% 自定义颜色
% -------------------------------------------------

\definecolor{myblue}{HTML}{003153} % 蓝色字
\definecolor{myred}{HTML}{85120F}  % 红色字
\definecolor{hlblue}{HTML}{ADC1DB} % 蓝色高亮
\definecolor{hlred}{HTML}{D66A83}  % 红色高亮
\definecolor{hlyellow}{HTML}{E3C79F}  % 奢金高亮
\definecolor{hlgreen}{HTML}{BED49D}  % 抹茶绿高亮

% -------------------------------------------------
% 极限空间压缩设置 (核心部分)
% -------------------------------------------------

% 1. 页边距设置为极小 (0.5cm)
\geometry{top=0.5cm,left=0.5cm,right=0.5cm,bottom=0.5cm}

% 2. 去掉段落首行缩进,改为段落间略微留空(可选,这里为了紧凑设为0)
\setlength{\parindent}{0pt}
\setlength{\parskip}{0pt}

% 3. 设置正文基础字体大小为 scriptsize (约8pt),如果还觉得大,可以改为 \tiny
\renewcommand{\baselinestretch}{0.9} % 压缩行间距
\let\oldfootnotesize\footnotesize
\renewcommand{\footnotesize}{\fontsize{7pt}{8pt}\selectfont}

% 4. 压缩列表环境 (Itemize/Enumerate) 的间距
\setlist{nolistsep} 
\setlist[itemize]{leftmargin=*}
\setlist[enumerate]{leftmargin=*}

% 5. 压缩标题间距
\titleformat{\section}{\bfseries\scriptsize\color{myblue}}{}{0em}{}[\hrule] % 标题带下划线,蓝色,省空间
\titlespacing*{\section}{0pt}{2pt}{1pt} % 上方留2pt,下方留1pt
\titleformat{\subsection}
    [runin] % 不换行
    {\bfseries\scriptsize} % 粗体、scriptsize,黑色字体
    {} % 不显示编号
    {0pt} % 标题与正文间距
    {\hlyellow} % 用hlyellow高亮命令包裹标题
    [] % 标题内容后无内容
\titleformat{\subsubsection}
    [runin] % 不换行
    {\bfseries\tiny} % 粗体、scriptsize,黑色字体
    {} % 不显示编号
    {0pt} % 标题与正文间距
    {\hlgreen} % 用hlgreen高亮命令包裹标题
    [] % 标题内容后无内容
\titlespacing*{\subsection}{0pt}{1pt}{0.5em} % 上方1pt,下方0.5em(水平间距)
\titlespacing*{\subsubsection}{0pt}{1pt}{0.5em} % 上方1pt,下方0.5em(水平间距)

% -------------------------------------------------
% 自定义命令
% -------------------------------------------------
% 颜色字
\newcommand{\red}[1]{\textbf{\textcolor{myred}{#1}}}  
\newcommand{\blue}[1]{\textbf{\textcolor{myblue}{#1}}} 
\newcommand{\entry}[2]{$\bullet$ \textbf{#1}: #2\par\vspace{0.5pt}}
% 高亮
\newcommand{\cbox}[2][yellow]{\begingroup\setlength{\fboxsep}{1pt}\colorbox{#1}{\strut#2}\endgroup}
\newcommand{\hlblue}[1]{\cbox[hlblue]{#1}}
\newcommand{\hlred}[1]{\cbox[hlred]{#1}}
\newcommand{\hlyellow}[1]{\cbox[hlyellow]{#1}}
\newcommand{\hlgreen}[1]{\cbox[hlgreen]{#1}} 
% 图片插入
\newcommand{\img}[2][0.9\linewidth]{%
    {\par\vspace{1pt}\centering\includegraphics[width=#1]{#2}\par\vspace{1pt}}%
}
% 左图右文 (参数: [图片宽度比例]{图片路径}{右侧文字内容})
\newcommand{\imgleft}[3][0.3]{%
    \noindent\begin{minipage}[t]{#1\linewidth}%
        \vspace{0pt}%
        \includegraphics[width=\linewidth]{#2}%
    \end{minipage}%
    \hfill%
    \begin{minipage}[t]{0.98\linewidth - #1\linewidth}%
        \vspace{0pt}%
        #3%
    \end{minipage}\par\vspace{2pt}%
}
% 左文右图 (参数: [图片宽度比例]{图片路径}{左侧文字内容})
\newcommand{\imgright}[3][0.3]{%
    \noindent\begin{minipage}[t]{0.98\linewidth - #1\linewidth}%
        \vspace{0pt}%
        #3%
    \end{minipage}%
    \hfill%
    \begin{minipage}[t]{#1\linewidth}%
        \vspace{0pt}%
        \includegraphics[width=\linewidth]{#2}%
    \end{minipage}\par\vspace{2pt}%
}
% -------------------------------------------------
% 新增:概念速查表专用命令
% -------------------------------------------------
% 表格容器
\newcommand{\concepttable}[1]{%
    {\setlength{\tabcolsep}{1.5pt}% 局部减小列间距
     \renewcommand{\arraystretch}{0.92}% 局部紧缩行距
     \par\vspace{2pt}{\color{myblue}\hrule height 0.6pt}\vspace{1pt}% 上边框(蓝色,0.6pt粗)
     \noindent\begin{tabular}{@{}p{0.22\linewidth}p{0.76\linewidth}@{}}%
     #1%
     \end{tabular}%
     \vspace{1pt}{\color{myblue}\hrule height 0.6pt}\par\vspace{2pt}}% 下边框(蓝色,0.6pt粗)
}
% 表格行 (参数: {概念名}{解释})
\newcommand{\conrow}[2]{\blue{#1} & #2 \\}




% -------------------------------------------------
% 正文
% -------------------------------------------------
\begin{document}

\tiny

% 三栏布局
\begin{multicols*}{3}

\section{第一章\ 半导体物理基础}

\subsection{能带的产生}

\subsubsection{允带与禁带}

\concepttable{
    \conrow{价带 $E_v$}{被价电子填满的能带}
    \conrow{导带 $E_c$}{主要由自由电子占据的能带}
    \conrow{禁带宽度 $E_g$}{$E_g = E_c - E_v$,区分金属、半导体和绝缘体的关键参数}
}

\subsubsection{能带影响因素}

能带受多种因素影响,主要包括温度和掺杂。

\noindent
\begin{minipage}[t]{0.32\linewidth}
    \blue{温度影响}
    \concepttable{
        \conrow{机制}{温度升高 $\to$ 晶格膨胀 $\to$ 原子间作用减弱}
        \conrow{结果}{$E_g$ 变窄}
    }
\end{minipage}
\hfill
\begin{minipage}[t]{0.32\linewidth}
    \blue{掺杂影响}
    \concepttable{
        \conrow{机制}{掺杂浓度增加 $\to$ 能带变窄效应}
        \conrow{结果}{影响能带结构}
    }
\end{minipage}
\hfill
\begin{minipage}[t]{0.32\linewidth}
\blue{本征激发}

\concepttable{
    \conrow{定义}{价电子吸收热能跃迁至导带}
    \conrow{载流子}{成对产生\red{电子}(导带)和\red{空穴}(价带)}
}
\end{minipage}

\subsection{载流子统计分布}

\subsubsection{费米分布与载流子浓度}

费米-狄拉克分布描述了量子态被电子占据的概率,进而决定了半导体中载流子的浓度。

\concepttable{
    \conrow{费米-狄拉克分布}{$f(E) = \frac{1}{1 + \exp[(E - E_F)/kT]}$,描述能量为 $E$ 的量子态被电子占据的概率}
    \conrow{\red{费米能级 $E_F$}}{化学势的具体体现,$f(E_F) = 1/2$,是表征半导体统计性质的参考能级}
    \conrow{导带电子浓度}{$n_0 = N_c \exp[-(E_c - E_F)/kT]$,其中 $N_c$ 为导带有效状态密度}
    \conrow{价带空穴浓度}{$p_0 = N_v \exp[-(E_F - E_v)/kT]$,其中 $N_v$ 为价带有效状态密度}
    \conrow{质量作用定律}{$n_0 p_0 = n_i^2 = N_c N_v \exp(-E_g/kT)$,$n_i$ 只与温度和 $E_g$ 有关}
}

\subsubsection{掺杂对载流子浓度的影响}

\concepttable{
    \conrow{N型掺杂}{施主原子提供电子,$n_0 \approx N_d$(室温全电离),$E_F$ 靠近 $E_c$}
    \conrow{P型掺杂}{受主原子提供空穴,$p_0 \approx N_a$(室温全电离),$E_F$ 靠近 $E_v$}
    \conrow{补偿掺杂}{同时掺入施主和受主,多数载流子浓度由 $|N_d - N_a|$ 决定}
}

\subsubsection{\red{温度对载流子浓度的影响}}

温度区域划分:

\noindent
\begin{minipage}[t]{0.32\linewidth}
    \blue{低温区(冻结区)}
    \concepttable{
        \conrow{特征}{杂质未完全电离}
        \conrow{结果}{载流子浓度随温度升高而\red{增加}}
    }
\end{minipage}
\hfill
\begin{minipage}[t]{0.32\linewidth}
    \blue{中温区(饱和区)}
    \concepttable{
        \conrow{特征}{杂质全电离,本征激发可忽略}
        \conrow{结果}{载流子浓度\red{基本恒定},$n_0 \approx N_d$}
    }
\end{minipage}
\hfill
\begin{minipage}[t]{0.32\linewidth}
    \blue{高温区(本征区)}
    \concepttable{
        \conrow{特征}{本征激发占主导}
        \conrow{结果}{$n_i$ 指数增长,$n_0 \approx p_0 \approx n_i$}
    }
\end{minipage}

\subsubsection{费米能级位置的物理意义}

为后面 PN 结等接触分析做铺垫

\concepttable{
    \conrow{本征半导体}{$E_F = E_i \approx (E_c + E_v)/2$(禁带中央),$n_0 = p_0 = n_i$}
    \conrow{N型半导体}{$E_F$ 上移靠近 $E_c$,掺杂越重 $E_F$ 越接近 $E_c$}
    \conrow{P型半导体}{$E_F$ 下移靠近 $E_v$,掺杂越重 $E_F$ 越接近 $E_v$}
    \conrow{接触电势}{不同材料接触时费米能级必须拉平,形成内建电场(PN结、金半接触的基础)}
}

\subsection{半导体的载流子输运}

主要分为漂移和扩散两种机制。

\noindent
\begin{minipage}[t]{0.49\linewidth}
    \blue{漂移运动}

    \concepttable{
        \conrow{定义}{载流子在电场作用下的定向运动}
        \conrow{漂移速度}{$v_d = \mu E$,其中 $\mu$ 为迁移率}
        \conrow{漂移电流密度}{$J_{drift} = nq\mu_n E + pq\mu_p E$}
    }

    \entry{迁移率 $\mu$}{单位电场下载流子的平均漂移速度,反映运动能力,单位 $\mathrm{cm}^2/(\mathrm{V}\cdot\mathrm{s})$,受\red{晶格散射}和\red{杂质散射}影响}

    \entry{饱和速度 $v_{sat}$}{强电场下漂移速度趋于上限}
\end{minipage}
\hfill
\begin{minipage}[t]{0.49\linewidth}
    \blue{扩散运动}

    \concepttable{
        \conrow{定义}{载流子在浓度梯度驱动下从高浓度向低浓度区的运动}
        \conrow{扩散电流密度}{$J_{diff} = qD_n\frac{dn}{dx} - qD_p\frac{dp}{dx}$}
    }

    \entry{扩散系数 $D$}{\red{爱因斯坦关系}}{\red{$D = \frac{kT}{q}\mu$},扩散与漂移受相同散射机制限制}
\end{minipage}

\subsubsection{总电流密度}

\concepttable{
    \conrow{电子电流}{$J_n = qn\mu_n E + qD_n\frac{dn}{dx}$}
    \conrow{空穴电流}{$J_p = qp\mu_p E - qD_p\frac{dp}{dx}$}
    \conrow{总电流}{$J = J_n + J_p$(漂移 + 扩散)}
}

\subsection{非平衡载流子的产生与复合}

\subsubsection{过剩载流子}

\concepttable{
    \conrow{非平衡态}{外界作用(光照、电压)使载流子浓度偏离平衡,$np \neq n_i^2$}
    \conrow{过剩载流子}{$\Delta n = n - n_0$,$\Delta p = p - p_0$。\red{小注入条件下 $\Delta n = \Delta p$}}
}

\subsubsection{复合与产生}

\concepttable{
    \conrow{复合}{电子从导带跃迁至价带,与空穴湮灭,释放能量(光子或声子)}
    \conrow{产生}{价带电子吸收能量跃迁至导带,产生电子-空穴对}
    \conrow{平衡态}{$G = R$(产生率 = 复合率),载流子浓度恒定}
}

\entry{复合机制}{}

\noindent
\begin{minipage}[t]{0.32\linewidth}
    \blue{直接复合}
    \concepttable{
        \conrow{机制}{电子直接跃迁至价带}
        \conrow{特点}{发光(GaAs等直接带隙)}
    }
\end{minipage}
\hfill
\begin{minipage}[t]{0.32\linewidth}
    \blue{间接复合}
    \concepttable{
        \conrow{机制}{通过\red{复合中心}(杂质、缺陷)}
        \conrow{特点}{Si、Ge 等间接带隙半导体}
    }
\end{minipage}
\hfill
\begin{minipage}[t]{0.32\linewidth}
    \blue{俄歇复合}
    \concepttable{
        \conrow{机制}{能量转移给第三个载流子}
        \conrow{特点}{重掺杂或高注入下显著}
    }
\end{minipage}

\subsubsection{载流子寿命与扩散长度}

\concepttable{
    \conrow{\red{寿命 $\tau$}}{过剩载流子衰减至 $1/e$ 的时间,$\Delta n(t) = \Delta n(0) e^{-t/\tau}$}
    \conrow{\red{扩散长度 $L$}}{过剩载流子在寿命期内扩散的平均距离,$L = \sqrt{D\tau}$}
}

\subsubsection{准费米能级}

\concepttable{
    \conrow{定义}{非平衡态下,电子和空穴各有独立的费米能级:$E_{Fn}$(电子)、$E_{Fp}$(空穴)}
    \conrow{载流子浓度}{$n = n_i e^{(E_{Fn} - E_i)/kT}$,$p = n_i e^{(E_i - E_{Fp})/kT}$}
    \conrow{平衡态极限}{$E_{Fn} = E_{Fp} = E_F$,$np = n_i^2$}
}

\section{第二章\ PN结}

\subsection{PN 结的形成过程}

\subsubsection{制备方法}

通过不同工艺引入杂质,形成特定的杂质浓度分布 $N(x)$,进而影响 PN 结的电学特性。

\noindent
\begin{minipage}[t]{0.32\linewidth}
    \blue{扩散法}
    \concepttable{
        \conrow{过程}{杂质从表面向内部扩散}
        \conrow{结类型}{缓变结、线性缓变结}
        \conrow{特点}{浓度随深度 $x$ 逐渐变化}
    }
\end{minipage}
\hfill
\begin{minipage}[t]{0.32\linewidth}
    \blue{合金法}
    \concepttable{
        \conrow{过程}{金属杂质熔解后重结晶}
        \conrow{结类型}{突变结(理想模型)}
        \conrow{特点}{在 $X_j$ 处浓度阶跃突变}
    }
\end{minipage}
\hfill
\begin{minipage}[t]{0.32\linewidth}
    \blue{离子注入法}
    \concepttable{
        \conrow{过程}{高能离子束轰击半导体}
        \conrow{结类型}{高斯分布}
        \conrow{特点}{峰值在投影射程 $R_p$ 处}
    }
\end{minipage}

\subsubsection{PN 结平衡过程}

\entry{初始状态}{
    因浓度梯度,P 区空穴向 N 区扩散,N 区电子向 P 区扩散
}

\entry{空间电荷区形成}{扩散过界的电子-空穴在交界面附近相遇并复合,两侧各失去多子,留下带正、负电的施、受主离子:$N_D^+$ 和 $N_A^-$。交界面附近自由载流子被消耗殆尽,形成\red{空间电荷区}(也称耗尽区)}

\entry{内建电场与动态平衡}{

    \concepttable{
        \conrow{内建电场}{空间电荷区分离正负电荷,形成从 N 区指向 P 区的电场 $E_i$}
        \conrow{\red{平衡条件}}{扩散流密度 $J_{diff}$ 与漂移流密度 $J_{drift}$ 大小相等、方向相反,$J_{total} = 0$}
        \conrow{热平衡态}{费米能级 $E_F$ 拉平,无宏观净电流,\red{但微观载流子交换持续}}
    }

}

\noindent
\begin{minipage}[t]{0.49\linewidth}
    \subsection{平衡 PN 结}

    平衡状态下 PN 结的能带结构:

    \img[0.5\textwidth]{images/image-2026-01-01-13-25-54.png}

    电子从 N 区向 P 区运动需克服势能差 $eV_{bi}$(阻挡多子继续扩散的势垒高度),维持动态平衡。
\end{minipage}
\hfill
\begin{minipage}[t]{0.49\linewidth}

    \hlgreen{\red{内建电势$V_{bi}$}}
    \concepttable{
        \conrow{定义表达式}{
        $eV_{bi} = E_{c,N} - E_{c,P}$\par
        $= E_{v,P} - E_{v,N} = |\phi_{Fp}| + |\phi_{Fn}|$}
        \conrow{表达式}{$V_{bi} = \frac{kT}{e}\ln\left(\frac{N_a N_d}{n_i^2}\right)$\par
        \red{$=V_t\ln\left(\frac{N_a N_d}{n_i^2}\right)$}}
        \conrow{影响因素}{掺杂浓度 $N_a$、$N_d$ 和温度 $T$}
    }

\end{minipage}

\entry{\hlgreen{电荷中性条件}}{\red{$N_A x_p = N_D x_n$,耗尽层展宽与掺杂浓度成反比}}

\noindent
\begin{minipage}[t]{0.49\linewidth}
    \hlgreen{\red{内建电场 $E(x)$}}:单边突变结自动\red{删除高掺侧量}。
    \concepttable{
        \conrow{P区耗尽层}{$E(x) = \frac{-eN_A}{\varepsilon_s}(x + x_p)$\par
        }
        \conrow{N区耗尽层}{$E(x) = \frac{-eN_D}{\varepsilon_s}(x_n - x)$\par
        }
        \conrow{积分可得}{$V_{bi} = \frac{e}{2\varepsilon_s}(N_D x_n^2 + N_A x_p^2)$}
        \conrow{最大场强}{$E_{max} = \frac{eN_Dx_n}{\varepsilon_s} = \frac{2V_{bi}}{W}$}
    }
\end{minipage}
\hfill
\begin{minipage}[t]{0.49\linewidth}

\hlgreen{\red{空间电荷区宽度}}:单边突变结总宽度为\red{低掺杂侧耗尽宽度}。

\concepttable{
    \conrow{P侧宽度}{$x_p = \sqrt{\frac{2\varepsilon_s V_{bi}}{e} \cdot \frac{N_D}{N_A(N_A+N_D)}}$}
    \conrow{N侧宽度}{$x_n = \sqrt{\frac{2\varepsilon_s V_{bi}}{e} \cdot \frac{N_A}{N_D(N_A+N_D)}}$}
    \conrow{总宽度}{$W = \sqrt{\frac{2\varepsilon_s V_{bi}}{e} \cdot \frac{N_A + N_D}{N_A N_D}}$}
}

\end{minipage}

\subsection{PN 结的直流特性}

能带图正偏让 N 侧能级上升($V_{bi} - V_a$),反偏让 N 侧能级下降($V_{bi} + V_R$)

\noindent
\begin{minipage}[t]{0.49\linewidth}

\subsubsection{PN 结正偏}

    $n_{p}(-x_p) = n_{p0} \exp\left(\frac{eV_a}{kT}\right)$

    $p_n(x_n) = p_{n0} \exp\left(\frac{eV_a}{kT}\right)$

    \img[0.5\textwidth]{images/image-111.png}

    通过解\blue{双极输运方程},得到通解(少子分布):

    $
    \delta p_{n}(x) = p_{n0} \left[ \exp\left( \frac{eV_{a}}{kT} \right) - 1 \right] \exp\left( \frac{x_{n} - x}{L_{p}} \right)
    $

    $
    \delta n_{p}(x) = n_{p0} \left[ \exp\left( \frac{eV_{a}}{kT} \right) - 1 \right] \exp\left( \frac{x_{p} + x}{L_{n}} \right)
    $

    正偏为\red{少子扩散电流},通过扩散电流公式可求边界电流:

    $J_p(x_n) = \frac{e D_p p_{n0}}{L_p} \left[ \exp\left(\frac{e V_a}{k T}\right) - 1 \right]$

    $J_n(-x_p) = \frac{e D_n n_{p0}}{L_n} \left[ \exp\left(\frac{e V_a}{k T}\right) - 1 \right]$

    总电流(\red{肖克利方程})为

    $J = J_s \left[ \exp\left( \frac{eV_a}{kT} \right) - 1 \right]$
    
\end{minipage}
\hfill
\begin{minipage}[t]{0.49\linewidth}

\subsubsection{PN 结反偏}

    少子分布如图所示:

    \img[0.5\textwidth]{images/image-2026-01-01-20-07-05.png}

    \entry{考虑复合产生过程}{

        \concepttable{

            \conrow{反偏产生}{$J_{\text{gen}} = \frac{en_i W}{2\tau_0}$}

            \conrow{反偏总电流}{$J_{R} = J_{S} + J_{\text{gen}}$}

            \conrow{正偏复合}{$J_{\text{rec}} = \frac{e W n_{i}}{2 \tau_{0}} \exp \left( \frac{e V_{a}}{2 k T} \right) $}

            \conrow{正偏总电流}{$J = J_{\text{rec}} + J_{D}$}
        }

        电流较小时以复合为主,电流较大时以扩散为主。

        \entry{通用表达式(修正后)}{\red{$I = I_{s}\left[\exp\left(\frac{eV_{a}}{nkT}\right) - 1\right]$}}
        \concepttable{
                \conrow{正偏较大}{$n \approx 1$(扩散为主,\red{大注入也会导致$n = 2$})}
                \conrow{正偏较小}{$n \approx 2$(复合为主)}
        }
    }
        
    \entry{温度特性}{
        温度升高,电流密度变大,虽然正偏还有缩小项,但是不如反向饱和电流增加得多:$J_s \propto n_i^2,n_i \propto T^{3/2} e^{-E_g /(2kT)}$
    }

\end{minipage}
\vspace{-5pt}
\subsection{PN 结电容}

两种电容:势垒电容和扩散电容。

\noindent
\begin{minipage}[t]{0.49\linewidth}

    \subsubsection{势垒电容}

    \entry{单位面积势垒电容}{\red{$C_T' = \frac{\varepsilon_S}{W}$}}

\end{minipage}
\hfill
\begin{minipage}[t]{0.49\linewidth}

    \subsubsection{扩散电容}
    \red{$C_D = \frac{e^2}{kT} \left( L_p p_{n0} + L_n n_{p0} \right) \exp\left( \frac{e V_a}{k T} \right)$}

\end{minipage}

\entry{小信号模型下计算扩散电阻、电容}{
    \concepttable{
        \conrow{扩散电导}{$g_d = \frac{(I_{p0} + I_{n0})}{V_t} = \frac{I_{DQ}}{V_t}$}
        \conrow{扩散电容}{$C_d = \frac{1}{2V_t} (I_{p0}\tau_{p0} + I_{n0}\tau_{n0})$}
    }
}

\subsection{动态开关特性}

从关态转变到开态所需开启时间很短,从开态转变到关态($+U \rightarrow -U$)所需关闭时间却很长。
\entry{根本原因}{反向延迟由 PN 结的\red{电荷贮存}引起(正向导通时,互相注入少子,非平衡少子($p_n$、$n_p$)在耗尽层附近扩散区大量积累,形成\red{贮存电荷 $Q$})}

\entry{关联规律}{正向电流 $I_F \uparrow \Rightarrow$ 注入少子 $\uparrow \Rightarrow$ 贮存电荷 $Q \uparrow \Rightarrow$ 关断时清理时间 $\uparrow$,恢复时间 $\uparrow$}

\imgleft[0.3]{images/image-2026-01-01-21-53-18.png}{

\noindent
\begin{minipage}[t]{0.32\linewidth}
    \blue{正向导通 ($t < t_0$)}
    \concepttable{
        \conrow{物理过程}{电流为正向偏置电流,电压几乎全部加在电阻 $R$ 上}
    }
\end{minipage}
\hfill
\begin{minipage}[t]{0.32\linewidth}
    \blue{切换瞬间 ($t = t_0$)}
    \concepttable{
        \conrow{跳变过程}{贮存电荷使 PN 结仍呈现低阻,$I_R \approx U_2/R$}
    }
\end{minipage}
\hfill
\begin{minipage}[t]{0.32\linewidth}
    \blue{贮存时间 $t_s$}
    \concepttable{
        \conrow{物理过程}{电流保持 $I_R$ 不变,反向电压持续抽取贮存电荷}
    }
\end{minipage}

\entry{衰减时间 $t_f$}{电流从 $I_R$ 降至 $0.1 I_R$,耗尽层建立,恢复高阻}

\entry{恢复时间}{$t_{rr} = t_f + t_s$,让输出伴有延迟,决定了工作频率。}

}

\imgleft[0.3]{images/image-2026-01-01-23-29-31.png}{
    \entry{全流程详解}{
        \concepttable{
            \conrow{正向导通(1)}{积累大量少子,形成贮存电荷 $Q$}
            \conrow{反偏施加(1$\to$2)}{边界少子浓度瞬间下降但仍高于 $p_{n0}$,扩散梯度维持“导通”特征,外部电流恒定为 $I_R$}
            \conrow{$t_s$ 结束(2)}{边界浓度降至 $p_{n0}$,耗尽层开始建立}
            \conrow{$t_f$(2$\to$4)}{边界处载流子抽干,耗尽层已建立,深层残余电荷靠扩散/复合消失,电流从 $I_R$ 逐渐降至 0}
            \conrow{$t_{rr}$ 影响因素}{贮存电荷量 $Q$ 越大 $t_{rr}$ 越长;抽取速度 $I_R = U_2/R$ 越大 $t_{rr}$ 越短}
        }
    }
}

\subsubsection{提高开关速度的措施}

\entry{途径一:减小贮存电荷 $Q$}{}

\noindent
\begin{minipage}[t]{0.49\linewidth}
    \blue{减小正向电流 $I_D$}
    \concepttable{
        \conrow{原理}{$I_D \downarrow \Rightarrow V_a \downarrow \Rightarrow n_{p0}e^{eV_a/kT} \downarrow$}
        \conrow{结果}{注入少子浓度降低,$Q$ 减小}
    }
\end{minipage}
\hfill
\begin{minipage}[t]{0.49\linewidth}
    \blue{降低少子寿命 $\tau$}
    \concepttable{
        \conrow{原理}{$\tau \downarrow \Rightarrow L_n = \sqrt{D_n\tau_n} \downarrow$}
        \conrow{结果}{扩散长度变短,浓度衰减加快,$Q$ 减小}
    }
\end{minipage}

\entry{途径二:加快 $Q$ 消失(\red{最有效})}{}

\noindent
\begin{minipage}[t]{0.49\linewidth}
    \blue{增大反向抽取电流}
    \concepttable{
        \conrow{方法}{使 $I_R = (U_2-V)/R$ 增大}
        \conrow{效果}{$I_R \uparrow \Rightarrow t_{rr} \downarrow$}
    }
\end{minipage}
\hfill
\begin{minipage}[t]{0.49\linewidth}
    \blue{\red{掺金工艺}}
    \concepttable{
        \conrow{机制}{Au 在禁带中引入深能级复合中心}
        \conrow{效果}{$\tau$ 大幅降低,$t_{rr}$ 可减至数十分之一}
    }
\end{minipage}

\entry{定量关系}{在 $I_D = I_F$ 条件下:突变结 $t_{rr} \approx 0.9\tau$;缓变结 $t_{rr} \approx 0.5\tau$}

\section{第三章\ MOSFET 初步}

\subsection{\red{MOS 电容}}

随表面势的不同,半导体表面可以处于积累、平带、耗尽、弱反、强反型,下面能带图为\red{P型衬底}。

\noindent
\begin{minipage}[t]{0.19\linewidth}

    \img[\linewidth]{images/image-2026-01-01-23-51-53.png}

    \entry{积累型}{在栅极施加负电压,吸引空穴到表面,形成\red{积累层。}}

\end{minipage}
\hfill
\begin{minipage}[t]{0.19\linewidth}

    \img[\linewidth]{images/image-2026-01-01-23-52-18.png}

    \entry{平带型}{在栅极施加适当电压,使\red{半导体表面电势为零},能带平坦。}

\end{minipage}
\hfill
\begin{minipage}[t]{0.19\linewidth}

    \img[\linewidth]{images/image-2026-01-02-00-05-36.png}

    \entry{耗尽型}{在栅极施加正电压,驱赶空穴离开表面,形成耗尽层,但\red{本征费米能级仍高于表面费米能级}。}

\end{minipage}
\hfill
\begin{minipage}[t]{0.19\linewidth}

    \img[\linewidth]{images/image-2026-01-02-00-07-32.png}

    \entry{弱反型}{在栅极施加更大正电压,使表面费米能级低于本征费米能级,\red{形成反型层,但未达到掺杂浓度。}}

\end{minipage}
\hfill
\begin{minipage}[t]{0.19\linewidth}

    \img[\linewidth]{images/image-2026-01-01-23-53-35.png}

    \entry{强反型}{在栅极施加足够大正电压,使沟道处\red{载流子浓度达到掺杂浓度},形成强反型层(沟道)。}

\end{minipage}

\noindent
\begin{minipage}[t]{0.49\linewidth}
    \blue{NMOS}
    \concepttable{
        \conrow{费米势}{$\phi_{fp}=V_{t}\ln\left(\frac{N_{a}}{n_{i}}\right)$}
        \conrow{表面势}{$\phi_{s}$,体内到表面的势垒}
        \conrow{耗尽层宽度}{$x_{d}=\left(\frac{2\epsilon_{s}\phi_{s}}{eN_{a}}\right)^{1/2}$}
        \conrow{反型临界表面电荷浓度}{$n_{st}=n_{i}\exp\left(\frac{\phi_{fp}}{V_{t}}\right)$}
        \conrow{功函数差}{$\phi_{ms} = \phi'_{m} - \left( \chi' + \frac{E_g}{2e} + \phi_{fp} \right)$}
        \conrow{$n^{+}$ 多晶硅}{$\phi_{ms} = - \left( \frac{E_g}{2e} + \phi_{fp} \right)$}
        \conrow{$p^{+}$ 多晶硅}{$\phi_{ms} = \left( \frac{E_g}{2e} - \phi_{fp} \right)$}
    }
\end{minipage}
\hfill
\begin{minipage}[t]{0.49\linewidth}
    \blue{PMOS}
    \concepttable{
        \conrow{费米势}{$\phi_{fn}=V_{t}\ln\left(\frac{N_{d}}{n_{i}}\right)$}
        \conrow{表面势}{$\phi_{s}$,体内到表面的势垒}
        \conrow{耗尽层宽度}{$x_{d}=\left(\frac{2\epsilon_{s}\phi_{s}}{eN_{d}}\right)^{1/2}$}
        \conrow{反型临界表面电荷浓度}{$p_{st}=n_{i}\exp\left(\frac{\phi_{fn}}{V_{t}}\right)$}
        \conrow{功函数差}{$\phi_{ms} = \phi_{m}' - \left( \chi' + \frac{E_g}{2e} - \phi_{fn} \right)$}
        \conrow{$n^{+}$ 多晶硅}{$\phi_{ms} = \left( \frac{E_g}{2e} - \phi_{fn} \right)$}
        \conrow{$p^{+}$ 多晶硅}{$\phi_{ms} = - \left( \frac{E_g}{2e} + \phi_{fn} \right)$}
    }
\end{minipage}

\subsection{平带电压、阈值电压}

主要是公式与影响因素:

\noindent
\begin{minipage}[t]{0.49\linewidth}
    \blue{NMOS}
    \concepttable{
        \conrow{平带电压}{$V_{FB} = \phi_{ms} - \frac{Q'_{ss}}{C_{\mathrm{ox}}}$}
        \conrow{最大耗尽电荷密度}{$|Q_{SD}^{\prime}(\max)| = eN_{a}x_{dT}$}
        \conrow{阈值电压}{$V_{TN} = \frac{|Q_{SD}^{\prime}(\max)|}{C_{\mathrm{ox}}} - \frac{Q_{SS}^{\prime}}{C_{\mathrm{ox}}} + \phi_{ms} + 2\phi_{fp}$\par
        $V_{TN}=\frac{\left|Q_{S D}^{\prime}(\max)\right|}{C_{\mathrm{ox}}}+V_{FB}+2\phi_{fp}$}
    }
\end{minipage}
\hfill
\begin{minipage}[t]{0.49\linewidth}
    \blue{PMOS}
    \concepttable{
        \conrow{平带电压}{$V_{FB} = \phi_{ms} - \frac{Q'_{ss}}{C_{\mathrm{ox}}}$}
        \conrow{最大耗尽电荷密度}{$|Q_{SD}^{\prime}(\max)| = eN_{d}x_{dT}$}
        \conrow{阈值电压}{$V_{TP} = -\frac{|Q_{SD}^{\prime}(\max)|}{C_{\mathrm{ox}}} - \frac{Q_{SS}^{\prime}}{C_{\mathrm{ox}}} + \phi_{ms} - 2\phi_{fn}$\par
        $V_{TP}=-\frac{\left|Q_{S D}^{\prime}(\max)\right|}{C_{\mathrm{ox}}}+V_{FB}-2\phi_{fn}$}
    }
\end{minipage}

\subsection{MOS 电容的 C-V 特性}

根据栅压和频率的不同,MOS 电容呈现不同的特性

\imgleft[0.3]{images/image-2026-01-02-14-33-15.png}{

    \concepttable{
        \conrow{积累区}{$C_{\text{acc}} = C_{\text{ox}} = \varepsilon_{\text{ox}}/t_{\text{ox}}$}
        \conrow{平带区}{$C_{\text{FB}} = \dfrac{C_{\text{ox}}}{t_{ox}+\frac{\varepsilon_{ox}}{\varepsilon_s}\sqrt{V_t\frac{\varepsilon_s}{eN_a}}}$}
        \conrow{耗尽区}{$C_{\text{depl}} = \dfrac{C_{\text{ox}}}{1 + C_{\text{ox}}/C_{SD}}$}
        \conrow{最小电容}{$C_{\min} = \dfrac{\varepsilon_{\text{ox}}}{t_{\text{ox}} + (\varepsilon_{\text{ox}}/\varepsilon_s)x_{dT}}$}
        }

    \noindent
    \begin{minipage}[t]{0.49\linewidth}
        \entry{低频强反型}{$C_{\text{inv}}^{\,\text{LF}} \approx C_{\text{ox}}$}
    \end{minipage}
    \hfill
    \begin{minipage}[t]{0.49\linewidth}
        \entry{高频强反型}{$C_{\text{inv}}^{\,\text{HF}} = C_{\min}$}
    \end{minipage}

}

\subsection{MOSFET 的工作原理}

I-V 特性:

\noindent
\begin{minipage}[t]{0.49\linewidth}
    \blue{NMOS}
    \concepttable{
        \conrow{饱和漏压}{$V_{DS(sat)} = V_{GS} - V_{TN}$}
        \conrow{线性区}{$\frac{W \mu_n C_{ox}}{2L} [2(V_{GS}-V_{T})V_{DS}]$}
        \conrow{饱和区}{$I_D = \frac{W \mu_n C_{ox}}{2L} (V_{GS} - V_{TN})^2$}
        \conrow{线性跨导}{$g_m = \frac{W \mu_n C_{ox}}{2L} V_{DS}$}
        \conrow{饱和跨导}{$g_m = \frac{W \mu_n C_{ox}}{L} (V_{GS} - V_{TN})$}
    }
\end{minipage}
\hfill
\begin{minipage}[t]{0.49\linewidth}
    \blue{PMOS}
    \concepttable{
        \conrow{饱和漏压}{$V_{SD(sat)} = V_{SG} - |V_{TP}|$}
        \conrow{线性区}{$\frac{W \mu_p C_{ox}}{2L} [2(V_{SG}-|V_{T}|)V_{SD}]$}
        \conrow{饱和区}{$I_D = \frac{W \mu_p C_{ox}}{2L} (V_{SG} - |V_{TP}|)^2$}
        \conrow{线性跨导}{$g_m = \frac{W \mu_p C_{ox}}{2L} V_{SD}$}
        \conrow{饱和跨导}{$g_m = \frac{W \mu_p C_{ox}}{L} (V_{SG} - |V_{TP}|)$}
    }
\end{minipage}

\entry{普适公式}{$I_D = \frac{W \mu_{n(p)} C_{ox}}{L} \left[ (V_{GS(SG)} - V_{TN(|V_{TP}|)}) V_{DS(SD)} - \frac{V_{DS(SD)}^2}{2} \right]$}

\entry{推导}{由高斯定理得 $-\epsilon_{ox} E_{ox} = Q'_{ss} + Q'_n + Q'_{SD(max)}$;栅压分配为 $V_{GS} - V_x = V_{ox} + \phi_{ms} + 2\phi_{fp}$,利用 $V_{ox} = E_{ox} t_{ox}$ 和 $C_{ox} = \epsilon_{ox}/t_{ox}$ 消去 $E_{ox}$,定义 $V_T = \phi_{ms} + 2\phi_{fp} + (Q'_{ss} + Q'_{SD(max)})/C_{ox}$ 得反型电荷 $|Q'_n(x)| = C_{ox}[(V_{GS} - V_x) - V_T]$;由漂移电流 $I_x = W|Q'_n|\mu_n dV_x/dx$ 沿通道积分 $\int_0^L I_D dx = \int_0^{V_{DS}} W\mu_n C_{ox}(V_{GS} - V_T - V_x)dV_x$ 得 \red{$I_D = \frac{W\mu_n C_{ox}}{L}[(V_{GS} - V_T)V_{DS} - \frac{1}{2}V_{DS}^2]$}(适用条件:$V_{GS} \ge V_T$,$0 < V_{DS} < V_{D(sat)}$)。}

\entry{PMOS转换说明}{对于P型衬底PMOS:(1) 将电压符号改为 $V_{SG}$、$V_{SD}$,阈值电压改为 $|V_{TP}|$;(2) 迁移率 $\mu_n \to \mu_p$;(3) 反型电荷为空穴 $Q'_p$;(4) 电流方向从源极到漏极,公式形式不变:$I_D = \frac{W\mu_p C_{ox}}{L}[(V_{SG} - |V_{TP}|)V_{SD} - \frac{1}{2}V_{SD}^2]$。}

\subsubsection{截止频率}

截止频率 $f_T$ 是电流增益为 1 时的频率。\red{$f_T = \frac{g_m}{2\pi (C_{gsT} + C_M)} = \frac{g_m}{2\pi C_G}$}

在理想饱和区,\red{$f_T = \frac{\mu_n}{2\pi L^2} (V_{GS} - V_T)$},\textbf{提高频率特性的途径:}

\noindent
\begin{minipage}[t]{0.49\linewidth}
    \blue{提高迁移率 $\mu_n$}
    \concepttable{
        \conrow{优化晶向}{选择高迁移率晶向(如硅的100方向)}
        \conrow{新材料}{使用 GaAs 等高迁移率材料}
    }
\end{minipage}
\hfill
\begin{minipage}[t]{0.49\linewidth}
    \blue{缩短沟道长度 $L$}
    \concepttable{
        \conrow{效果}{$f_T \propto 1/L^2$,\red{最有效方法}}
        \conrow{双重收益}{减小寄生电容 $C_{gs}$;增大跨导 $g_m$}
    }
\end{minipage}

\section{第四章\ MOSFET 深入}

\subsection{非理想效应}

这里的图也需要记一记,可能没有那么多空来画。

\noindent
\begin{minipage}[t]{0.49\linewidth}
    \hlgreen{\textbf{亚阈值电导}}

    \entry{定义}{在弱反型($\phi_{fp} \leq \phi_s \leq 2\phi_{fp}$)中,电流$I_D$并没有截止,而是呈指数衰减。}

    \imgleft[0.3]{images/image-2026-01-03-10-14-50.png}{
        \entry{理想与实际过渡区对比}{在 $V_T$ 以下,电流平滑过渡,存在"尾巴",即亚阈值电流。}

        \entry{物理机理}{弱反型势垒较低,根据玻尔兹曼分布,源区总有一部分高能量电子有概率越过势垒。此时电流的主要驱动机制是扩散,而非漂移}

    }

    \entry{I-V特性影响}{
        $I_{D(sub)} \propto \left[ \exp \left( \frac{eV_{GS}}{kT} \right) \right] \cdot \left[ 1 - \exp \left( \frac{-eV_{DS}}{kT} \right) \right]$,$V_{DS}$过大时,$I_{D(sub)}$趋于饱和。
    }


\end{minipage}
\hfill
\begin{minipage}[t]{0.49\linewidth}
    \hlgreen{\textbf{沟道长度调制}}

    \entry{定义}{饱和区,过剩的电压 $V_{DS} - V_{DS(sat)}$ 会导致夹断点向源极方向移动。}

    \imgleft[0.3]{images/image-2026-01-03-10-30-27.png}{

        $V_{DS} \uparrow \Rightarrow \Delta V_{DS} \uparrow \Rightarrow \Delta L \uparrow \Rightarrow L' = L - \Delta L \downarrow \Rightarrow I_D \uparrow$

        \red{$I'_D = \frac{L}{L - \Delta L} I_{D(sat)}$}
    
    }

    % 逻辑链符号化总结
    
    $N_A \downarrow \;\Longrightarrow\; x_d \uparrow \;\Longrightarrow\; \Delta L \uparrow \;\Longrightarrow\; L \downarrow \;\Longrightarrow\; \frac{\Delta L}{L} \uparrow \;\Longrightarrow\; \text{沟调效应增强} \quad \text{抑制方法:}\quad N_A \uparrow \text{ 或 } L \uparrow$

    \hlgreen{\textbf{迁移率变化}}
    \entry{定义}{

    $V_{GS} \uparrow \;\Longrightarrow\; E_{\text{纵}} \uparrow \;\Longrightarrow\; \text{载流子靠近界面} \;\Longrightarrow\; \text{表面散射增强} \;\Longrightarrow\; \mu_{\text{eff}} = \frac{\mu_0}{1 + \theta[V_{GS} - V_T(x)]}$

    }

\end{minipage}

\noindent
\begin{minipage}[t]{0.49\linewidth}

    \hlgreen{\textbf{速度饱和}}

    \entry{定义}{饱和漂移速度 $v_{sat}$,漏源电流提前饱和。\red{实际的饱和电压小于理想值}}

    $I_{D(sat)} = W C_{ox} (V_{GS} - V_T) v_{sat}$(成线性关系了)$f_T = \frac{v_{sat}}{2\pi L}$ 

\end{minipage}
\hfill
\begin{minipage}[t]{0.49\linewidth}

    \hlgreen{\textbf{弹道输运}}

    \entry{定义}{由于\red{沟道长度非常短($L <$ 散射平均自由程)},载流子在沟道内几乎没有散射,直接从源极到达漏极,速度极快(主要出现在先进制程的短沟道器件中)}

\end{minipage}

\hlyellow{\textbf{按比例缩小}}
\par
\hlgreen{\textbf{完全按比例缩小}}
尺寸与电压按同样比例缩小,电场强度保持不变,\red{最为理想,但难以实现}

\noindent
\begin{minipage}[t]{0.49\linewidth}
$W',L',t_{ox}',x_D' = kW, kL, kt_{ox}, kx_D$
\end{minipage}
\hfill
\begin{minipage}[t]{0.49\linewidth}
$V_{DS}',V_{GS}',V_T' = kV_{DS}, kV_{GS}, kV_T$
\end{minipage}

\noindent
\begin{minipage}[t]{0.32\linewidth}
\entry{掺杂调整}{$N_A' = N_A/k$ }
\entry{电流}{$I_D' = kI_D$}
\end{minipage}
\hfill
\begin{minipage}[t]{0.32\linewidth}
\entry{功率}{$P' = k^2P$}
\entry{电阻}{$R' = R$}\entry{电容}{$C_{ox}' = kC_{ox}$}
\end{minipage}
\hfill
\begin{minipage}[t]{0.32\linewidth}
\entry{延迟}{$\tau' = k\tau$}
\entry{功率密度}{$P'' = P$}
\end{minipage}

\entry{阈值电压不按比例缩小}{
    \concepttable{
        \conrow{原因}{$\phi_{fp} = V_t \ln(N_A/n_i) \approx \text{const}$,$\phi_{ms} \approx \text{const}$}
        \conrow{实际}{$V_T' \approx V_T \neq kV_T$}
        \conrow{后果}{$V_{DD} \downarrow \implies (V_{GS}-V_T) \downarrow \implies I_D, f_T \downarrow$}
    }
}

\noindent
\begin{minipage}[t]{0.49\linewidth}
    \hlgreen{\textbf{恒压按比例缩小}}
    \textbf{尺寸缩小}:$L, W$ 缩小(按 $k$)。\textbf{电压不变}:$V_{DD}$ 保持。
    \entry{后果}{电场增强:$E = V/L$。$V$ 不变,$L$ 减小 $\to$ \red{$E$ 剧增}(温度升高,乃至于击穿器件)}
\end{minipage}
\hfill
\begin{minipage}[t]{0.49\linewidth}
    \hlgreen{\textbf{一般化按比例缩小}}
    \textbf{尺寸}:按比例因子 $k$ 缩小。\textbf{电场}:按另一个因子缩小(电压稍微降低一点,但降得没尺寸那么快)。\textbf{目的}:在保证可靠性和性能之间寻找平衡
\end{minipage}

\hlyellow{\textbf{阈值电压修正}}
\par
\hlgreen{\textbf{短沟道效应}}
源和漏的 N+ 掺杂,与 P 型衬底之间会形成耗尽区。源和漏的电场会“协助”耗尽沟道两端的区域,使栅极需要耗尽的一部分电荷被源和漏分担了,受\red{栅极控制的耗尽层形状成为梯形}(即 $L'$ 的区域)。

\noindent
\begin{minipage}[t]{0.49\linewidth}

    \vspace{-10pt}
    \begin{minipage}[t]{0.49\linewidth}
        \img{images/image-2026-01-03-11-55-02.png}
    \end{minipage}
    \hfill
    \begin{minipage}[t]{0.49\linewidth}
        \img{images/image-2026-01-03-11-57-28.png}
    \end{minipage}
    
\end{minipage}
\hfill
\begin{minipage}[t]{0.49\linewidth}

$\Delta L = r_j (\sqrt{1 + \frac{2x_{dT}}{r_j}} - 1)$

$\Delta V_T = - \frac{e N_A x_{dT}}{C_{ox}} [\frac{\Delta L}{L}] < 0$

$r_j$ 为结深度,$L$ 和$r_j$ 同量级时,短沟道效应显著。

\end{minipage}

\hlgreen{\textbf{窄沟道效应}}
当沟道变窄,源/漏结及沟道边缘的\red{耗尽区会向沟道中心延伸},在沟道宽度的两侧存在附加的空间电荷区;这些附加的电荷也受栅压控制,栅极要使剩余的硅区反型,就需要施加更高的栅电压。因此,阈值电压增大。

\noindent
\begin{minipage}[t]{0.49\linewidth}

    \vspace{-6pt}
    \begin{minipage}[t]{0.49\linewidth}
        \img{images/image-2026-01-03-12-17-11.png}
    \end{minipage}
    \hfill
    \begin{minipage}[t]{0.49\linewidth}
        \img{images/image-2026-01-03-12-10-55.png}
    \end{minipage}
    
\end{minipage}
\hfill
\begin{minipage}[t]{0.49\linewidth}

$\Delta V_T = V_T(\text{窄}) - V_T(\text{宽})$

$= \frac{e N_A x_{dT}}{C_{ox}} \cdot \frac{\xi x_{dT}}{W} > 0$

$\xi$是几何因子,当 $W$ 和$x_{dT}$ 同量级时,窄沟道效应显著。

\end{minipage}

\hlgreen{\textbf{离子注入效应}}
离子注入主要改变的是半导体表面的杂质浓度,进而改变耗尽层内的空间电荷密度 $|Q'_{SD(max)}|$

$V_T = V_{T0} \pm \frac{e D_I}{C_{ox}},+\text{为同性掺杂},-\text{为异性掺杂}$

\section{第五章\ 双极型晶体管}

\subsection{工作原理}

少子分布、能带图:

\noindent
\begin{minipage}[t]{0.24\linewidth}
    \blue{正向有源区}
    
    \vspace{2pt}
    \noindent
    \begin{minipage}[t]{0.48\linewidth}
        \blue{NPN 型}

        \img[\linewidth]{images/image-2026-01-02-19-28-20.png}

    \end{minipage}
    \hfill
    \begin{minipage}[t]{0.48\linewidth}
        \blue{PNP 型}
    \end{minipage}

\end{minipage}
\hfill
\begin{minipage}[t]{0.24\linewidth}
    \blue{饱和区}

    \vspace{2pt}
    \noindent
    \begin{minipage}[t]{0.48\linewidth}
        \blue{NPN 型}

        \img[\linewidth]{images/image-2026-01-02-19-29-32.png}

    \end{minipage}
    \hfill
    \begin{minipage}[t]{0.48\linewidth}
        \blue{PNP 型}
    \end{minipage}
\end{minipage}
\hfill
\begin{minipage}[t]{0.24\linewidth}
    \blue{截止区}

    \vspace{2pt}
    \noindent
    \begin{minipage}[t]{0.48\linewidth}
        \blue{NPN 型}

        \img[\linewidth]{images/image-2026-01-02-19-29-57.png}

    \end{minipage}
    \hfill
    \begin{minipage}[t]{0.48\linewidth}
        \blue{PNP 型}
    \end{minipage}
\end{minipage}
\hfill
\begin{minipage}[t]{0.24\linewidth}
    \blue{反向有源区}

    \vspace{2pt}
    \noindent
    \begin{minipage}[t]{0.48\linewidth}
        \blue{NPN 型}

        \img[\linewidth]{images/image-2026-01-02-19-30-42.png}

    \end{minipage}
    \hfill
    \begin{minipage}[t]{0.48\linewidth}
        \blue{PNP 型}
    \end{minipage}
\end{minipage}

理想情况下,\red{集电结边界的少子的浓度为零},希望从发射区注入的电子能越过基区扩散到集电结的空间电荷区,尽可能多的电子被集电极收集,而不是在基区复合,因此需要\red{基区的宽度与扩散长度相比很小}。

BJT 有共射、共基、共集三种接法,为了使三极管处于正向有源区,从而实现正常的电流放大作用,\red{必须同时满足以下两个条件}:

\noindent
\begin{minipage}[t]{0.49\linewidth}

\entry{\blue{发射结正向偏置}}{降低发射结势垒,\red{使发射区的高浓度多子(NPN是电子)能够顺利注入到基区。}}

\end{minipage}
\hfill
\begin{minipage}[t]{0.49\linewidth}

\entry{\blue{集电结反向偏置}}{在集电结建立较强的电场(耗尽层加宽),有利于\red{收集从基区扩散过来的少子}。形成集电极电流 $I_C$。}

\end{minipage}

\subsection{低频共基极电流增益}

\subsubsection{电流成分划分}

以 NPN 管为例,$J_{nE}$ 是形成集电极电流的基础,$J_{pE}$、$J_{RB}$ 等为损耗,理解各分量的来源与相互关系

\imgleft[0.4]{images/image-2026-01-05-16-13-37.png}{

    \concepttable{
        \conrow{$J_{nE}$}{BE 结正向注入的电子电流,主要贡献集电极电流}
        \conrow{$J_{pE}$}{BE 结反向注入的空穴电流}
        \conrow{$J_R$}{基区复合电流}
        \conrow{$J_{nC}$}{BC 结反向注入的电子电流}
    }

}

电流由载流子浓度梯度驱动。\red{基区越薄($W_B$ 越小),浓度梯度越大,扩散越快且复合越少,$J_{nC}$ 越接近 $J_{nE}$}

\subsubsection{直流共基极电流增益}

\entry{定义}{$\alpha_0 = I_C/I_E = \frac{J_{nC} + J_{G} + J_{pC0}}{J_{nE} + J_{R} + J_{pE}}$,即集电极电流与发射极电流之比。}

在交流小信号下,\red{常量漏电流($J_G, J_{pc0}$)不随电压变化,因此被微分滤除},只剩下与信号相关的项,即下面的样子:

\red{$\alpha=\frac{J_{n E}}{J_{nE}+J_{pE}}\cdot\frac{J_{nC}}{J_{nE}}\cdot\frac{J_{nE}+J_{pE}}{J_{nE}+J_{R}+J_{pE}} \quad \alpha=\gamma\alpha_{T}\delta$}

其中:

\concepttable{
    \conrow{$\gamma$(发射极注入效率)}{$\gamma = \dfrac{J_{nE}}{J_{nE}+J_{pE}}$;衡量发射极发射出的电流中有用电子的比例}
    \conrow{$\alpha_T$(基区输运系数)}{$\alpha_T = \dfrac{J_{nC}}{J_{nE}}$;注入基区的电子在穿过基区后\red{被集电极收集的比例},损失为基区复合 $J_{RB}$}
    \conrow{$\delta$(复合系数)}{$\delta = \dfrac{J_{nE}+J_{pE}}{J_{nE}+J_R+J_{pE}}$;反映发射结耗尽区复合 $J_R$ 的影响,现代工艺或简化模型常取 $\delta \approx 1$}
}

\entry{工程意义}{
    三因子分别对应三个主要损耗:发射端电流纯度($\gamma$)、基区复合($\alpha_T$)和耗尽区复合($\delta$)。提高 $\alpha$ 的方法即针对这三项减小损耗。
}

\hlblue{\textbf{$\gamma$(发射极注入效率)}}

\entry{推导要点}{解扩散方程得到边界过剩载流子分布,代入 $J=-eD\,dn/dx$ 可得 $J_{pE}\propto D_E p_{E0}/L_E\cdot\tanh(x_E/L_E)$ 和 $J_{nE}\propto D_B n_{B0}/L_B\cdot\tanh(x_B/L_B)$。在正向偏置下 $e^{eV_{BE}/kT}\gg1$,因此可写为:}

$\gamma=\dfrac{1}{1+J_{pE}/J_{nE}}$

\entry{计算公式}{

    \concepttable{
        \conrow{\red{表达式}}{
            在 $x_B\ll L_B,\;x_E\ll L_E$时,\red{$\displaystyle \gamma\approx\dfrac{1}{\,1+\dfrac{N_B}{N_E}\dfrac{D_E}{D_B}\dfrac{x_B}{x_E}\,}$}
        }
        \conrow{参数}{所有参数都会在题干中给出。}
        \conrow{设计规律}{
            增大 $N_E\gg N_B$(重掺杂发射区)、减小基宽 $x_B$、减小发射区有效厚度 $x_E$,以使 $\gamma\to1$。}
    }

}

\hlblue{\textbf{$\alpha_T$(基区输运系数)}}

\entry{推导概要}{解基区少子稳态扩散方程并在两端利用 $J=-eD_B\,d(\delta n_B)/dx$ 求得 $J_{nE}$($x=0$)和 $J_{nC}$($x=x_B$),两者相除可消去公因子,结果仅含双曲函数项。}

\entry{计算公式}{

    \concepttable{
        \conrow{\red{表达式}}{当 $x_B\ll L_B$ 时,\red{$\displaystyle \alpha_T \approx 1-\tfrac{1}{2}\left(\dfrac{x_B}{L_B}\right)^2$}}
        \conrow{参数}{所有参数都会在题干中给出。}
        \conrow{设计规律}{为使 $\alpha_T\to1$,要求 \red{$x_B\ll L_B$},采用薄基区/长扩散长度的工艺设计。}
}

}

\hlblue{\textbf{$\delta$(复合系数)}}

\entry{计算公式}{
    \concepttable{
        \conrow{\red{表达式}}{\red{$\displaystyle \delta=\frac{1}{1+\dfrac{J_{R0}}{J_{S0}}\exp\!\left(-\dfrac{eV_{BE}}{2kT}\right)}$}}
        \conrow{参数}{$J_{R0}$题干里会给,$J_{s0}=\frac{e D_{B}n_{B0}}{L_{B}\tanh\left(x_{B} /L_{B}\right)}$}
        \conrow{设计规律}{随着 $V_{BE}$ 增大,指数项迅速减小,$\delta\to1$;低 $V_{BE}$ 时 $J_R$ 占优,$\delta$ 较小,导致增益下降。}
    }
}

\subsection{非理想效应}

\subsubsection{厄利效应}

\entry{结论}{基区越窄、掺杂越低,Early 效应越显著($V_A$ 越小)。}


\imgleft[0.3]{images/image-2026-01-05-17-10-31.png}{

    \entry{物理机制}{
        \concepttable{
            \conrow{耗尽层扩展}{$V_{CB}$ 增大 $\to$ 集电结耗尽区向基区扩展}
            \conrow{基区变窄}{耗尽层侵入基区,导致有效中性基区宽度 $x_B$ 减小}
            \conrow{浓度梯度增大}{$J_n \propto D_n\frac{dn}{dx}\approx D_n\frac{n_p(0)}{x_B}$,$x_B\downarrow\Rightarrow I_C\uparrow$}
        }
    }

}

\entry{输出特性影响}{
    \concepttable{
        \conrow{曲线倾斜}{$I_C$ 与 $V_{CE}$ 开始相关,输出电阻为有限值 $r_o$}
        \conrow{Early 电压}{$V_A$:曲线与 X 轴的截距。}
        \conrow{电流修正}{$I_C = I_{C(\text{ideal})}\,(1 + V_{CE}/V_A)$}
        \conrow{输出导纳}{$g_o = \dfrac{dI_C}{dV_{CE}} \approx \dfrac{I_C}{V_A}$}
    }
}

\subsubsection{大注入效应}

\entry{定义}{大注入时 $n_p(0)$ (注入少子浓度)可接近或超过 $N_A$(基区多子浓度)。}

\imgleft[0.23]{images/image-2026-01-05-17-16-39.png}{

    \entry{物理机理}{
        \concepttable{
            \conrow{电中性约束}{大量注入电子需伴随空穴增加以维持电中性,基区边界处 $p_p(0)$ 不再恒等于 $N_A$,而\red{随注入显著升高}。}
            \conrow{对发射极注入效率的影响}{$\gamma=\dfrac{1}{1+J_{pE}/J_{nE}}$。由于 $p_p(0)$ 增大,反向空穴流 $J_{pE}$ 增幅通常大于 $J_{nE}$ 的增幅,\red{导致 $\gamma$ 下降},进而使电流增益下降。}
        }
    }

}

\entry{电流特性曲线}{}

\imgleft[0.3]{images/image-2026-01-05-19-12-01.png}{

    电流曲线和 MOSFET 一样的大注入特性。

    \entry{电流增益 $\beta$ 曲线}{
        \concepttable{
            \conrow{低电流区}{$\beta$ 增大,发射结耗尽区复合($J_R$)占优时随 $I_C$ 提高而改善}
            \conrow{中电流区}{$\beta$ 到达峰值并基本平坦,是器件的\red{理想工作区}。}
            \conrow{高电流区}{$\beta$ 开始下降,\red{主要由大注入引起的 $\gamma$ 降低}(反向空穴注入 $J_{pE}$ 增加)以及串联电阻/高注入效应等共同作用导致。}
        }
    }
}



\subsection{等效电路模型}

\subsection{频率上限}

\subsection{大信号开关}


\section{第六章\ 金半接触}

\subsection{肖特基接触}

\hlgreen{\textbf{基本概念}}
\red{电子将从功函数小的地方跑到功函数大的地方,空穴则相反。}

\noindent
\begin{minipage}[t]{0.49\linewidth}
    \entry{整流接触}{
    \concepttable{
        \conrow{定义}{在半导体表面形成了表面势垒,也称为阻挡层。}
        \conrow{特性}{类似于PN结,具有单向导电性(整流作用)。}
        \conrow{命名}{这就是我们通常所说的\red{肖特基接触}。}
    }   
}
\end{minipage}
\hfill
\begin{minipage}[t]{0.49\linewidth}
\entry{非整流接触}{
    \concepttable{
        \conrow{定义}{在界面处形成了\red{反阻挡层},即高电导区。}
        \conrow{特性}{没有整流作用,电流可以双向自由流动。}
        \conrow{命名}{这就是我们通常所说的\red{欧姆接触}。}
    }
}
\end{minipage}

\imgleft[0.4]{images/image-2025-12-30-12-07-58.png}{
    \concepttable{
        \conrow{功函数 $\phi$}{电子从 $E_F$ 逸出到真空能级 $E_0$ 所需最小能量。}
        \conrow{金属功函数}{$e\phi_m = E_0 - (E_F)_m$}
        \conrow{半导体功函数}{$e\phi_s = E_0 - (E_F)_s$}
        \conrow{电子亲和能 $\chi$}{导带底 $E_c$ 电子逸出到真空能级 $E_0$ 所需能量。}
        \conrow{\red{能量关系}}{\red{$e\phi_s = \chi + \phi_n$},其中 $\phi_n = E_c - E_F$。$\chi$ 是固有属性,$\phi_s$ 会随掺杂改变(因为费米能级会因掺杂浓度变化)}
    }
}

\noindent
\begin{minipage}[t]{0.49\linewidth}

\img{images/image-2025-12-30-12-30-52.png}

\blue{N型肖特基接触 ($\phi_m > \phi_s$)}
\concepttable{
    \conrow{初始条件}{$\phi_m > \phi_s \implies E_{Fm} < E_{FN}$}
    \conrow{物理过程}{电子自发从 $N$ 型半导体流向金属}
    \conrow{电荷分布}{半导体侧施主失去电子带正电,形成\red{耗尽层};金属侧带负电}
    \conrow{能带弯曲}{表面 $n_s$ 降低,由 $n = N_c \exp[-(E_c - E_F)/kT]$ 知 $E_c$ \red{向上}弯曲}
    \conrow{势垒形成}{形成表面势垒 $e\phi_{B0}$(阻挡层),阻碍电子进入金属}
    \conrow{平衡状态}{热平衡建立,系统费米能级 $E_F$ 处处拉平}
}

\end{minipage}
\hfill
\begin{minipage}[t]{0.49\linewidth}

\img{images/image-2025-12-30-12-30-21.png}
\vspace{2pt}
\blue{P型肖特基接触 ($\phi_s > \phi_m$)}
\concepttable{
    \conrow{初始条件}{$\phi_s > \phi_m \implies E_{Fm} > E_{FP}$}
    \conrow{物理过程}{电子从金属流向半导体 (空穴从 $P$ 型流向金属)}
    \conrow{电荷分布}{半导体侧受主得到电子带负电,形成\red{耗尽层};金属侧带正电}
    \conrow{能带弯曲}{表面 $p_s$ 降低,能带 ($E_c, E_v$) \red{向下}弯曲}
    \conrow{势垒形成}{形成表面势垒 $e\phi_{B0}$(阻挡层),阻碍空穴进入金属}
    \conrow{平衡状态}{热平衡建立,系统费米能级 $E_F$ 处处拉平}
}

\end{minipage}

\textbf{施加偏压(以 N 型接触为例):}

\noindent
\begin{minipage}[t]{0.49\linewidth}
\img[0.5\linewidth]{images/image-2025-12-30-12-36-22.png}
\blue{正向偏压 (Metal +, Semi -)}
\concepttable{
    \conrow{势垒变化}{外加电压 $U$ 抵消内建电势,势垒降低为 $e(V_{bi} - U)$}
    \conrow{物理过程}{电子易于越过势垒从半导体流向金属}
    \conrow{电流特性}{产生巨大的正向电流}
}
\end{minipage}
\hfill
\begin{minipage}[t]{0.49\linewidth}
\img[0.6\linewidth]{images/image-2025-12-30-12-46-10.png}
\blue{反向偏压 (Metal -, Semi +)}
\concepttable{
    \conrow{势垒变化}{外加电压 $U$ 叠加在内建电势上,势垒增加为 $e(V_{bi} + U)$}
    \conrow{物理过程}{半导体侧电子无法越过更高的势垒}
    \conrow{电流特性}{金属侧电子受限于固定势垒 $e\phi_{B0}$,电流极小,反向截止}
}
\end{minipage}

\noindent
\begin{minipage}[t]{0.49\linewidth}
    \blue{N 型计算}
    \concepttable{
        \conrow{费米势}{$\phi_n = V_t \ln(\frac{N_c}{N_d})$}
        \conrow{肖特基势垒}{$e\phi_{\mathrm{B0}} = e\phi_{\mathrm{m}} - \chi$}
        \conrow{内建电势}{$V_{bi} = \phi_{\mathrm{m}} - \phi_{\mathrm{s}} = \phi_{\mathrm{B0}} - \phi_n$}
    }
\end{minipage}
\hfill
\begin{minipage}[t]{0.49\linewidth}
    \blue{P 型计算}
    \concepttable{
        \conrow{费米势}{$\phi_p = V_t \ln(\frac{N_v}{N_a})$}
        \conrow{肖特基势垒}{$e\phi_{\mathrm{B0}} = E_g - (e\phi_{\mathrm{m}} - \chi)$}
        \conrow{内建电势}{$V_{bi} = \phi_{\mathrm{s}} - \phi_{\mathrm{m}} = \phi_{\mathrm{B0}} - \phi_p$}
    }
\end{minipage}

\blue{通用特性 ($N$ 代表 $N_d$ 或 $N_a$)}
\concepttable{
    \conrow{参数说明}{$\phi_{\mathrm{B0}}$ 一般题干\blue{直接给值},否则按上表计算。}
    \conrow{耗尽层宽度}{$W(x_n)=\left[\frac{2\varepsilon\left(V_{bi}+V_{\mathrm{R}}\right)}{e N}\right]^{1 /2}$ (与单边突变结一致),其中 $V_R$ 为外加反向偏压。}
    \conrow{最大电场}{$E_{max} = \frac{e N W}{\varepsilon}$}
    \conrow{势垒电容}{$C= A \frac{\varepsilon}{W}=A\left[\frac{e \varepsilon N}{2\left(V_{bi}+V_{R}\right)}\right]^{1 / 2}$}
    \conrow{$C-V$ 特性}{$\frac{1}{C^{2}}=\frac{2}{e \varepsilon N A^{2}}\left(V_{R}+V_{bi}\right)$,可由曲线斜率求 $N$,截距求 $V_{bi}$}
}

\entry{整流特性}{正偏时半导体侧势垒降低,电流大;反偏时势垒升高,电流极小。由于金属电子浓度极高,金属侧势垒 $q\phi_{\mathrm{b}}$ 随偏压几乎不变。}

\subsubsection{非理想因素}

从这里开始讨论非理想因素,即为什么实际势垒不完全等于 $\phi_m - \chi$。

\noindent
\begin{minipage}[t]{0.49\linewidth}
    \entry{肖特基效应 (\red{镜像力降低})}{}
    \concepttable{
        \conrow{势能修正}{和大物一样,靠近金属的电荷会感应出镜像电荷,引入负电势能项 $-\frac{e}{16\pi\epsilon_s x}$,与电场叠加。}
        \conrow{\red{势垒降低}}{$\Delta \phi = \sqrt{\frac{eE}{4\pi\epsilon_s}}$}
        \conrow{总势能最高点}{$x_m = \sqrt{\frac{e}{16\pi \epsilon_s E}}$}
    }
\end{minipage}
\hfill
\begin{minipage}[t]{0.49\linewidth}
    \entry{界面态 (\red{费米能级钉扎})}{}
    \concepttable{
        \conrow{表面态}{禁带中由缺陷等引起的能级。施主型(失电子正电)、受主型(得电子负电)。}
        \conrow{中性能级}{$E_F < \phi_0$ 呈正电,$E_F > \phi_0$ 呈负电。}
        \conrow{物理机制}{若 $D_{it}$ 很大,表面态储存大量电荷,使 $E_F$ 被“钉扎”在 $\phi_0$ 附近,势垒高度几乎与 $\phi_m$ 无关。就是一个经验值了。}
    }
\end{minipage}

\subsubsection{电流-电压关系}

\entry{热电子发射理论}{}
\concepttable{
    \conrow{适用范围}{描述肖特基接触电流传输的主流模型(适用于 Si, GaAs 等高迁移率半导体)。}
    \conrow{核心假设}{只有\red{能量足够高}($E > E_F + e\phi_{Bn}$)($\phi_{Bn} = \phi_{B0} - \Delta \phi$,是修正后的肖特基势垒)的“热电子”才能从半导体进入金属,电流的大小取决于单位时间内能够“跳过”势垒高度的电子数量。}
}

\entry{电流分量分析}{
    \concepttable{
        \conrow{$J_{s \to m}$}{半导体 $\to$ 金属:电子需克服势垒 $e(V_{bi} - V_a)$。正偏时势垒降低,电流\red{指数级增加}。\par
        $J_{s \to m} = A^* T^2 \exp\left(\frac{-e\phi_{Bn}}{kT}\right) \exp\left(\frac{eV_a}{kT}\right)$}
        \conrow{$J_{m \to s}$}{金属 $\to$ 半导体:电子需克服势垒 $e\phi_{B0}$。势垒固定,此分量视为\red{常数}(反向饱和电流)。\par
        $J_{m \to s} = -A^* T^2 \exp\left(\frac{-e\phi_{Bn}}{kT}\right)$}
    }
}

\entry{有效理查德森常数 $A^*$}{
    \concepttable{
        \conrow{表达式}{$A^* = \frac{4\pi e m_n^* k^2}{h^3}$}
        \conrow{物理意义}{在理查德森常数中用有效质量 $m^*$ 代替 $m_0$,反映了晶格势场对电子运动的影响。}
    }
}

\imgleft[0.25]{images/image-2025-12-30-14-08-58.png}{
    \entry{肖特基二极管方程}{
        \concepttable{
            \conrow{总电流密度}{$J = J_{s \to m} + J_{m \to s} = J_{ST} \left[ \exp\left(\frac{eV_a}{kT}\right) - 1 \right]$}
            \conrow{饱和电流密度}{$J_{ST} = A^* T^2 \exp\left(\frac{-e\phi_{Bn}}{kT}\right)$\par
            $= A^* T^2 \exp\left(\frac{-e\phi_{B0}}{kT}\right)\exp\left(\frac{e\Delta \phi}{kT}\right)$\par
            因此,$J_{ST} \propto \exp\left(\frac{e\Delta \phi}{kT}\right)$}
        }
    }
}

\subsubsection{肖特基二极管与 PN 结对比}

从电流输运机制和数量级两个维度,对比了两种二极管的特性

\noindent
\begin{minipage}[t]{0.49\linewidth}
    \blue{肖特基二极管 (SBD)}
    \concepttable{
        \conrow{载流子类型}{\red{多子器件}}
        \conrow{电流机制}{热电子发射理论}
        \conrow{反向电流}{$J_{ST}$ 较大,随电压增加而增加 (非饱和)}
        \conrow{导通电压}{低 (约 0.3 V)}
        \conrow{开关速度}{\red{极快},无少子存储效应,仅受 $RC$ 限制}
        \conrow{应用}{高频检波、高速开关、肖特基箝位}
    }
\end{minipage}
\hfill
\begin{minipage}[t]{0.49\linewidth}
    \blue{PN 结二极管}
    \concepttable{
        \conrow{载流子类型}{少子器件}
        \conrow{电流机制}{少子扩散理论}
        \conrow{反向电流}{$J_S$ 极小,具有良好的饱和特性}
        \conrow{导通电压}{高 (约 0.7 V)}
        \conrow{开关速度}{较慢,存在\red{电荷存储效应}和反向恢复时间}
        \conrow{应用}{整流、稳压、一般逻辑电路}
    }
\end{minipage}

\subsection{欧姆接触}

由于表面态的存在,欧姆接触只是一个\red{理想化模型}。

\entry{反阻挡层}{通常Schottky接触形成耗尽层起阻挡作用,而此处形成积累层,电导率极高,不仅不阻挡电流反而比体内更利于导电,故称“反”阻挡层。}

\noindent
\begin{minipage}[t]{0.49\linewidth}
    \img{images/image-2025-12-29-23-38-28.png}
    \blue{N型 ($\phi_m < \phi_s$)}
    \concepttable{
        \conrow{形成条件}{$E_{Fm} > E_{FN}$}
        \conrow{载流子输运}{电子 $M \to S$}
        \conrow{弯曲}{能带\red{向下}弯曲}
        \conrow{表面}{积累层 ($n_s \gg n_0$)}
    }
\end{minipage}
\hfill
\begin{minipage}[t]{0.49\linewidth}
    \img{images/image-2025-12-29-23-38-59.png}
    \blue{P型 ($\phi_m > \phi_s$)}
    \concepttable{
        \conrow{形成条件}{$E_{Fm} < E_{FP}$}
        \conrow{载流子输运}{空穴 $M \to S$}
        \conrow{弯曲}{能带\red{向上}弯曲}
        \conrow{表面}{积累层 ($p_s \gg p_0$)}
    }
\end{minipage}

\entry{结论}{只要接触使半导体表面的\red{多数载流子浓度增加}(形成积累层),就能实现欧姆接触。}

\entry{施加偏压能带图}{高电势一侧能带\red{向下}弯曲,低电势一侧能带\red{向上}弯曲。}

\noindent
\begin{minipage}[t]{0.49\linewidth}
    \center
    \img[0.5\linewidth]{images/image-2025-12-30-11-53-53.png}
    \textbf{给半导体一侧施加负电压(N 沟道)}
\end{minipage}
\hfill
\begin{minipage}[t]{0.49\linewidth}
    \center
    \img[0.5\linewidth]{images/image-2025-12-30-11-56-22.png}
    \textbf{给半导体一侧施加正电压(N 沟道)}
\end{minipage}

\subsection{异质结基本知识}



\section{第六章\ 结型场效应晶体管}

\subsection{基本概念}

\red{多子器件},栅电压没有关断沟道时,漏源电压在沟道区产生电场,\red{使沟道中的多子通过漂移运动从源极流向漏极},形成电流。通过控制栅电压到适当电压值使沟道处于耗尽状态,达到晶体管关断。\red{分为 pn 结管和 MES 管}。

\noindent
\begin{minipage}[t]{0.49\linewidth}

    \noindent
    \begin{minipage}[t]{0.49\linewidth}
        \img[0.6\linewidth]{images/image-2026-01-03-15-56-08.png}
    \end{minipage}
    \hfill
    \noindent
    \begin{minipage}[t]{0.49\linewidth}
        \img{images/image-2026-01-03-15-54-33.png}
    \end{minipage}

\end{minipage}
\hfill
\noindent
\begin{minipage}[t]{0.49\linewidth}

    \blue{$V_{GS}$对沟道的影响:}不管是什么沟道,$V_{GS}$都是负责加宽耗尽层的,耗尽层越宽,电阻越大

    \blue{$V_{DS}$对沟道的影响:}主要考虑夹断作用,$V_{DS}$负责加宽漏极区域耗尽层,耗尽层越宽,宽到一定程度时,沟道被夹断,电流饱和。再之后是击穿。$V_{GS}<0$时,饱和电压和击穿电压都会降低。

\end{minipage}

\noindent
\begin{minipage}[t]{0.49\linewidth}

    \concepttable{
        \conrow{高频高速}{多子导电 $\to$ 无少子存储 $\to C_{diff} \approx 0$}
        \conrow{高输入阻抗}{$R_{in} \gg R_{in(BJT)}$ (电压控制)}
        \conrow{强抗辐射}{多子器件 $\to$ 不受少子寿命 $\tau$ 影响}
    }

\end{minipage}
\hfill
\begin{minipage}[t]{0.49\linewidth}

    \blue{MESFET:}用金属代替了 P 型半导体的地位,行成肖结,也是扩张耗尽层达到控制目的。\textbf{沟道}:n-GaAs外延层 (高电子迁移率)。\red{栅极(G) = 肖特基接触 (控制);源/漏(S/D) = 欧姆接触}。分为\red{增强型和耗尽型}两种,可能会考画图。

\end{minipage}

\hlyellow{\textbf{器件特性}}

\noindent
\begin{minipage}[t]{0.59\linewidth}

\hlgreen{\textbf{pnJFET}}

    \noindent
    \begin{minipage}[t]{0.49\linewidth}
        \blue{N沟道 JFET}

        $V_{P0} = \frac{ea^2 N_d}{2\epsilon_s}$
        $V_P = V_{bi} - V_{P0}$

        $h = \sqrt{\frac{2\epsilon_s(V_{bi} - V_{GS})}{eN_d}}$

        $V_{sat} = V_{P0} - (V_{bi} - V_{GS})$

    \end{minipage}
    \hfill
    \begin{minipage}[t]{0.49\linewidth}
        \blue{P沟道 JFET}

        $V_{P0} = \frac{ea^2 N_a}{2\epsilon_s}$

        $V_P = V_{P0} - V_{bi}$

        $h = \sqrt{\frac{2\epsilon_s(V_{bi} + V_{GS})}{eN_a}}$

        $V_{sat} = V_{P0} - (V_{bi} + V_{SG})$

    \end{minipage}

    \red{夹断电流}(栅极零偏且内建电势忽略时的理论最大漏极电流):
    $I_{P1} = \frac{\mu_n (eN_d)^2 W a^3}{6 \epsilon_s L}$

    \red{漏源电流}:
    $
    I_{D1} = I_{P1} \left[ 3 \frac{V_{DS}}{V_{p0}} - 2 \left(\frac{V_{DS} + V_{bi} - V_{GS}}{V_{p0}}\right)^{3/2} \right.
    $

    $\left. + 2 \left(\frac{V_{bi} - V_{GS}}{V_{p0}}\right)^{3/2} \right]$

    \red{沟道电导}:
    $
    g_d = \frac{3I_{P1}}{V_{p0}} \left[ 1 - \left( \frac{V_{bi} - V_{GS}}{V_{p0}} \right)^{1/2} \right]
    $

    \red{最大电导}:$G_{01} = \frac{3I_{P1}}{V_{p0}}$

    \red{饱和电流}:
    
    $I_{D1(sat)} = I_{P1} \left[ 1 - 3 \frac{(V_{bi} - V_{GS})}{V_{p0}} \right.$
    
    $\left. \times \left( 1 - \frac{2}{3} \sqrt{\frac{V_{bi} - V_{GS}}{V_{p0}}} \right) \right]$

\end{minipage}
\hfill
\begin{minipage}[t]{0.39\linewidth}

\hlgreen{\textbf{MESFET}}

\end{minipage}

\subsection{非理想因素}

\subsubsection{沟道长度调制效应}

\entry{1.\ 现象定义(核心概念)}{
    \concepttable{
        \conrow{定义}{在沟道夹断(pinch-off)后,若继续增大漏极电压 $V_{DS}$,电流不会像理想模型那样完全饱和不变。}
        \conrow{物理过程}{当 $V_{DS} > V_{DS(\mathrm{sat})}$ 时,栅漏 PN 结反向偏置增大 $\Rightarrow$ 漏端耗尽区沿沟道方向扩展。}
        \conrow{结果}{电中性导电沟道的\red{有效长度} $L'$ 变短。该“$L'$ 随 $V_{DS}$ 变化”的现象称为沟道长度调制效应。}
    }
}

\imgleft[0.2]{images/image-2026-01-04-16-27-21.png}{

    \entry{图解分析}{
        \concepttable{
            \conrow{$V_{DS}>V_{DS(\mathrm{sat})}$}{过夹断:夹断点向源端移动,漏端与夹断点之间形成耗尽区,长度记为 $\Delta L$(原理对所有导通情况适用)。}
            \conrow{有效长度修正}{\red{$L' \approx L - \frac{\Delta L}{2}$}}
        }
    }

}

\hlblue{\textbf{电流方程修正}}

\entry{$L'$ 变短 $\Rightarrow I_D$ 增加}{
沟道纵向电阻满足 $R \propto \frac{\text{Length}}{\text{Area}}$。当有效沟道长度 $L'$ 变短时,沟道电阻减小,因此在相同电场驱动下,漏极电流 $I_D$ 相比理想情况(长度为 $L$)会略有增加。
}

\entry{修正因子}{
    \concepttable{
        \conrow{理想关系}{理想夹断电流 $I_{P1}$(即 $I_{DSS}$/饱和电流)分母含 $L$,因此电流与沟道长度成反比:$I \propto \frac{1}{L}$。}
        \conrow{电流修正}{$I'_{D1} = I_{D1}\cdot \frac{L}{L-\frac{\Delta L}{2}}$,验证“长度越短,电流越大”。}
        \conrow{耗尽区延伸量 $\Delta L$}{$\Delta L \approx \left[\frac{2\varepsilon_s\left(V_{DS}-V_{DS(\mathrm{sat})}\right)}{eN_d}\right]^{1/2}$,来源于 PN 结耗尽宽度公式。$\Delta L \propto \sqrt{V_{DS}-V_{DS(\mathrm{sat})}}$,即 $V_{DS}$ 超过饱和电压越多,延伸越长。}
    }
}

\entry{\red{最终修正方程}}{
    $I'_{D1(\mathrm{sat})}=I_{D1(\mathrm{sat})}(1+\lambda V_{DS})$。\par
    其中 $I_{D1(\mathrm{sat})}$ 为理想饱和电流;$\lambda$ 为沟道长度调制系数
}

\entry{小信号输出阻抗 $r_{ds}$}{
    $r_{ds}=\frac{\partial V_{DS}}{\partial I'_{D1}}\approx \frac{\Delta V_{DS}}{\Delta I'_{D1}}$。\par
    理想情况下饱和区 $\Delta I_D=0$,故 $r_{ds}\to\infty$;考虑沟道长度调制后 $\Delta I_D\neq 0$,$r_{ds}$ 为有限值。
}

\subsection{等效电路和频率限制}


\newpage

\section{附录 A\ 习题整理-01}

这里仅整理作业题以及期中考试习题,不包含章节后习题。

\subsection{半导体材料物理}

\entry{期中-01}{请简述费米能级的物理意义,并说出影响费米能级位置的因素以及在其影响下费米能级如何变化。}

\entry{答}{
    费米能级指半导体中被\red{电子占据概率为0.5的假定能级(5 分)},标志了电子填充能级的水平,能量低于$E_{F}$的能级被电子占据的概率大于0.5,能量高于$E_{F}$的能级被电子占据的概率小于0.5。费米能级的位置受温度和半导体掺杂浓度影响。对于p型掺杂,随着掺杂浓度增加费米能级向价带方向移动;对于n型掺杂,掺杂浓度增加费米能级向导带方向移动;\red{掺杂半导体随着温度升高本征激发逐渐主导时,费米能级向本征费米能级移动(5 分)}。
}

\subsection{PN 结}

\subsubsection{PN 结的形成过程}

这部分没有习题,所以对应的我前面整理的也比较少。

\subsubsection{平衡 PN 结}

\entry{课堂练习-C2-01}{硅pn结所处环境温度为300K,掺杂浓度为$N_a = 10^{16} \text{cm}^{-3}$,$N_d = 10^{15} \text{cm}^{-3}$,计算pn结中的空间电荷区宽度$W$和零偏时结内的最大电场$E_{\max}$。}

\entry{启示}{
    就是单纯地练公式,注意单位换算就行。
}

\entry{答}{
\begin{align*}
V_{bi} &= \frac{kT}{e} \ln \left( \frac{N_A N_D}{n_i^2} \right) = V_t \ln \left( \frac{N_A N_D}{n_i^2} \right) = 0.635\,\text{V} \\
W &= \left\{ \frac{2\epsilon_s V_{bi}}{e} \left[ \frac{N_a + N_d}{N_a N_d} \right] \right\}^{1/2} \\
&= \left\{ \frac{2(11.7)(8.85 \times 10^{-14})(0.635)}{1.6 \times 10^{-19}} \left[ \frac{10^{16} + 10^{15}}{(10^{16})(10^{15})} \right] \right\}^{1/2} \\
&= 0.951 \times 10^{-4}\,\text{cm} = 0.951\,\mu\text{m} \\
x_n &= \left( \frac{2\epsilon_s V_{bi}}{e} \cdot \frac{N_A}{N_D} \cdot \frac{1}{N_A + N_D} \right)^{1/2} = 0.864 \times 10^{-4}\,\text{cm} \\
E_{\max} &= \frac{-e N_d x_n}{\epsilon_s} = \frac{-(1.6 \times 10^{-19})(10^{15})(0.864 \times 10^{-4})}{(11.7)(8.85 \times 10^{-14})} = -1.34 \times 10^{4}\,\text{V/cm}
\end{align*}
}

\subsection{问答题整理}

\blue{孟庆巨版本教材:} \red{红色题干}为作业、课件出现过的题目。
\concepttable{
    \conrow{界面态对肖特基势垒高度的影响}{在大多数实用的肖特基势垒中,\red{界面态在决定 $\phi_b$ 数值中处于支配地位},势垒高度基本上与两个功函数差以及半导体中的掺杂度无关。由于表面态密度无法预知,势垒高度通常为经验值。}
    \conrow{加偏压时肖特基势垒能带图中 $q\phi_b$ 几乎不变的原因}{由于金属中电子浓度极高,空间电荷区极薄,电势连续性决定了加偏压时肖特基势垒能带图中 $q\phi_b$ 几乎不变。}
    \conrow{\red{肖特基势垒二极管与 PN 结二极管的区别}}{\textbf{肖特基势垒二极管}是\red{多子器件},\textbf{PN 结二极管}是少子器件。主要区别:\par
    (1) 无少数载流子存储,存储时间可忽略,适合高频和快速开关;\par
    (2) 多数载流子电流远高于少数载流子,饱和电流远高于同面积 PN 结二极管;\par
    (3) 对同样电流,肖特基势垒上的正向电压降远低于 PN 结,适合箝位和限幅应用;\par
    (4) 多子数目起伏小,噪声小;\par
    (5) 温度特性好。}
    \conrow{金属与重掺杂半导体接触为何可形成欧姆接触}{若半导体为重掺杂(如 $10^{19}\,\mathrm{cm}^{-3}$ 或更高),空间电荷层宽度极薄,载流子可\red{隧道穿透}而非越过势垒。两侧电子均可隧穿,正反向偏压下 $I$-$V$ 曲线基本对称,表现为非整流、低电阻的欧姆接触。}
}

\blue{第四次作业相关:}

\concepttable{
    \conrow{在理想情况下,金属和半导体之间形成非整流接触势垒的条件是什么?}{前面有}
    \conrow{画出n型欧姆接触时,零偏、正偏、反偏条件下的能带图}{这三个图前面都有}
    \conrow{根据给出的金属与半导体,画出形成金半接触后的能带图}{原则就是让金属的费米能级不变,然后让半导体的费米能级和金属对齐,画出弯曲的能带图即可。然后根据半导体类型以及载流子的流向标注是阻挡层还是反阻挡层。}
}

\newpage

\section{附录 B\ 习题整理-02}

从此开始完全整理 Neaman 和 孟庆巨老师的教材/考研指导后面的计算类习题。(前四章就主要是我在期中之前整理的内容)

\subsection{半导体物理基础}

\hlgreen{\textbf{能带的产生}} 和 \hlgreen{\textbf{载流子的统计分布}} 上次也没考计算题相关的,这个记住概念和影响因素就行。

\subsubsection{半导体载流子输运}


\entry{5-1}{
    硅中施主杂质原子的浓度为 $N_{d}=10^{15}\,\mathrm{cm}^{-3}$。设电子迁移率为 $\mu_{n}=1300\,\mathrm{cm}^2/\mathrm{V}\cdot\mathrm{s}$,空穴迁移率为 $\mu_{p}=450\,\mathrm{cm}^2/\mathrm{V}\cdot\mathrm{s}$。
    (a) 求材料的电阻率;
    (b) 求材料的电导率。
}

\entry{启发意义}{多数载流子决定导电性,少数载流子可忽略;掌握 $\rho$ 与 $\sigma$ 的互逆关系及电导率公式。}

\entry{解答}{
\[
\rho=\frac{1}{e\mu_n N_d}=\frac{1}{(1.6\times10^{-19})(1300)(10^{15})}=4.808\ \Omega\text{-cm}
\]
\[
\sigma=\frac{1}{\rho}=0.208\ (\Omega\text{-cm})^{-1}
\]
}


\entry{5-6}{
$T=300K$ 时,均匀掺杂的GaAs半导体的参数为 $N_{d}=10^{16}\,\mathrm{cm}^{-3}$,$N_{a}=0$。
(a) 计算热平衡时的电子和空穴浓度;
(b) 外加电场为 $E=10\,\mathrm{V/cm}$,计算漂移电流密度;
(c) 当 $N_{d}=0$,$N_{a}=10^{16}\,\mathrm{cm}^{-3}$ 时,重做(a)和(b)的计算。
}
\entry{启发意义}{
考查本征载流子浓度、漂移电流密度公式及对 N 型/P 型的迁移率选用。
}

(a) $N_d=10^{16}\,\mathrm{cm}^{-3}$,$N_a=0$,$n_i=1.8\times10^6\,\mathrm{cm}^{-3}$

\[
n_0 = N_d = 10^{16}\,\mathrm{cm}^{-3}
\]
\[
p_0 = \frac{n_i^2}{n_0} = \frac{(1.8\times10^6)^2}{10^{16}} = 3.24\times10^{-4}\,\mathrm{cm}^{-3}
\]

(b) 电子迁移率 $\mu_n \approx 7500\,\mathrm{cm}^2/(\mathrm{V}\cdot\mathrm{s})$,$E=10\,\mathrm{V/cm}$

\[
J = e\mu_n n_0 E = (1.6\times10^{-19}) \times 7500 \times 10^{16} \times 10 = 120\,\mathrm{A/cm}^2
\]

(c) $N_d=0$,$N_a=10^{16}\,\mathrm{cm}^{-3}$

\[
p_0 = N_a = 10^{16}\,\mathrm{cm}^{-3}
\]
\[
n_0 = \frac{n_i^2}{p_0} = 3.24\times10^{-4}\,\mathrm{cm}^{-3}
\]
空穴迁移率 $\mu_p \approx 310\,\mathrm{cm}^2/(\mathrm{V}\cdot\mathrm{s})$

\[
J = e\mu_p p_0 E = (1.6\times10^{-19}) \times 310 \times 10^{16} \times 10 = 4.96\,\mathrm{A/cm}^2
\]



% --- 手动换栏命令(如果需要强制换列)---
% \columnbreak 

\end{multicols*}

\end{document}