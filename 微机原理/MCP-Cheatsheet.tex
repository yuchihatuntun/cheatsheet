\documentclass[10pt, a4paper, landscape]{article}

% -------------------------------------------------
% 宏包引入
% -------------------------------------------------
\usepackage[fontset=mac]{ctex}       % 中文支持
\usepackage{multicol}   % 多分栏
\usepackage{calc}
\usepackage{ifthen}
\usepackage[landscape]{geometry} % 页面设置
\usepackage{amsmath,amsthm,amsfonts,amssymb} % 数学公式
\usepackage{color,graphicx,overpic} % 颜色与图片
\usepackage{hyperref}   % 超链接
\usepackage{enumitem}   % 列表环境控制
\usepackage{titlesec}   % 标题控制
\usepackage{bm}         % 加粗数学符号
\usepackage{xcolor}
\usepackage{tikz}       % 绘图
\usetikzlibrary{decorations.pathreplacing, positioning} % 加载brace装饰库
\setCJKmainfont{PingFang SC}
\setCJKsansfont{PingFang SC}
\setCJKmonofont{PingFang SC}

% -------------------------------------------------
% 自定义颜色
% -------------------------------------------------

\definecolor{myblue}{HTML}{003153} % 蓝色字
\definecolor{myred}{HTML}{85120F}  % 红色字
\definecolor{hlblue}{HTML}{ADC1DB} % 蓝色高亮
\definecolor{hlred}{HTML}{F3755E}  % 红色高亮
\definecolor{hlyellow}{HTML}{E3C79F}  % 奢金高亮
\definecolor{hlgreen}{HTML}{BED49D}  % 抹茶绿高亮

% -------------------------------------------------
% 极限空间压缩设置 (核心部分)
% -------------------------------------------------

% 1. 页边距设置为极小 (0.5cm)
\geometry{top=0.5cm,left=0.5cm,right=0.5cm,bottom=0.5cm}

% 2. 去掉段落首行缩进,改为段落间略微留空(可选,这里为了紧凑设为0)
\setlength{\parindent}{0pt}
\setlength{\parskip}{0pt}

% 3. 设置正文基础字体大小为 scriptsize (约8pt),如果还觉得大,可以改为 \tiny
\renewcommand{\baselinestretch}{0.9} % 压缩行间距
\let\oldfootnotesize\footnotesize
\renewcommand{\footnotesize}{\fontsize{7pt}{8pt}\selectfont}

% 4. 压缩列表环境 (Itemize/Enumerate) 的间距
\setlist{nolistsep} 
\setlist[itemize]{leftmargin=*}
\setlist[enumerate]{leftmargin=*}

% 5. 压缩标题间距
\titleformat{\section}{\bfseries\scriptsize\color{myblue}}{}{0em}{}[\hrule] % 标题带下划线,蓝色,省空间
\titlespacing*{\section}{0pt}{2pt}{1pt} % 上方留2pt,下方留1pt
\titleformat{\subsection}
    [runin] % 不换行
    {\bfseries\scriptsize} % 粗体、scriptsize,黑色字体
    {} % 不显示编号
    {0pt} % 标题与正文间距
    {\hlyellow} % 用hlyellow高亮命令包裹标题
    [] % 标题内容后无内容
\titleformat{\subsubsection}
    [runin] % 不换行
    {\bfseries\tiny} % 粗体、scriptsize,黑色字体
    {} % 不显示编号
    {0pt} % 标题与正文间距
    {\hlgreen} % 用hlgreen高亮命令包裹标题
    [] % 标题内容后无内容
\titlespacing*{\subsection}{0pt}{1pt}{0.5em} % 上方1pt,下方0.5em(水平间距)
\titlespacing*{\subsubsection}{0pt}{1pt}{0.5em} % 上方1pt,下方0.5em(水平间距)

% -------------------------------------------------
% 自定义命令
% -------------------------------------------------
% 颜色字
\newcommand{\red}[1]{\textbf{\textcolor{myred}{#1}}}  
\newcommand{\blue}[1]{\textbf{\textcolor{myblue}{#1}}} 
\newcommand{\entry}[2]{$\bullet$ \textbf{#1}: #2\par\vspace{0.5pt}}
% 高亮
\newcommand{\cbox}[2][yellow]{\begingroup\setlength{\fboxsep}{1pt}\colorbox{#1}{\strut#2}\endgroup}
\newcommand{\hlblue}[1]{\cbox[hlblue]{#1}}
\newcommand{\hlred}[1]{\cbox[hlred]{#1}}
\newcommand{\hlyellow}[1]{\cbox[hlyellow]{#1}}
\newcommand{\hlgreen}[1]{\cbox[hlgreen]{#1}} 
% 图片插入
\newcommand{\img}[2][0.9\linewidth]{%
    {\par\vspace{1pt}\centering\includegraphics[width=#1]{#2}\par\vspace{1pt}}%
}
% 左图右文 (参数: [图片宽度比例]{图片路径}{右侧文字内容})
\newcommand{\imgleft}[3][0.3]{%
    \noindent\begin{minipage}[t]{#1\linewidth}%
        \vspace{0pt}%
        \includegraphics[width=\linewidth]{#2}%
    \end{minipage}%
    \hfill%
    \begin{minipage}[t]{0.98\linewidth - #1\linewidth}%
        \vspace{0pt}%
        #3%
    \end{minipage}\par\vspace{2pt}%
}
% 左文右图 (参数: [图片宽度比例]{图片路径}{左侧文字内容})
\newcommand{\imgright}[3][0.3]{%
    \noindent\begin{minipage}[t]{0.98\linewidth - #1\linewidth}%
        \vspace{0pt}%
        #3%
    \end{minipage}%
    \hfill%
    \begin{minipage}[t]{#1\linewidth}%
        \vspace{0pt}%
        \includegraphics[width=\linewidth]{#2}%
    \end{minipage}\par\vspace{2pt}%
}
% -------------------------------------------------
% 新增:概念速查表专用命令
% -------------------------------------------------
% 表格容器
\newcommand{\concepttable}[1]{%
    {\setlength{\tabcolsep}{1.5pt}% 局部减小列间距
     \renewcommand{\arraystretch}{0.92}% 局部紧缩行距
     \par\vspace{2pt}{\color{myblue}\hrule height 0.6pt}\vspace{1pt}% 上边框(蓝色,0.6pt粗)
     \noindent\begin{tabular}{@{}p{0.22\linewidth}p{0.76\linewidth}@{}}%
     #1%
     \end{tabular}%
     \vspace{1pt}{\color{myblue}\hrule height 0.6pt}\par\vspace{2pt}}% 下边框(蓝色,0.6pt粗)
}
% 表格行 (参数: {概念名}{解释})
\newcommand{\conrow}[2]{\blue{#1} & #2 \\}




% -------------------------------------------------
% 正文
% -------------------------------------------------
\begin{document}

\tiny

% 三栏布局
\begin{multicols*}{3}

\section{第一章\ 微机概述}
\entry{地址引脚数}{为 \red{16 条},可以寻址 \red{$2^{16}=64$KB} 的存储单元。\red{$2^{10}=1$K},\red{$2^{20}=1$M},\red{$2^{30}=1$G};\red{b} 表示位,\red{B} 表示字节。}
\entry{补码}{补码为原码取反加 1,最高位 \red{1} 表示负数,\red{0} 表示正数。例:\red{[-89]补=10100111}。十六进制数后面要加 \red{H},否则默认十进制。}

\concepttable{
    \conrow{微处理器}{微型计算机中用 \red{CPU} 表示,由一片或几片大规模集成电路组成,具有\red{运算和控制器功能}的中央处理器部件。}
    \conrow{微型计算机}{以\red{微处理器}为核心,配上\red{存储器}、\red{输入输出接口电路}及\red{系统总线}组成,又称\red{主机}。}
    \conrow{微型计算机系统}{以\red{微型计算机}为中心,配以外围设备、电源和辅助电路(硬件系统),以及指挥工作的\red{软件系统}。}
    \conrow{软件系统}{包括:\red{系统软件}(操作系统、编译器、数据库等)和\red{应用软件}(Office、微信等)。}
}

\entry{典型微机硬件系统结构}{
    计算机的经典结构为\red{冯·诺依曼结构}(\red{存储程序式计算机}结构),特点包括:由\red{运算器}、\red{控制器}、\red{存储器}、\red{输入设备}和\red{输出设备}五个基本部分组成;\red{程序和数据以二进制代码形式存放在存储器中},位置由\red{地址}指定,地址码也是二进制;\red{控制器}根据存储器中的\red{指令序列(程序)}工作,并由\red{程序计数器(PC)}控制指令执行,具有判断能力,可根据计算结果选择不同动作流程。\red{微处理器}是执行指令的核心部件,包含\red{运算器}和\red{控制器}。\red{存储器}用于存储当前正在使用的程序和数据。\red{I/O 设备和接口}实现微处理器与外部设备的连接,如\red{显示器接口}、\red{硬盘接口}等。\red{系统总线}连接微处理器和其他部件,分为\red{地址总线}、\red{数据总线}和\red{控制总线},分别传输地址、数据和控制信息。
}
\imgleft[0.3]{images/image-2025-12-30-15-59-39.png}{
    \entry{总线}{
        \red{CPU}通过总线读取指令,与\red{内存}、\red{外设}之间进行数据交换。在\red{CPU}、\red{内存}和\red{外设}确定的情况下,\red{总线速度}是制约计算机整体性能的关键。\red{总线}是一组传输公共信息的信号线集合,是各部件之间传输\red{地址}、\red{数据}和\red{控制信息}的公共通道。处理器内部、处理器与高速缓冲器及内存之间、处理器与外围设备之间,均通过总线连接。根据结构组织方式不同,总线分为:\red{单总线结构}、\red{双总线结构}、\red{双重总线结构}(\red{80X86系统采用},\red{局部总线}用于CPU访问局部存储器和局部I/O,\red{全局总线}用于CPU访问全局存储器和全局I/O,由\red{总线控制逻辑}统一安排,实现\red{并行操作},提高系统数据处理和传输效率)。
    }
}

\subsection{数制与计算}

\entry{有符号数的机器表示}{
    \blue{真值}指带正负号的实际数值(如 $+5, -5$),而\blue{机器数}是计算机内部存储的编码形式。有符号数常用三种标准编码:
    \begin{itemize}
        \item \blue{原码}:$[X]_{\text{原}}$
        \item \blue{反码}:$[X]_{\text{反}}$
        \item \blue{补码}:$[X]_{\text{补}}$
    \end{itemize}
}

\entry{正负数的编码差异}{
    \blue{正数}:三种表示法完全一致,即 $[X]_{\text{原}} = [X]_{\text{反}} = [X]_{\text{补}}$,符号位为 $0$,数值位直接表示绝对值。
    \blue{负数}:三种表示法存在差异,所有区别仅体现在负数的编码规则上。
}

\entry{补码的主导地位}{
    \red{现代计算机均采用补码}。补码便于硬件实现加减法,成为实际工程标准。
}

\entry{补码加减法运算规则}{
    \blue{核心思想}:\red{减法变加法},即 $X-Y$ 可转化为 $X+(-Y)$,简化硬件设计。
    \begin{itemize}
        \item \blue{加法}:$[X+Y]_{\text{补}} = [X]_{\text{补}} + [Y]_{\text{补}}$
        \item \blue{减法}:$[X-Y]_{\text{补}} = [X]_{\text{补}} - [Y]_{\text{补}} = [X]_{\text{补}} + [-Y]_{\text{补}}$
    \end{itemize}
    \blue{符号位参与运算}:补码加法时,符号位与数值位同等对待,全部参与二进制加法(逢二进一)。
}

\entry{补码求负(变补规则)}{
    $[-X]_{\text{补}}$ 的求法:将 $[X]_{\text{补}}$ 的\red{所有位(包括符号位)按位取反,再加 $1$}。即“全字取反加一”。
    \blue{注意}:\red{包括符号位},不区分符号和数值位。
}

\section{第二章\ 微机结构}

\imgleft[0.4]{images/image-2025-12-28-22-49-19.png}{

    \entry{8086/8088 对比}{
        \red{8086}:16 位外部数据总线;\red{8088}:8 位外部数据总线(准 16 位),为兼容 8 位外设。
        两者均为 \red{20 位} 地址线(寻址 \red{1MB}),I/O 寻址 \red{16 位}(\red{64KB})。
    }

    \concepttable{
        \conrow{BIU (总线接口)}{负责\red{取指}、\red{总线操作}、\red{物理地址形成}。含指令队列、段寄存器(CS/DS/ES/SS)、IP、地址加法器。}
        \conrow{EU (执行单元)}{负责\red{译码}、\red{执行}、\red{运算}。含 ALU、通用寄存器、标志寄存器、控制电路。}
        \conrow{并行工作}{BIU 取指与 EU 执行重叠,实现流水线雏形,提高总线利用率。}
    }

}

\entry{物理地址计算}{
    $PA = \text{段基址} \times 10H + \text{偏移地址}$。
    段基址由\red{段寄存器}(16位)左移 4 位提供,偏移地址由 \red{IP} 或 \red{EU} 计算。
}

\concepttable{
    \conrow{指令队列}{8086 为 \red{6 字节},8088 为 \red{4 字节}。FIFO 原则。当队列空出 2 字节(8086)或 1 字节(8088)时,BIU 自动取指。执行跳转/调用/返回时\red{清空}。}
    \conrow{通用寄存器}{AX (累加器), BX (基址), CX (计数), DX (数据)。可拆分为 H/L 8 位使用。}
    \conrow{指针与变址}{SP (堆栈指针), BP (基址指针), SI (源变址), DI (目的变址)。}
}


% --- 第2部分:列表压缩测试 ---
\section{第三章\ 80x86 指令系统}

\subsection{寻址方式}

需要看清楚问的是源操作数还是目标操作数的寻址方式,如果是字操作数,要写两个单元的地址。

\concepttable{
    \conrow{固定寻址}{如 \red{AAA}}
    \conrow{立即数寻址}{操作数直接包含在指令中。举例 \red{MOV AL,15H || MOV AX,1234H}}
    \conrow{寄存器寻址}{操作数在寄存器中。例:\red{MOV AX, BX}}
    \conrow{存储器寻址}{比较复杂,见下文详细说明。}
}

\entry{存储器寻址}{
    当执行单元 \red{EU} 需要读/写位于存储器的操作数时,应根据指令给出的寻址方式,由 \red{EU} 先计算出操作数的有效地址 \red{$EA$}(偏移地址),并将它送给 \red{BIU};同时请求 \red{BIU} 执行一个总线周期,\red{BIU} 将某个段寄存器的内容左移 4 位,加上 \red{EU} 送来的有效地址 \red{$EA$} 形成 20 位的物理地址,然后执行总线周期,读/写指令所需的操作数。有效地址 \red{$EA$}(偏移地址) 的值要根据指令所采用的寻址方式计算得出。计算 \red{$EA$} 的通式为:
}
\vspace{-8pt}
{\setlength{\abovedisplayskip}{0pt}
\setlength{\belowdisplayskip}{0pt}
\[
\boxed{
EA = \text{基址值} \left\{
\begin{array} { l }
{ BX } \\
{ BP }
\end{array}
\right. +
\text{变址值} \left\{
\begin{array} { l }
{ SI } \\
{ DI }
\end{array}
\right. +
\text{位移量} \left\{
\begin{array} { l }
{ 0 } \\
{ 8 } \\
{ 16 }
\end{array}
\right\}
}
\]}
\vspace{-8pt}

\concepttable{
    \conrow{直接寻址}{操作数的 $EA$ 由指令直接给出。段地址默认数据段 \red{DS},其它数据段应在指令中用段前缀指出。这种寻址方式的指令执行速度较快,主要用于存取位于存储器中的简单变量。例:\red{MOV AX, [1234H]}}
    \conrow{间接寻址}{操作数在存储器中,\red{存储单元的有效地址由寄存器指出}。\par
    \red{BX、SI、DI} — 默认数据段 \red{DS}\par
    \red{BP} — 默认堆栈段 \red{SS},\par
    \blue{注意}:间接寻址的地址寄存器只能是 \red{BX、BP、SI、DI},不能是其它寄存器。指令中地址寄存器要加方括号, 如: \red{[BX]}。根据所采用的地址寄存器的不同,间接寻址方式又可分为以下 3 种:}
}

\concepttable{
    \conrow{基址寻址}{操作数的有效地址由基址寄存器 (\red{BX} 或 \red{BP}) 的内容和指令中给出的地址位移量 ($0$ 位或 $8$ 位或 $16$ 位) 之和来确定。$EA = BX/BP + 0 \text{位}/8 \text{位}/16 \text{位移量}$。例:\red{MOV AX, [BX]},假设 $BX = 1122H, DS = 3000H$,则 $PA = 30000H + 1122H = 31122H, 30000H + 1123H = 31123H$。假设 $[31122H] = 34H, [31123H] = 56H$,则指令执行后,\red{AX = 5634H}。例:假设 $BETA=8, DS=6000H, BX=5000H$,则 \red{MOV AL, [BX+8]} || \red{MOV AL, 8[BX]};“8”是偏移地址位移量,不是乘的倍数 || \red{MOV AL, [BX+BETA]} || \red{MOV AL, BETA [BX]}。注:上面 4 条指令是等价的。$EA = 5000H+8= 5008H, PA = DS \times 16+5008H= 60000H+5008H= 65008H$。假设 $[65008H]=68H$,执行后,\red{AL=68H}}
    \conrow{变址寻址}{操作数的有效地址由变址寄存器 (\red{SI} 或 \red{DI}) 的内容与指令中给出的地址位移量 ($0$ 位、$8$ 位或 $16$ 位) 之和来确定。$EA$(偏移地址)$= SI/DI + 0 \text{位}/8 \text{位}/16 \text{位移量}$。例:\red{MOV BETA[DI], AX} || \red{MOV BX, [SI+BETA]}}
    \conrow{基址加变址}{操作数的有效地址 $EA$ 为以下三部分之和:基址寄存器 (\red{BX} 或 \red{BP}) 的值、变址寄存器 (\red{SI} 或 \red{DI}) 的值、指令中的地址位移量 ($0$ 位、$8$ 位或 $16$ 位)。$EA = [BX/BP] + [SI/DI] + 0/8/16 \text{位偏移量}$。例: \red{MOV BX, [BX+SI]} || \red{MOV [BX][DI], AX} || \red{MOV AX, 8[BX][SI]} || \red{MOV AX, 8[BX+SI]} || \red{MOV AX, [BX+SI+8]} || \red{MOV AX, ES:[BX+SI+10H]} “ES:”---段前缀,指定操作数在附加段,源操作数 $PA = ES \times 16+ BX+SI+10H, ES \times 16+ BX+SI+10H+1$ || \red{MOV AX, [BP+SI+20H]} \red{BP}---操作数在堆栈段,源操作数 $PA = SS \times 16+ BP+SI+20H, SS \times 16+ BP+SI+20H+1$}
}


\subsection{指令系统}

\subsubsection{处理器控制类指令}
\entry{\hlblue{标志操作指令}}{
    \img[0.9\linewidth]{images/image-2026-01-01-04-02-42.png}
    \vspace{-2pt}

    \concepttable{
        \conrow{CLC}{清除进位标志,\red{CF=0}。}
        \conrow{STC}{设置进位标志,\red{CF=1}。}
        \conrow{CMC}{进位标志取反,\red{CF=$\overline{CF}$}。}
        \conrow{CLD}{清除方向标志,\red{DF=0}(地址递增)。}
        \conrow{STD}{设置方向标志,\red{DF=1}(地址递减)。}
        \conrow{CLI}{清除中断允许标志,\red{IF=0}(禁止中断)。}
        \conrow{STI}{设置中断允许标志,\red{IF=1}(允许中断)。}
    }
}
% --- 第3部分:图表占位 ---
\section{第四章 汇编语言程序设计基础}
\red{注意}:如果有图片,尽量不要插入大图。建议手绘或者用 TikZ,或者只是文字描述流程。

% --- 模拟填充内容以展示三栏效果 ---
\section{第五章 微机总线技术}

\subsection{概念}

\entry{总线定义}{
    总线是用来\red{连接各部件的一组通信线}。换言之,总线是一种在多于两个模块(设备或子系统)间传送信息的公共通路。为在各模块之间实现信息共享和交换,总线由传送信息的物理介质以及一套管理信息传输的协议构成。
}
\entry{三态逻辑}{除了逻辑 0 和逻辑 1 之外,还有第三种状态——高阻态(Hi-Z)。高阻态相当于断开电路,使得该信号线不对总线产生影响,从而允许其他设备驱动总线。}

\entry{微机总线分类}{
按功能可分为\red{系统总线(内总线)}和\red{通信总线(外总线)};按信号传输方式可分为\red{串行总线}与\red{并行总线}。
}

\concepttable{
    \conrow{片内总线}{又称元件级总线,位于\red{集成电路芯片内部},连接芯片内各功能单元}
    \conrow{片总线}{连接同一块插件板(PCB)上各个芯片的总线,常见如\red{IIC 总线}、\red{SPI 总线}等。}
    \conrow{\red{内总线}}{又称系统总线、板级总线或微机总线,用于微机系统中各插件之间的信息传输。是我们主要研究的总线类型}
    \conrow{外总线}{又称通信总线,常见如\red{USB}、\red{PCIe}、\red{SATA}等。}
}

\concepttable{
    \conrow{数据总线}{\blue{双向数据线},是一个总线周期内可以传送的数据位数,}
    \conrow{地址总线}{\blue{单向地址线}(通常为三态逻辑控制),位宽如\red{16、20 等}, 位宽代表了芯片的寻址能力(20位地址线意味着$2^{20}$的寻址能力)。}
    \conrow{控制总线}{传送定时与控制信号以协调各部件动作(如读/写、中断请求、复位等),控制总线一般\red{不采用三态逻辑}。}
}


\concepttable{
    \conrow{总线宽度}{即数据总线的位数,决定了\red{一个总线周期内可并行传输的数据位数}。}
    \conrow{总线频率}{总线的工作时钟频率(单位为 Hz),表示\red{每秒总线可执行的数据传输次数}。}
    \conrow{传输速率}{衡量总线性能的最终指标,表示\red{单位时间内总线可传输的数据总量}(单位为 MB/s)。三者关系:$\text{传输速率} = (\text{总线宽度}/8) \times \text{总线频率}$。}
}
\subsection{常用芯片}
\entry{常用缓冲器芯片}{\red{74LS245}(双向总线收发器)、\red{74LS244}(单向三态缓冲器)、\red{74LS373}(八位透明锁存器)、\red{74LS138}(3-8 线译码器)、\red{74LS157}(4-1 线数据选择器)等。}

\entry{74LS244}{\red{单向三态缓冲器},8 路分为两组\red{两个使能端均为低电平有效}。常用于数据总线的缓冲与隔离。
}
\concepttable{
    \conrow{$1A1$--$1A4$}{$1Y1$--$1Y4$(由 $1G$ 控制)}
    \conrow{$2A1$--$2A4$}{$2Y1$--$2Y4$(由 $2G$ 控制)}
}
    

\entry{74LS245}{
    又称\red{收发器}。在\red{A 端}和\red{B 端}之间,\red{并联了两个方向相反的单向三态门}。拥有两个控制信号:
}
\concepttable{
    \conrow{E}{\red{全局使能信号},决定整个双向通道是否工作或处于高阻态,\red{低电平有效}。}
    \conrow{DIR}{\red{方向控制信号},在通道被 $E$ 使能时,选择数据是 $A\rightarrow B$ 还是 $B\rightarrow A$ 传输。}
}

\entry{74LS373}{
    由两个独立逻辑单元组成:
}
\concepttable{
    \conrow{D型锁存器}{\red{存储/保持数据}(由\red{G}控制)。使能时,锁存器处于\red{透明}状态,输出\red{1Q}跟随输入\red{1D}的逻辑电平变化;非使能时,锁存器被关闭,\red{保存上一瞬间1D上的逻辑状态}。}
    \conrow{三态输出门}{负责控制数据是否被驱动到输出总线(由\red{OE}控制,使能时传输,否则为\red{高阻态})。}
}
\subsection{引脚复用}

8086 一共的40个引脚不足以承载20位地址总线和16位数据总线及其他的控制线

\concepttable{
    \conrow{AD0--AD15}{\red{地址/数据复用总线}。在总线周期的 T1,传递地址总线低 $16$ 位($A_0$--$A_{15}$);在 T2--T4,转而用作数据总线($D_0$--$D_{15}$)。}
    \conrow{A16/S3--A19/S6}{\red{地址/状态复用总线}。在总线周期的 T1,传送地址总线高 $4$ 位($A_{16}$--$A_{19}$,构成 $20$ 位完整地址);在 T2--T4,转而输出 CPU 的内部状态信号($S_3$--$S_6$)。}
}
\subsection{工作模式}

即使复用了地址和数据总线,剩余的引脚仍不足以同时支持简单系统。

\concepttable{
    \conrow{最小方式}{
        $\mathrm{MN/MX\#}$ 连接到高电平($\mathrm{V_{CC}}$)时激活。CPU自行产生总线控制信号(直接输出控制信号,如 $\mathrm{INTA\#}$、$\mathrm{ALE}$、$\mathrm{RD\#}$、$\mathrm{WR\#}$、$\mathrm{M/IO\#}$ 等)。适用于\red{单处理器系统},CPU直接控制总线,无需外部总线控制器。
    }
    \conrow{最大方式(Maximum Mode)}{
        $\mathrm{MN/MX\#}$ 连接到低电平($\mathrm{GND}$)时激活。CPU不直接产生控制信号,而是输出编码后的状态信号。引脚24-31被重定义:
    }
}

\concepttable{
    \conrow{$\mathrm{S0\#}$、$\mathrm{S1\#}$、$\mathrm{S2\#}$}{输出状态,需外接 8288 解码生成 $\mathrm{RD\#}$、$\mathrm{WR\#}$、$\mathrm{ALE}$ 等}
    \conrow{$\mathrm{RQ\#/GT0\#}$、$\mathrm{RQ\#/GT1\#}$}{请求/准许信号,用于总线仲裁}
    \conrow{$\mathrm{LOCK\#}$}{总线锁定信号,保证原子操作,适用于多处理器系统}
}

\subsection{总线时序}

\subsubsection{名词解释}
 一个基本的总线周期需要\red{至少4个时钟周期}(即4个T状态,通常标记为T1, T2, T3, T4)。
\concepttable{
    \conrow{时序}{信息在总线上的出现不仅要有\red{空间顺序},还要有\red{严格的顺序和准确的时间}。这种时间和逻辑上的配合关系被称为时序。}
    \conrow{时钟}{由时钟发生器产生的具有固定频率和占空比的脉冲序列。是整个微机系统的时间基准,所有部件的动作都必须与此信号同步。}
    \conrow{主频}{时钟的频率,衡量 CPU 处理速度的一个指标。}
    \conrow{时钟周期}{主频的倒数($T = 1/f$)。}
    \conrow{总线周期}{CPU 通过总线对存储器或 I/O 端口进行\red{一次完整的访问(读或写)所需的时间}。}
}

\subsubsection{常用信号}

除了CLK外的常用控制信号包括:

\concepttable{
    \conrow{M/IO}{\red{存储器/I/O 控制信号},用于区分 CPU 是访问存储器($M/IO=1$)还是访问 I/O 端口($M/IO=0$)。}
    \conrow{DT/R\#}{\red{数据发送/接收信号},在 $T_1$ 周期用于指示 245 等数据缓冲芯片的传输方向。}
    \conrow{RD\#}{\red{读控制信号}。$RD$ 信号为低电平时,表示 8086 CPU 执行读操作。在 DMA 方式时,$RD$ 处于高阻态。}
    \conrow{WR\#}{\red{写控制信号}(输出,三态)。当 8086 CPU 对存储器或 I/O 端口进行写操作时,$WR$ 为低电平。}
    \conrow{ALE}{\red{地址锁存允许信号}(输出)。8086 CPU 在总线周期的第一个时钟周期内发出的正脉冲信号,其下降沿用来把地址/数据总线($AD_{15}\sim AD_{0}$)以及地址/状态总线($A_{19}/S_{6}\sim A_{16}/S_{3}$)中的地址信息锁住存入地址锁存器中。}
    \conrow{DEN\#}{\red{数据缓冲器使能信号}}
    \conrow{BHE\#/S7}{\red{总线高允许/状态 $S_7$ 信号}。分时复用的双重总线,在总线周期开始的 $T_1$ 周期,作为总线高半部分允许信号,低电平有效。在总线周期的其他 $T$ 周期,该引脚输出状态信号 $S_7$。在 DMA 方式下,该引脚为高阻态。}
}

\concepttable{
    \conrow{BHE\#}{总线高允许信号,低电平有效。用于控制数据总线高 8 位(D15--D8)的读写。}
    \conrow{A0}{地址最低位,访问偶地址时,A0 必然为 0。}
}

奇偶地址访问方式:

\concepttable{
    \conrow{BHE\#=0, A0=0}{访问\red{16 位偶地址}(全字访问),数据线 D15--D0 全部有效。}
    \conrow{BHE\#=1, A0=0}{访问\red{8 位偶地址},仅 D7--D0 有效(低字节)。}
    \conrow{BHE\#=0, A0=1}{访问\red{8 位奇地址},仅 D15--D8 有效(高字节)。}
    \conrow{BHE\#=1, A0=1}{无效访问(不选中任何字节)。}
}

\subsubsection{复位、启动时序}

\entry{RESET 信号}{需保持高电平至少 \red{4 个时钟周期}。有效期间总线处于\red{高阻态}(浮空)。}

\imgleft[0.3]{images/image-2025-12-28-20-26-42.png}{

\concepttable{
    \conrow{CS : IP}{\red{FFFFH : 0000H}(启动地址 \red{FFFF0H})}
    \conrow{标志位}{\red{IF=1}(允许中断),其余清零}
    \conrow{DS/SS/ES}{\red{0000H}}
    \conrow{指令队列}{清空}
    }
}

\entry{复位执行流}{
    CPU 复位后 \red{CS:IP = FFFFH:0000H} (物理地址 \red{FFFF0H})。从该处取出 \red{JMP} 指令,跳转至低地址空间的系统初始化程序(如 BIOS)。
    \red{原因}:从 \red{FFFF0H} 到内存顶端 \red{FFFFFH} 仅剩 \red{16 字节},空间极小,无法容纳完整程序。
}

\subsubsection{最小方式总线读操作时序}

8086 CPU需要与外部存储器或I/O端口交换数据,或者需要填充指令队列时,必须执行一个总线周期(四个或更多时钟周期,考虑到 $T_W$ 的存在)。

\imgleft[0.5]{images/image-2025-12-28-20-30-45.png}{

\concepttable{
    \conrow{T1}{确定 $M/\overline{IO}$ 状态;输出地址信息;$ALE$ 发出正脉冲(下降沿锁存地址);$DT/\overline{R}$ 置低(接收)。}
    \conrow{T2}{$AD_{15}\sim AD_0$ 进入高阻态;$\overline{RD}$ 与 $\overline{DEN}$ 有效(低电平);输出状态信号 $S_3\sim S_7$。}
    \conrow{T3}{外设将数据驱动至总线。在 $T_3$ 前沿采样 $READY$ 信号,若为 0 则插入等待周期 $T_W$。}
    \conrow{T4}{在 $T_4$ 前沿(即 $T_3/T_W$ 结束处)采样数据;撤销 $\overline{RD}$、$\overline{DEN}$,总线周期结束。}
}
\entry{关键状态位}{
    $S_4, S_3$ 组合指示当前段寄存器:\red{00:ES, 01:SS, 10:CS, 11:DS}。$S_5$ 指示中断允许标志 $IF$。
}

}

\subsubsection{最小方式总线写操作时序}

基本和读操作类似,区别在于数据传输方向相反。

\imgleft[0.35]{images/image-2025-12-28-20-35-32.png}{

\concepttable{
    \conrow{T1}{输出地址;$ALE$ 发出正脉冲(锁存地址);$DT/\overline{R}$ 置高(发送)。}
    \conrow{T2}{$\overline{WR}$ 信号变低(有效);CPU \red{直接驱动数据}到 $AD$ 总线(不进入高阻态)。}
    \conrow{T3/Tw}{采样 $READY$ 信号;数据在总线上保持稳定,等待外设响应。}
    \conrow{T4}{$\overline{WR}$ \red{上升沿}触发外设写入数据;撤销 $\overline{WR}$、$\overline{DEN}$,周期结束。}
}
\entry{读写区别}{写操作中 $AD$ 总线在 $T_2$ \red{不进入高阻态},由 CPU 持续驱动;数据在 $\overline{WR}$ 上升沿锁存。}


}

\subsubsection{中断时序}

当 CPU 接收到外部中断请求($INTR$)且中断允许标志 IF=1 时,CPU 会执行一个中断响应操作。这个操作在总线时序上表现为\red{两个连续的总线周期}。

\imgleft[0.5]{images/image-2025-12-28-20-38-15.png}{

    \concepttable{
        \conrow{第一周期}{CPU 发出第一个 $\overline{INTA}$ 负脉冲,用于\red{中断握手},此周期不传输数据。}
        \conrow{第二周期}{CPU 发出第二个 $\overline{INTA}$ 负脉冲,8259A 将\red{中断类型码}送往数据总线 $AD_7 \sim AD_0$。}
        \conrow{后续动作}{CPU 在 $T_4$ 前沿采样类型码,计算向量表地址并跳转至\red{中断服务程序 (ISR)}。}
    }
    \entry{注意}{两个周期之间可能插入 $T_I$(空闲态)以兼容外设速度。}

}

\section{第六章 存储系统}

\subsection{存储器分类与基本指标}

\subsubsection{分类}

\concepttable{
    \conrow{按读写功能}{\blue{读写存储器 (RAM)}:可读可写,数据暂存;
        \blue{只读存储器 (ROM)}:只能读,存固件/引导程序}
    \conrow{按存储介质}{\blue{半导体存储器}:如 DRAM、ROM,速度快,体积小;
        \blue{磁存储器}:如硬盘、磁带,容量大,速度慢}
    \conrow{按存取方式}{\blue{随机存取存储器 (RAM)}:存取时间与位置无关;
        \blue{顺序存取存储器}:如磁带,存取时间与位置有关}
    \conrow{按信息保存性}{\blue{易失性存储器}:断电丢失,如 RAM;
        \blue{非易失性存储器}:断电保存,如 ROM、硬盘、SSD}
}

\textbf{内外存区别:}

\concepttable{
    \conrow{内存}{存放当前运行的程序和数据。特点:\red{快}、容量小、随机存取,\red{CPU 可直接通过系统总线访问}。通常由半导体存储器(RAM、ROM)构成。}
    \conrow{外存}{存放非当前使用的程序和数据。特点:\red{慢}、容量大、顺序/块存取,\red{CPU 不能直接访问},需通过\red{I/O 接口电路}调入内存。如硬盘、U 盘、移动硬盘等。}
}

\textbf{RAM 类型特点:}

\concepttable{
    \conrow{SRAM}{静态 RAM。利用\red{双稳态触发器}存储逻辑 0/1,\red{无需刷新},只要不掉电信息不丢失。集成度低,外围控制电路简单,常用于小容量存储(如 \red{Cache})。}
    \conrow{DRAM}{动态 RAM。利用\red{MOS 管栅极分布电容}存储电荷,\red{需定期刷新}。集成度高,外围控制电路复杂,常用于大容量存储(如 \red{主存})。}
}

\concepttable{
    \conrow{存储容量}{$N \times M$ (字数 $\times$ 字长)}
    \conrow{字数 $N$}{单元总数,决定地址线数量 $k$ ($2^k = N$)}
    \conrow{字长 $M$}{每单元位数,决定数据线位宽}
}

\textbf{常用存储芯片:}

\concepttable{
    \conrow{6264 (S\red{RAM})}{$8K \times 8$ (8KB),13 根地址线 ($2^{13}=8K$),8 根数据线}
    \conrow{2114 (S\red{RAM})}{$1K \times 4$,10 根地址线,4 根数据线 (常两片并联组成 8 位)}
    \conrow{2764 (EP\red{ROM})}{$8K \times 8$ (8KB),13 根地址线,8 根数据线,用于存储固件}
}

\entry{地址译码}{将 CPU 高位地址信号转换为芯片片选信号 \red{CS\#}。常用 \red{74LS138} (3-8 译码器) 实现。}

\concepttable{
    \conrow{Bit (位)}{最小存储单位,对应硬件中的\red{双稳态触发器}状态。}
    \conrow{Byte (字节)}{基本处理单位,\red{1 B = 8 bits}。}
    \conrow{Word (字)}{基本处理单位,\red{1 Word = 2 Bytes}。}
    \conrow{常用容量换算}{1KB=\red{$2^{10}$}B, 1MB=\red{$2^{20}$}B, 1GB=\red{$2^{30}$}B。}
}
\subsubsection{基本性能指标}
\concepttable{
    \conrow{存取时间}{从 CPU 发出地址信号到数据有效(读)或写入完毕(写)的时间。}
    \conrow{存储周期}{进行一次完整读写所需的\red{最小时间间隔}。}
    \conrow{二者关系}{\red{存储周期 > 存取时间}(内部电路需恢复时间)。}
    \conrow{读/写周期}{$t_{cyc}(R)$:连续两次读最小间隔;$t_{cyc}(W)$:连续两次写最小间隔。}
}

计算末尾地址的公式为:$\boxed{\text{末尾地址} = \text{首地址} + \text{存储容量} - 1}$ \textbf{注}:减1是因为地址是从0开始计数的,且首地址本身占用了一个存储单元。

\subsection{常见芯片}

\subsubsection{通用引脚特点}

\concepttable{
    \conrow{地址线 $A_0 \sim A_n$}{接地址总线 $AB$,输入信号,决定芯片容量 $N=2^{n+1}$。}
    \conrow{数据线 $D_0 \sim D_m$}{接数据总线 $DB$,双向(RAM)或输出(ROM),决定字长 $M$。}
    \conrow{片选线 $\overline{CS}/\overline{CE}$}{由\red{高位地址译码}产生,低电平有效,用于选中该芯片。}
    \conrow{读写线}{读允许 $\overline{OE}$ 接 CPU 的 $\overline{RD}$;写允许 $\overline{WE}$ 接 CPU 的 $\overline{WR}$。}
}

\subsubsection{6264}

\concepttable{
    \conrow{6264 组织}{地址线 \red{$A_0 \sim A_{12}$} ($2^{13}=8K$);数据线 \red{$D_0 \sim D_7$}。}
    \conrow{双片选机制}{\red{$\overline{CS1}$} (低有效) 和 \red{$CS2$} (高有效)。选中条件:\red{$\overline{CS1}=0$} 且 \red{$CS2=1$}。}
    \conrow{读写控制}{\red{$\overline{OE}$} (输出允许,接系统 $\overline{RD}$);\red{$\overline{WE}$} (写允许,接系统 $\overline{WR}$)。}
}

\subsubsection{2114}

\concepttable{
    \conrow{容量与组织}{$1K \times 4$ 位 ($0.5$ KB);地址线 \red{$A_0 \sim A_9$} ($2^{10}$),数据线 \red{$D_1 \sim D_4$}}
    \conrow{位扩展}{8位系统需\red{两片并联},分别负责高 4 位和低 4 位}
}

由于只有一个$\overline{WE}$而没有$\overline{OE}$,因此2114 的读写模式由$\overline{CS}$和$\overline{WE}$共同决定:

\concepttable{
    \conrow{写模式}{$\overline{CS}=0, \overline{WE}=0$,数据端口为输入 (DIN)}
    \conrow{读模式}{$\overline{CS}=0, \overline{WE}=1$,数据端口为输出 (DOUT)}
    \conrow{待机模式}{$\overline{CS}=1$,输出为\red{高阻态 (Hi-Z)},实现总线隔离}
}

\subsubsection{2764}

\concepttable{
    \conrow{容量与组织}{地址线 \red{$A_0 \sim A_{12}$};数据线 \red{$D_0 \sim D_7$}。与 6264 引脚兼容。}
    \conrow{CE\#}{片选使能,低电平有效。控制芯片激活及低功耗模式。}
    \conrow{OE\#}{输出使能,低电平有效。读操作时打开输出缓冲器。}
    \conrow{PGM\#}{编程脉冲,\red{低电平有效}。烧录时施加负脉冲,正常读取时置高。}
    \conrow{Vpp}{编程电压引脚,烧录时需施加高压(12.5V 或 21V)。}
}

在对芯片进行数据烧录(编程)时,需要在PGM\#施加特定宽度(如50ms)的负脉冲,配合$V_{pp}$引脚的高压(通常12.5V或21V),将数据写入\red{浮栅晶体管}(EPROM没有$\overline{WE}$引脚,因为它在正常工作时不可写。)

\hlyellow{低位地址线用于片内寻址,高位地址线用于片选寻址:}

\img{images/image-2025-12-28-21-02-01.png}

\subsection{存储器扩展}

\entry{存储器扩展}{当单片芯片容量(字数 $N$ 或字长 $M$)不足以满足系统需求时,需通过多片组合进行扩展。}

\concepttable{
    \conrow{位扩展}{增加\red{字长},字数不变。$N \times M \rightarrow N \times (M \times k)$。各片\red{地址线、读写线并联},数据线分别连接 CPU 数据总线的不同位。}
    \conrow{字扩展}{增加\red{字数}(容量),字长不变。$N \times M \rightarrow (N \times k) \times M$。各片\red{数据线、低位地址线并联},高位地址线经\red{译码器}产生片选信号 $\overline{CS}$。}
    \conrow{位和字节扩展}{同时增加字长和字数。通常先通过位扩展组成满足字长要求的“存储组”,再通过字扩展(译码片选)连接各组。}
}

\subsubsection{位扩展}

当芯片字长小于 CPU 数据总线宽度时,通过多片并联增加\red{存储字长}。如两片 $1K \times 4$ 的 2114 并联可组成 $1K \times 8$ 存储器。

\imgleft[0.3]{images/image-2025-12-28-21-12-23.png}{

    \concepttable{
        \conrow{地址/片选/读写线}{全部\red{并联}。所有芯片接收相同地址,由同一片选信号同时选中,并同步执行读/写操作。}
        \conrow{数据线($D_n$)}{\red{分段连接}。各芯片的数据端分别连接 CPU 数据总线的不同位段(如一片接 $D_0 \sim D_3$,另一片接 $D_4 \sim D_7$)。}
    }

}

\subsubsection{字节扩展}

当芯片字长满足但容量不足时,增加\red{存储单元总数}(地址空间深度)。

\imgleft[0.3]{images/image-2025-12-28-21-19-26.png}{

    \concepttable{
        \conrow{地址线($A_n$)}{低位地址线全部\red{并联},用于片内寻址。}
        \conrow{数据线($D_n$)}{全部\red{并联}到系统数据总线(与位扩展的区别)。}
        \conrow{控制线}{读/写控制信号($\overline{OE}, \overline{WE}$)全部\red{并联}。}
        \conrow{片选线($\overline{CS}$)}{高位地址经\red{译码器}产生,各片\red{独立连接},确保同一时刻仅一处有效。}
    }
    \entry{核心逻辑}{通过高位地址译码,将不同芯片映射到系统内存空间的不同地址段(如 $0000H \sim 1FFFH$)。}
}

\subsubsection{位和字节扩展}

当芯片字长和容量均不足时,需同时进行扩展。

\imgleft[0.3]{images/image-2025-12-28-21-25-45.png}{
    \entry{芯片数计算}{所需总芯片数 $Z = (M/L) \times (N/K)$。其中 $M/L$ 为字扩展组数,$N/K$ 为位扩展组数。L 为单片字长,K 为单片字数。}
    \concepttable{
        \conrow{第一步:位扩展}{将 $N/K$ 片并联,地址/读写线并联,数据线分段连接,构成一个字长满足要求的存储组(Bank)。}
        \conrow{第二步:字扩展}{将 $M/L$ 个存储组级联,数据/低位地址线并联,高位地址经译码器产生各组的片选信号 $\overline{CS}$。}
    }
}

\subsection{地址译码电路设计(片选)}

\entry{译码电路设计}{
    1. \red{片内寻址}:根据芯片容量 $N$ 确定低位线数 $k$($2^k=N$),如 4KB 需 $A_{11} \sim A_0$。
    2. \red{片选逻辑}:高位地址通过译码器(如 74LS138)或逻辑门产生 $\mathrm{CS\#}$。
    3. \red{范围确定}:首地址为高位固定、低位全 0;末地址为高位固定、低位全 1。
} 

\imgleft[0.3]{images/image-2025-12-28-21-46-21.png}{
    \concepttable{
        \conrow{全译码}{CPU \red{全部地址线}均被利用。低位用于片内寻址,\red{剩余所有高位}参与译码产生片选信号 $\mathrm{CS\#}$。特点:地址\red{唯一性}(不重叠),地址空间连续。}
        \conrow{部分译码}{\red{仅利用部分高位地址}参与译码。存在未使用的“悬空”地址线(Don't Care)。特点:电路简单,但会导致\red{地址重叠}(镜像),即一个物理单元对应多个逻辑地址。}
    }
    \entry{地址重叠计算}{若有 $n$ 根高位地址线未参与译码,则一个物理单元对应 \red{$2^n$} 个重叠地址。}
}




\subsection{存储器与 CPU 连接}

\section{第七章\ I/O接口}

\subsection{概念}

\entry{端口}{接口电路中用于缓存数据、状态、及控制信息的部件,分为:\red{数据端口}、\red{状态端口}、\red{控制端口}。}
\entry{接口电路}{计算机系统中包含多个不同功能的接口电路,\red{每个接口电路又可能包含 1 个或多个端口}。}
\entry{寻址端口方法}{先找到端口所在的接口电路芯片(\red{片选}),在该芯片上找到具体要访问的端口(\red{片内地址})。若接口中仅有一个端口,则找到芯片即找到端口;若接口中有多个端口,则找到芯片后需再找端口。每个端口地址 = \red{片选地址(高位地址)} + \red{片内地址}。}
\entry{8086/8088 的 I/O 端口编址}{采用 \red{I/O 独立编址方式};I/O 操作只使用 20 根地址线中的 \red{16 根(A15~A0)};可寻址的 I/O 端口数为:\red{$2^{16}=64$K (65536) 个};I/O 地址范围为:\red{0~0FFFFH}。}

\subsection{8259A}
\entry{8259A 特性}{\red{功能}:8259A 是一个功能很强的中断扩充和多中断管理芯片,具有\red{中断扩展}、\red{自动提供中断类型码}、\red{中断优先级裁决}等中断管理功能。\red{可编程}:内部有多个寄存器以及功能部件都是可编程的,使用方便。\red{级联扩展}:单片可连接\red{8 个中断请求源},多片级联可扩展到\red{64 级中断}。通过编程可设置\red{中断触发方式}、\red{中断类型码}、\red{中断屏蔽方式}、\red{中断优先级方式}、\red{中断结束方式}等。}

\imgleft[0.4]{images/image-2025-12-25-20-04-51.png}{
    \entry{8259A 内部结构}{
    \red{数据总线缓冲器}:三态、双向、8 位寄存器。
    \red{读写控制逻辑}:接收 CPU 的读写控制信号。
    \red{级联缓冲/比较器}:支持单片或多片级联,主片/从片管理,最多可扩展到\red{64 级中断}。\par
    \red{控制逻辑}:向片内各部件发送控制信号,向 CPU 发送中断请求信号 \red{INT},接收 CPU 回送的 \red{INTA*} 信号,控制 8259A 进入中断管理状态。\par
    \red{中断请求寄存器(IRR)}:8 位寄存器,记录外部中断请求,\red{IRi} 有请求时 IRR 的相应位 \red{Di} 置 1,中断响应后清除。\par
    \red{中断屏蔽寄存器(IMR)}:\red{IMR} 中 \red{Di} 位为 1 时禁止对应 \red{IRi} 请求,为 0 时允许。\par
    \red{优先权判决器}:对 IRR 中未屏蔽的中断进行优先级比较,选出当前优先级最高的中断请求。\par
    \red{中断服务寄存器(ISR)}:记录 CPU 当前正在服务的中断标志,\red{IRi} 请求响应时 ISR 相应位置 1,复位由中断结束方式决定。
    }
}

\section{第八章 常用接口技术}

\subsection{8255A}

\entry{8255 功能介绍}{
        \red{8255} 是一种\red{可编程的并行通信接口芯片},可用于 \red{CPU} 和外设之间进行并行数据传输。内部有\red{三个 8 位的数据端口},有三种工作方式。端口号的 \red{A0A1} 为 \red{00}、\red{01}、\red{10} 分别表示读写 \red{A、B、C 口};\red{11} 表示只写控制寄存器。
    }

\imgleft[0.3]{images/image-2025-12-26-00-05-16.png}{
    \entry{并行通信}{
        指\red{多位数据同时进行传送}的方式,其特点是\red{传输速度快}。
    }
        \concepttable{
            \conrow{方式 0}{基本的输入/输出方式,A/B/C 口均可用,C 口仅需设置方向。}
            \conrow{方式 1}{选通的输入/输出方式,A/B 口可用,支持中断方式传输,C 口部分引脚用于控制和中断信号。}
            \conrow{方式 2}{选通的双向传输方式,仅 A 口可用,支持双向传输和中断,C 口部分引脚用于控制和中断信号。}
        }
        \concepttable{
            \conrow{A 口}{可工作在方式 0、1、2,方式 2 时支持双向传输。}
            \conrow{B 口}{可工作在方式 0、1,仅支持单向传输。}
            \conrow{C 口}{只能工作在方式 0,高四位和低四位可分开使用,方式 1/2 时部分引脚用于控制和中断。}
        }
}
\red{方式 1、2}:选通输入输出方式,可以\red{中断方式传输},且\red{C 口会固定的引脚用作控制联络信号和中断请求信号}。\par
\red{仅 A 口工作在方式 2 时,可以双向传输},A 口工作在方式 0、1 及 B、C 口只能单向传输。\par
\entry{控制字}{
    控制字分为端口的\red{方式选择控制字}(可使 8255 的 3 个数据端口工作在不同的方式)和 C 口的\red{按位置位和复位控制字}(可使 C 口的任意一位置位和复位)。
    \red{★控制字送入的端口为最后一个端口}
}
\vspace{-0.3cm}
\begin{center}
\begin{tikzpicture}[scale=0.65, every node/.style={font=\tiny, inner sep=0pt}]
    % Draw 8 boxes for bits
    \foreach \i in {0,...,7} {
        \draw (\i,0) rectangle ++(1,0.6);
    }
    % Bit labels (D7...D0)
    \foreach \i/\b in {0/D7,1/D6,2/D5,3/D4,4/D3,5/D2,6/D1,7/D0} {
        \node at (\i+0.5,0.3) {\b};
    }
    % Value labels
    \node[red] at (0.5,0.85) {1};
    % Draw a brace and label
    \draw [decorate,decoration={brace,amplitude=4pt,mirror},thick] (0,0) -- (8,0) node[midway,below=4pt, font=\tiny] {方式控制字(8255A)};
\end{tikzpicture}
\end{center}
 
\subsection{8253}

\entry{概述}{
    ★\red{8253} 是一种\red{可编程的计数器/定时器接口芯片},最高计数频率为 \red{$2$MHz},可用于产生各种\red{定时波形},也可用于对\red{外部事件计数}。内部有\red{三个独立的 $16$ 位减一计数器}(互不干扰,支持\blue{二进制 (Binary) }或 \blue{二-十进制 (BCD码)} 计数),通过设置控制字,各计数器可以工作于 \red{$6$ 种工作方式}。
}
功能简述:

\imgleft[0.4]{images/image-2025-12-29-14-11-40.png}{

\concepttable{
    \conrow{CLK}{基准信号,即输入的\red{计数脉冲源}。所有输出变化均以此为时间基准。}
    \conrow{OUT1 (定时)}{启动后输出维持一段时间电平后跳变,用于产生确定的\red{时间延迟}。}
    \conrow{OUT2 (分频)}{输入信号频率较高,输出频率较低,输出周期是输入周期的\red{整数倍}。}
    \conrow{OUT3 (方波)}{分频的特殊形式,输出占空比为 \red{$50\%$},形成连续的\red{方波}。}
}

}

\imgleft[0.4]{images/image-2025-12-29-14-06-37.png}{

\entry{系统总线接口 (连接 CPU)}{
    \concepttable{
        \conrow{$D_7 \sim D_0$}{双向数据总线,用于 CPU 读写控制字或计数值。}
        \conrow{$\overline{CS}$}{片选信号,低电平有效,由地址译码电路产生。}
        \conrow{$\overline{RD} / \overline{WR}$}{读/写信号,低电平有效,控制数据传输方向。}
        \conrow{$A_1, A_0$}{端口选择:$00, 01, 10$ 分别对应计数器 $0, 1, 2$;$11$ 为控制寄存器。}
    }
}
\entry{计数通道信号 (连接外设)}{
    \concepttable{
        \conrow{$CLK_{0-2}$}{时钟输入,计数脉冲源,\red{下降沿}触发减 $1$ 操作。}
        \conrow{$GATE_{0-2}$}{门控输入,用于启动、停止或暂停计数过程。}
        \conrow{$OUT_{0-2}$}{输出信号,计数完成或达到条件时输出特定波形。}
    }
}

}

\entry{控制寄存器}{只能进行\red{写操作},不能读。CPU 通过向此寄存器写入“控制字”来设定各通道的工作方式。}

\entry{计数通道结构}{包含三个独立的计数通道(0、1、2),内部由以下三部分配合实现“预置 $\rightarrow$ 减计数 $\rightarrow$ 读出”逻辑:}
\concepttable{
    \conrow{初值寄存器 (CR)}{\red{16 位}。用于预置计数的起始值。}
    \conrow{执行部件 (CE)}{\red{16 位减法计数器}。在 $CLK$ 信号驱动下进行减 1 计数。}
    \conrow{输出锁存器 (OL)}{\red{16 位}。用于锁存当前计数值供 CPU 读取,不影响计数进行。}
}

\red{8253 没有状态寄存器},CPU \red{无法直接读取}状态寄存器来获知当前工作状态或回读控制字。这与具有状态回读功能的 \red{8254} 不同。

\noindent
\begin{minipage}[t]{0.38\linewidth}
    \entry{端口地址分配}{8253 占用 4 个 I/O 端口。CPU 通过 $A_1, A_0$ 选择:}
    \concepttable{
        \conrow{00}{计数通道 0}
        \conrow{01}{计数通道 1}
        \conrow{10}{计数通道 2}
        \conrow{11}{控制字寄存器}
    }
\end{minipage}
\hfill
\begin{minipage}[t]{0.58\linewidth}
    \entry{读写物理寄存器}{读写指向不同物理寄存器:}
    \concepttable{
        \conrow{写 (OUT)}{送入 \red{初值寄存器 (CR)}。}
        \conrow{读 (IN)}{来自 \red{输出锁存器 (OL)}。}
    }
    \tiny 这种“双缓冲”设计保证计数时读数稳定。
\end{minipage}
\par\vspace{2pt}

\noindent
\begin{minipage}[t]{0.49\linewidth}
    \entry{实例 1:连续地址 (8088)}{
        \concepttable{
            \conrow{设定}{地址范围:$0380H \sim 0383H$。地址连续(步长为 1)。}
            \conrow{分析}{$0380H (\dots 00) \rightarrow A_1A_0 = 00$ \par $0381H (\dots 01) \rightarrow A_1A_0 = 01$ \par $0382H (\dots 10) \rightarrow A_1A_0 = 10$ \par $0383H (\dots 11) \rightarrow A_1A_0 = 11$}
            \conrow{连接}{CPU 的 $A_1, A_0$ 直接连接到 8253 的 $A_1, A_0$。每一个逻辑地址都对应一个物理端口,无地址间隙。}
        }
    }
\end{minipage}
\hfill
\begin{minipage}[t]{0.49\linewidth}
    \entry{实例 2:偶地址对齐 (8086)}{
        \concepttable{
            \conrow{设定}{地址序列:$0380H, 0382H, \sim 0386H$ \par 只使用偶数地址。}
            \conrow{分析}{$0380H (\dots 00 \mathbf{0}) \rightarrow A_2A_1 = 00$;$0382H (\dots 01 \mathbf{0}) \rightarrow A_2A_1 = 01$;$0384H (\dots 10 \mathbf{0}) \rightarrow A_2A_1 = 10$;$0386H (\dots 11 \mathbf{0}) \rightarrow A_2A_1 = 11$。}
            \conrow{连接}{CPU $A_0$ 恒为 0。CPU 的 $A_2, A_1$ 错位连接到 8253 的 $A_1, A_0$。配合 16 位数据总线字节对齐要求。}
        }
    }
\end{minipage}

\entry{读/写逻辑真值表}{
    \concepttable{
        \conrow{片选前提}{所有操作必须在 $\overline{CS}=0$ 时有效。}
        \conrow{写 ($\overline{WR}=0$)}{根据 $A_1, A_0$ 写入通道 0/1/2 的计数初值寄存器,或写入控制字寄存器。}
        \conrow{读 ($\overline{RD}=0$)}{根据 $A_1, A_0$ 读取通道 0/1/2 的当前计数值。注意:\red{控制字寄存器只写不读}。}
    }
}

\entry{8 位总线与 16 位计数器接口}{
    8253 内部的计数初值寄存器 (CR) 和输出锁存器 (OL) 都是 16 位的,但外部数据总线 ($D_7 \sim D_0$) 只有 8 位。必须通过\red{两次 I/O 操作}来完成一个 16 位数据的传输,由控制寄存器中的控制字来指定读写顺序(如:先读/写低 8 位,再读/写高 8 位)。
}

\subsubsection{控制字}

\begin{center}
\begin{tikzpicture}[scale=0.65, every node/.style={font=\tiny, inner sep=0pt}]
    % Draw 8 boxes for bits
    \foreach \i in {0,...,7} {
        \draw (\i,0) rectangle ++(1,0.6);
    }
    % Bit labels (D7...D0)
    \foreach \i/\b in {0/D7,1/D6,2/D5,3/D4,4/D3,5/D2,6/D1,7/D0} {
        \node at (\i+0.5,0.3) {\b};
    }
    % Function labels
    \node at (0.5,0.85) {SC1}; \node at (1.5,0.85) {SC0};
    \node at (2.5,0.85) {RL1}; \node at (3.5,0.85) {RL0};
    \node at (4.5,0.85) {M2};  \node at (5.5,0.85) {M1}; \node at (6.5,0.85) {M0};
    \node at (7.5,0.85) {BCD};
    
    % Braces
    \draw [decorate,decoration={brace,amplitude=3pt},thick] (0,1) -- (2,1) node[midway,above=2pt] {计数器选择};
    \draw [decorate,decoration={brace,amplitude=3pt},thick] (2,1) -- (4,1) node[midway,above=2pt] {读写操作};
    \draw [decorate,decoration={brace,amplitude=3pt},thick] (4,1) -- (7,1) node[midway,above=2pt] {工作方式};
    \draw [decorate,decoration={brace,amplitude=3pt},thick] (7,1) -- (8,1) node[midway,above=2pt] {计数制};
    
    % Bottom label
    \node at (4,-0.4) {8253 控制字格式};
\end{tikzpicture}
\end{center}

\concepttable{
    \conrow{$SC_1, SC_0$}{
        \textbf{计数器选择}。因共用端口(3 个计数器共用一个控制端口),需指定目标通道。\par
        \red{00}: CNT0; \red{01}: CNT1; \red{10}: CNT2; \red{11}: 非法。
    }
    \conrow{$RL_1, RL_0$}{
        \textbf{读写格式}。定义 CR/OL 的读写位宽及顺序。\par
        \red{00}: \textbf{锁存命令} (不停止计数读取); \red{01}: 仅低8位; \red{10}: 仅高8位; \red{11}: \textbf{先低后高} (16位常用)。
    }
    \conrow{$M_2 \sim M_0$}{
        \textbf{工作方式}。设定方式 0 $\sim$ 5。
    }
    \conrow{$BCD$}{
        \textbf{计数制}。\red{0}: 二进制 ($FFFFH$); \red{1}: BCD码 ($9999$)。
    }
}

\noindent
\begin{minipage}[t]{0.49\linewidth}
    \entry{实例 1:8 位读写模式}{
        \blue{需求}:计数器 0,方式 2,仅使用低 8 位,初值 100,二进制计数。地址:$70H \sim 73H$。
    }
    \concepttable{
        \conrow{SC}{\red{00} (选择计数器 0)}
        \conrow{RL}{\red{01} (只读/写低 8 位)}
        \conrow{M}{\red{010} (方式 2,分频器)}
        \conrow{BCD}{\red{0} (二进制计数)}
        \conrow{控制字}{$00010100B = \red{14H}$}
    }
    \blue{汇编实现}:
    \begin{itemize}
        \item \texttt{MOV AL, 14H}
        \item \texttt{OUT 73H, AL \ \ ;写控制字}
        \item \texttt{MOV AL, 100 \ \ ;初值 100}
        \item \texttt{OUT 70H, AL \ \ ;写低 8 位}
    \end{itemize}
\end{minipage}
\hfill
\begin{minipage}[t]{0.49\linewidth}
    \entry{实例 2:16 位读写模式}{
        \blue{需求}:计数器 1,方式 1,16 位 (先低后高),初值 1234,BCD 码。地址:$70H \sim 73H$。
    }
    \concepttable{
        \conrow{SC}{\red{01} (选择计数器 1)}
        \conrow{RL}{\red{11} (先低 8 位后高 8 位)}
        \conrow{M}{\red{001} (方式 1,单稳态)}
        \conrow{BCD}{\red{1} (BCD 码计数)}
        \conrow{控制字}{$01110011B = \red{73H}$}
    }
    \blue{汇编实现}:
    \begin{itemize}
        \item \texttt{MOV AL, 73H}
        \item \texttt{OUT 73H, AL \ \ ;写控制字}
        \item \texttt{MOV AX, 1234H \ ;BCD 码初值}
        \item \texttt{OUT 71H, AL \ \ ;写低 8 位}
        \item \texttt{MOV AL, AH}
        \item \texttt{OUT 71H, AL \ \ ;写高 8 位}
    \end{itemize}
\end{minipage}

\entry{锁存命令}{解决在计数器运行过程中读取数值不稳定的问题。通过向控制寄存器写入 $RL_1RL_0 = 00$ 的控制字,将当前计数值复制到 \red{输出锁存器 (OL)} 中保持不变,而 \red{执行部件 (CE)} 继续计数。}

\entry{锁存读出示例}{
    \concepttable{
    \conrow{锁存机制}{向控制口发送控制字,其中 $RL_1RL_0 = 00$。8253 接收后锁存当前值,不影响内部计数。}
    \conrow{实例分析}{读取计数器 0 当前值:$SC=00, RL=00$,其余位无关,控制字为 \red{00H}。}
    \conrow{代码流程}{
        1. \texttt{MOV AL, 00H} \par
        2. \texttt{OUT 73H, AL} (向控制口发送锁存命令) \par
        3. \texttt{IN AL, 70H} (从数据口读取低 8 位) \par
        4. \texttt{IN AL, 70H} (从数据口读取高 8 位)
    }
    \conrow{注意}{读出操作必须符合初始化时设定的 \red{RL} 格式(如 16 位模式需连读两次)。}
    }
}

\entry{相关名词解释}{
    \concepttable{
    \conrow{CLK 脉冲}{指 \red{CLK} 引脚上的信号单元。在计数过程中,每一个 \red{CLK} 脉冲的\red{下降沿}到来时,计数器减 1。}
    \conrow{计数器}{与“计数通道”同义}
    \conrow{时常数}{指通过指令写入计数器的值,等同于\red{初值}。输出波形的周期或延时由该值决定:$\text{时间} = \text{初值} \times T_{CLK}$。}
    }
}

\subsubsection{方式 0}

\entry{功能定义}{
    主要用于\red{定时中断}。给定时间 $t_0$,到达后输出信号通知 CPU。
}

\noindent
\begin{minipage}[t]{0.55\linewidth}
    \concepttable{
        \conrow{初始化}{写入方式控制字后,输出引脚 \red{OUT 变为低电平}。}
        \conrow{计数过程}{写入初值后开始减 1 计数,期间 \red{OUT 保持低电平}。}
        \conrow{计数结束}{当计数值减到 0 时,\red{OUT 立即跳变为高电平}。产生的\red{上升沿信号}通常连接 CPU 中断请求引脚。}
    }
\end{minipage}%
\hfill
\begin{minipage}[t]{0.44\linewidth}
    \red{GATE} 是硬件门控信号,在方式 0 中充当“计数使能开关”:
    
    \concepttable{
        \conrow{0}{暂停计数,计数器保持当前值不变,忽略 CLK 脉冲。}
        \conrow{1}{允许计数,计数器正常工作。}
    }
\end{minipage}

\imgleft[0.4]{images/image-2025-12-29-15-24-32.png}{
    \entry{时序分析 (基本过程)}{
        重点在于 \red{$N+1$} 个时钟周期的由来。
        \concepttable{
            \conrow{写入控制字}{$\overline{WR}$ 有效 $\to$ \red{OUT 变低} (起始状态)。}
            \conrow{写入初值}{第二个 $\overline{WR}$ 将初值 $N$ 写入 CR。}
            \conrow{载入时刻}{写入后第 1 个 CLK 下降沿,数据 CR $\to$ CE。}
            \conrow{计数过程}{载入后开始减 1 ($N \to 0$)。总时长 \red{$N+1$} 个 CLK。}
            \conrow{终值输出}{CE 减为 0 时,\red{OUT 变高}。}
        }
    }
}

\imgleft[0.4]{images/image-2025-12-29-15-19-48.png}{
    \entry{GATE 控制 (暂停)}{
        \concepttable{
            \conrow{GATE=0}{暂停计数 (CE 保持),OUT 保持低。}
            \conrow{GATE=1}{恢复计数 (从暂停值继续)。}
        }
    }
    \entry{结论}{利用 GATE 可\red{加长 OUT 低电平宽度}。实际时间 = 预置 + 暂停。}
}

\imgleft[0.4]{images/image-2025-12-29-15-22-52.png}{
    \entry{重写初值 (重启动)}{
        若在计数未结束时写入新初值,8253 会放弃当前进程,在下一个 CLK \red{重新装入新初值并重新开始}。同样实现了\red{加长 OUT 低电平宽度}的效果。
    }
    \entry{周期数}{写入时常数为 $N$ 时,OUT 低电平宽度为 \red{$N+1$} 个 CLK 周期。}
    \entry{计数完成后}{CE 寄存器会变成FFFFH,OUT(OL 锁存器)一直输出高电平,但是 CE 持续在-1 计数}
}

\entry{方式 0 编程示例}{
    已知端口 $40H \sim 43H$,计数器 0,方式 0,初值 1500,二进制计数。
}
\concepttable{
    \conrow{控制字}{SC=00, RL=11, M=000, BCD=0 $\to$ \red{30H}}
    \conrow{时常数}{$1500 = 05DCH$ (低位 \red{DCH},高位 \red{05H})}
}
\blue{汇编实现}:
\begin{itemize}
    \item \texttt{MOV DX, 43H \ \ ;指向控制口}
    \item \texttt{MOV AL, 30H \ \ ;写控制字}
    \item \texttt{OUT DX, AL}
    \item \texttt{MOV DX, 40H \ \ ;指向计数器 0 数据口}
    \item \texttt{MOV AX, 1500 \ ;AX = 05DCH}
    \item \texttt{OUT DX, AL \ \ ;写低 8 位 (DCH)}
    \item \texttt{MOV AL, AH \ \ ;取高 8 位 (05H)}
    \item \texttt{OUT DX, AL \ \ ;写高 8 位}
\end{itemize}
\entry{核心逻辑}{
    写初值 $\to$ OUT 低 $\to$ 计数 $\to$ 0 $\to$ OUT 高。\red{定时时长 $= (N+1) \times T_{CLK}$。}可通过 GATE 硬件信号暂停,或重写初值延长定时。
}

\subsubsection{方式 1}

\entry{功能定义}{
    \red{单脉冲形成}。即产生一个宽度可控的\red{负脉冲}。

    \concepttable{
        \conrow{触发源}{\red{硬件触发}。区别于方式 0 的软件写入触发,方式 1 必须由 \red{GATE} 引脚的\red{上升沿}(由低电平变高电平)来触发。}
        \conrow{脉冲宽度}{输出负脉冲的宽度为 \red{$N \times T_{CLK}$},其中 $N$ 为预置的初值。}
    }

}

\imgleft[0.4]{images/image-2025-12-29-15-43-06.png}{

    \concepttable{
        \conrow{初始化}{写入方式控制字后 \red{OUT 变高};写入初值后 OUT 保持高电平。}
        \conrow{触发时刻}{\red{GATE 上升沿}触发。在随后的下一个 CLK 下降沿,OUT 由高变低。}
        \conrow{计数与结束}{计数器减 1 期间 OUT 为低;减至 0 时 OUT 跳变为高。低电平宽度 $= N \times T_{CLK}$。}
    }
    \entry{可重触发性}{
        在脉冲未结束时,若 \red{GATE 再次出现上升沿},8253 会在下一个 CLK 将初值寄存器 (CR) 的值\red{重新装入}执行单元 (CE),使计数器重新开始,从而\red{延长 OUT 低电平宽度}。
    }

}

\imgleft[0.4]{images/image-2025-12-29-15-43-35.png}{

\entry{修改初值 (方式 1)}{在脉冲输出过程中,若 CPU 修改计数初值:}
\concepttable{
    \conrow{系统响应}{新初值 $N_{new}$ 仅存入 \red{初值寄存器 (CR)}。当前计数器 (CE) 不受影响,继续按旧值 $N_{old}$ 减至 0,当前脉冲宽度不变。}
    \conrow{生效时刻}{仅在 \red{下一次 GATE 上升沿} 到来时,新值 $N_{new}$ 才从 CR 装入 CE。}
    \conrow{结论}{写入新初值\red{不会立即重启}计数,而是预置给下一次触发使用。}
}

}


\noindent{\setlength{\tabcolsep}{2pt}
    \begin{tabular}{@{}p{0.48\linewidth}|p{0.48\linewidth}@{}}
    \hline
    \centering \blue{方式 0 (软件触发)} & \centering \blue{方式 1 (硬件触发)} \tabularnewline \hline
    写初值操作直接启动计数 & GATE 上升沿触发启动 \\
    GATE \red{电平}敏感:1 计数,0 暂停 & GATE \red{边沿}敏感:上升沿重触发 \\
    低电平宽度为 \red{$N+1$} 个 CLK & 低电平宽度为 \red{$N$} 个 CLK \\ \hline
    \end{tabular}}
    \concepttable{
    \conrow{相同点}{均为减 1 计数;工作时输出低电平;均具备定时功能。}
}

\noindent
\begin{minipage}[t]{0.49\linewidth}
    \entry{实例 1-单一通道编程}{\par
        \blue{需求}:计数器 2,方式 1,初值 15,仅低 8 位。地址:控制口 \red{COUNTD},计数器 2 \red{COUNTC}。
    }
    \concepttable{
        \conrow{控制字}{$SC=10, RL=01, M=001, BCD=0 \to \red{92H}$}
        \conrow{汇编实现}{
            \texttt{MOV AL, 92H} \par
            \texttt{OUT COUNTD, AL \ ;写控制字} \par
            \texttt{MOV AL, 15} \par
            \texttt{OUT COUNTC, AL \ ;写初值}
        }
    }
\end{minipage}
\hfill
\begin{minipage}[t]{0.49\linewidth}
    \entry{实例 2-波形分析与混合}{\par
        \blue{需求}:根据波形反推。OUT0 为方式 0,OUT1 为方式 1。初值均为 7。
    }
    \concepttable{
        \conrow{CNT0 控制字}{$SC=00, RL=01, M=000 \to \red{10H}$}
        \conrow{CNT1 控制字}{$SC=01, RL=01, M=001 \to \red{52H}$}
        \conrow{汇编实现}{
            \texttt{MOV AL, 10H; OUT COUNTD, AL} \par
            \texttt{MOV AL, 52H; OUT COUNTD, AL} \par
            \texttt{MOV AL, 7} \par
            \texttt{OUT COUNTA, AL \ ;CNT0 初值} \par
            \texttt{OUT COUNTB, AL \ ;CNT1 初值}
        }
    }
\end{minipage}

\subsubsection{方式 2}

\entry{功能定义}{
    分频脉冲形成。作为 \red{$N$ 分频器}使用,产生\red{连续的、周期性的}负脉冲。每输入 $N$ 个时钟脉冲,输出 $1$ 个低电平脉冲。
}

\noindent
\begin{minipage}[t]{0.49\linewidth}
\vspace{-0.2cm}
\concepttable{
    \conrow{初始化}{写入控制字后 \red{OUT 立即变高}。写入初值 $N$ 后,在下一个 CLK 下降沿将初值装入执行单元 (CE)。}
    \conrow{计数过程}{计数器从 $N$ 开始递减。在 $N$ 减至 $1$ 期间,\red{OUT 保持高电平}。}
    \conrow{关键跳变}{当计数值减到 \red{$1$} 时(注意不是 $0$),\red{OUT 跳变为低电平},并维持 $1$ 个时钟周期 ($T_{CLK}$)。}
    \conrow{自动重装}{经过 $1$ 个周期的低电平后,\red{计数器自动重新装入初值 $N$,OUT 恢复高电平,开始新一轮计数。}}
}

\end{minipage}
\hfill
\begin{minipage}[t]{0.49\linewidth}

\entry{结论}{
    方式 2 输出的是周期信号,周期 $T = N \times T_{CLK}$。其中高电平持续 $N-1$ 个周期,低电平持续 $1$ 个周期。
}
\vspace{0.5cm}
\entry{GATE 硬件控制}{
    \concepttable{
        \conrow{GATE=1}{允许计数,正常执行分频功能。}
        \conrow{GATE=0}{禁止计数。若在计数过程中变低,OUT 立即变高;恢复为 $1$ 后,计数器将\red{重新装入初值}开始计数。}
    }
}

\end{minipage}

\entry{分频}{
        输出信号频率 $f_{out} = f_{in} / N$。
}

\imgleft[0.4]{images/image-2025-12-31-18-50-39.png}{

    \concepttable{
        \conrow{启动阶段}{$\overline{WR}$ 写入初值 $N$。下一个 CLK 下降沿,CE 载入 $N$,OUT 保持高电平。}
        \conrow{计数阶段}{CLK 驱动 CE 递减 ($N \to \dots \to 1$)。当 \red{CE=1} 时,\red{OUT 变为低电平}。}
        \conrow{循环阶段}{维持 $1$ 个 CLK 低电平后,CE 自动重装初值 $N$,OUT 恢复高电平,周而复始形成连续脉冲。}
    }

}

\imgleft[0.4]{images/image-2025-12-31-18-55-24.png}{
    \entry{GATE 信号控制与硬件同步}{
        方式 2 下,外部硬件信号 \red{GATE} 对计数过程具有控制和同步作用。
    }
    \concepttable{
        \conrow{GATE=0}{强制停止。计数器停止计数,\red{OUT 强制变为高电平}(若当前为低则立即拉高)。}
        \conrow{GATE=1}{允许计数,正常执行分频。}
        \conrow{GATE 上升沿}{触发\red{硬件同步}。无论当前计数值,立即在下一个 CLK \red{重新装入初值 $N$,从头开始计数}。常用于使计数周期与外部信号对齐。}
    }
}

\imgleft[0.4]{images/image-2025-12-31-18-57-39.png}{

    \entry{修改初值}{
        在计数过程中,若 CPU 通过软件修改计数初值:
    }
    \concepttable{
        \conrow{本次周期}{新初值仅存入 \red{初值寄存器 (CR)}。执行单元 (CE) 仍按旧值继续计数,直到完成当前周期(含低电平脉冲),不会出现“半截”脉冲。}
        \conrow{生效时刻}{在当前周期结束、触发\red{自动重装}时,新值才从 CR 装入 CE。}
        \conrow{对比方式 0}{方式 0 写入即生效(立即重启);方式 2 写入后\red{下周期生效},确保输出波形的完整性。}
    }

}

\entry{方式 2 特性总结}{

    \concepttable{
        \conrow{初始状态}{写入控制字后,\red{OUT 立即变为高电平}。}
        \conrow{启动时机}{写入初值后的下一个 \red{CLK} 下降沿,初值装入 \red{CE} 并开始减 1 计数。}
        \conrow{输出波形}{当计数减至 \red{1} 时,输出一个宽度为 \red{1 个 $T_{CLK}$} 的负脉冲。}
        \conrow{分频特性}{输出为连续周期信号,周期 \red{$T = N \times T_{CLK}$},频率 \red{$f_{out} = f_{in} / N$}。}
        \conrow{同步方式}{\blue{软件同步}:写入新初值,当前周期结束后生效;\blue{硬件同步}:\red{GATE 上升沿}触发立即重装初值并重新开始。}
        \conrow{GATE 逻辑}{\red{0}:停止计数且 \red{OUT} 变高;\red{1}:允许计数。}
    }

}

\subsubsection{方式 2 - 编程实例}

\entry{实例:5 分频器}{
    \blue{需求}:计数器 1,方式 2,初值 5,二进制,仅低 8 位。
}
\concepttable{
    \conrow{控制字}{SC=01, RL=01, M=010, BCD=0 $\to$ \red{54H}}
    \conrow{汇编实现}{
        \texttt{MOV AL, 54H} \par
        \texttt{OUT COUNTD, AL \ ;写控制字} \par
        \texttt{MOV AL, 5} \par
        \texttt{OUT COUNTB, AL \ ;写初值}
    }
}

\entry{波形特征}{
    总周期 $= N = 5$ 个 $T_{CLK}$;高电平持续 $N-1 = 4$ 个周期;低电平持续 $1$ 个周期。OUT1 在 $05, 04, 03, 02$ 期间为高,在 $01$ 期间为低,然后循环。
}

\subsubsection{方式 2 - 时常数计算}

\blue{频率已知}:$N = \dfrac{f_{in}}{f_{out}}$

\blue{周期已知}:$N = \dfrac{T_{out}}{T_{in}}$

\noindent
\begin{minipage}[t]{0.32\linewidth}
    \entry{实例 1}{
        \blue{已知}:$f_{in} = 2$ MHz,$f_{out} = 1$ kHz
    }
    \concepttable{
        \conrow{计算}{$N = \dfrac{2 \times 10^6}{1 \times 10^3}$\par
        $= \red{2000}$}
    }
\end{minipage}%
\hfill
\begin{minipage}[t]{0.32\linewidth}
    \entry{实例 2}{
        \blue{已知}:$T_{in} = 1\ \mu$s,$T_{out} = 1.3$ ms
    }
    \concepttable{
        \conrow{计算}{$1.3$ ms $= 1300\ \mu$s \par 
        \vspace{3pt}
        $N = \dfrac{1300}{1} = \red{1300}$}
    }
\end{minipage}%
\hfill
\begin{minipage}[t]{0.32\linewidth}
    \entry{实例 3}{
        \blue{已知}:$f_{in} = 1$ MHz ($T_{in} = 1\ \mu$s)
    }
    \concepttable{
        \conrow{波形图}{低电平 1.5 ms,高 3 ms}
        \conrow{计算}{$T_{out} = 4.5$ ms $= 4500\ \mu$s \par $N = \red{4500}$}
    }
\end{minipage}

\subsubsection{方式 3}

\entry{功能定义}{
    \red{方波信号发生器}。与方式 2 类似,也是一种分频器,但输出对称或近似对称的方波。输出信号周期 $T_{out} = N \times T_{CLK}$。
}

\noindent
\begin{minipage}[t]{0.49\linewidth}

    \img[\linewidth]{images/image-2025-12-31-20-50-11.png}

    \concepttable{
        \conrow{N 为偶数}{输出\red{完全对称}方波。高电平持续 $N/2$ 个 $T_{CLK}$,低电平持续 $N/2$ 个 $T_{CLK}$。}
        \conrow{N 为奇数}{输出\red{近似对称}方波。高电平持续 $(N+1)/2$ 个 $T_{CLK}$,低电平持续 $(N-1)/2$ 个 $T_{CLK}$。}
    }

\end{minipage}
\hfill
\begin{minipage}[t]{0.49\linewidth}

    \img[\linewidth]{images/image-2025-12-31-20-51-26.png}
    \vspace{4pt}
    \entry{GATE 硬件控制}{
        与方式 2 逻辑一致:
        \concepttable{
            \conrow{GATE=1}{允许计数,正常输出方波。}
            \conrow{GATE=0}{禁止计数,\red{OUT 立即跳变为高电平}。}
            \conrow{GATE 上升沿}{触发硬件同步,使计数器\red{重新装入初值}开始计数。}
        }   
    }

\end{minipage}

\entry{修改初值}{
    在计数过程中写入新初值,不会立即影响当前输出。新初值将在\red{当前半个周期(高电平或低电平)结束}、发生电平跳变时装入执行单元。
}

\noindent
\begin{minipage}[t]{0.49\linewidth}

    \img[\linewidth]{images/image-2025-12-31-20-56-48.png}

    \entry{方式 3 偶数计数逻辑}{
        当计数初值 $N$ 为偶数时,8253 内部执行单元 (CE) 采用\red{减 2 计数}方式以实现对称方波。
    }
    \concepttable{
        \conrow{初始化}{写入方式 3 控制字后,\red{OUT 端初始为高电平}。}
        \conrow{前半周期 (高)}{写入偶数初值 $N$ 后,CE 从 $N$ 开始每隔一个 CLK \red{减 2}。当 $CE=0$ 时,OUT 翻转为低,\red{并自动重装初值 $N$}。}
        \conrow{后半周期 (低)}{CE 再次从 $N$ 开始每脉冲\red{减 2}。当 $CE=0$ 时,OUT 翻转为高,并再次重装初值 $N$。}
        \conrow{波形特征}{高低电平宽度相等,均为 $(N/2) \times T_{CLK}$,输出为\red{完全对称方波}。}
    }

\end{minipage}%
\hfill
\begin{minipage}[t]{0.49\linewidth}

    \img[\linewidth]{images/image-2025-12-31-21-02-18.png}

    \entry{方式 3 奇数计数逻辑}{
        当初值 $N$ 为奇数时,8253 内部通过修正逻辑处理不对称性。
    }
    \vspace{0.65cm}
    \concepttable{
        \conrow{第一阶段 (高)}{装入时硬件自动将 $N$ 修正为偶数 $N-1$,并进行“减 2”计数。实际输出高电平时间为 $(N+1)/2 \times T_{CLK}$(由于修正周期的存在)。}
        \conrow{第二阶段 (低)}{计数器重新装入修正后的偶数 $N-1$,进行“减 2”计数。实际输出低电平时间为 $(N-1)/2 \times T_{CLK}$。}
    }

\end{minipage}

\noindent
\begin{minipage}[t]{0.49\linewidth}

    \img[\linewidth]{images/image-2025-12-31-20-56-48.png}

    \entry{方式 3 偶数初值时序分析 ($N=4$)}{
     
        \concepttable{
            \conrow{初始化}{写入控制字后,\red{OUT} 端初始化为高电平。}
            \conrow{写入初值}{通过 $\overline{WR}$ 信号的负脉冲将初值 \red{$N=4$} 写入。}
            \conrow{启动延迟}{写入初值后的下一个 \red{CLK} 下降沿,计数器正式开始工作。}
            \conrow{计数逻辑}{执行单元 (CE) 采用 \red{减 2} 方式变化:$04 \to 02 \to 00$。}
            \conrow{波形输出}{CE 递减期间 \red{OUT 保持高电平} (2 个周期);减至 $00$ 后,OUT 变低,CE 自动重装为 $04$ 并再次减至 $00$,此时 \red{OUT 保持低电平} (2 个周期)。}
        }

    }

    \entry{应用场景}{方式 3 模式非常适合生成\red{低频时钟信号},例如将系统高频晶振分频后提供给低速外设使用。}

\end{minipage}%
\hfill
\begin{minipage}[t]{0.49\linewidth}

\img[\linewidth]{images/image-2025-12-31-21-02-18.png}

\vspace{3pt}

\entry{方式 3 奇数初值时序分析 ($N=5$)}{
    \concepttable{
        \conrow{高电平阶段}{宽度为 \red{3} 个 $T_{CLK}$。内部状态:$04 \to 02 \to 00$。对应 $(N+1)/2$。}
        \conrow{低电平阶段}{宽度为 \red{2} 个 $T_{CLK}$。内部状态:$04 \to 02$。对应 $(N-1)/2$。}
        \conrow{波形特征}{高电平比低电平宽一个时钟周期,形成\red{近似对称}的方波。}
    }
}

\end{minipage}

\entry{方式 3 特性总结}{
    \concepttable{
        \conrow{初始状态}{写入控制字后,\red{OUT} 立即变高。}
        \conrow{偶数初值 $N$}{高电平 $N/2$,低电平 $N/2$,输出\red{完全对称}方波。}
        \conrow{奇数初值 $N$}{高电平 $(N+1)/2$,低电平 $(N-1)/2$,输出\red{近似对称}方波。}
        \conrow{功能本质}{产生周期为 $N$ 的方波(即 \red{$N$ 分频})。}
        \conrow{软件同步}{写入新初值后,不立即打断当前半周期,需等待当前半周期结束后在下一次翻转重装时生效。}
        \conrow{硬件同步}{\red{GATE 上升沿}触发,立即复位计数器并从高电平周期的起始点重新计数(相位同步)。}
        \conrow{GATE 控制}{\red{0}:计数停止,\red{OUT 保持当前电平不变}(注意:方式 2 是强制变高);\red{1}:允许工作。}
    }
}

\entry{\hlblue{方式 3 编程示例 01:基本设计}}{
    \blue{场景}:8254 计数器 0 连接 CPU 的 \red{5 MHz} 时钟,输出 \red{0.5 MHz} 方波。
}

\concepttable{
    \conrow{参数计算}{输入频率 $f_{in} = 5$ MHz;输出频率 $f_{out} = 0.5$ MHz;分频系数 $N = f_{in}/f_{out} = 5/0.5 = \red{10}$。}
    \conrow{控制字}{SC=00, RL=01 (低 8 位), M=011 (方式 3), BCD=0 $\to$ \red{16H}}
    \conrow{汇编实现}{
        \texttt{MOV DX, COUNTD} \par
        \texttt{MOV AL, 16H} \par
        \texttt{OUT DX, AL \ \ ;写控制字} \par
        \texttt{MOV DX, COUNTA} \par
        \texttt{MOV AL, 10} \par
        \texttt{OUT DX, AL \ \ ;写初值}
    }
}

\entry{波形特征}{
    $N=10$ 为偶数,输出\red{完全对称方波}。高电平持续 $5$ 个 $T_{CLK}$,低电平持续 $5$ 个 $T_{CLK}$。周期 $T_{out} = 10 \times 0.2\ \mu\text{s} = 2\ \mu\text{s}$,对应频率 $0.5$ MHz。
}

\entry{\hlblue{方式 3 - 编程示例 02:级联方案}}{

\entry{级联设计题目}{
    \blue{系统配置}:8086 CPU 主频为 \red{5 MHz}。
    
    \blue{任务要求}:
    \concepttable{
        \conrow{任务 A}{计数器 0 输出 \red{1 MHz} 方波。}
        \conrow{任务 B}{计数器 2 产生 \red{1 Hz} 的单脉冲周期信号。}
    }
}

\noindent
\begin{minipage}[t]{0.33\linewidth}
    \entry{\hlred{级联方式 1:串行级联}}{

        \entry{硬件连接逻辑}{
            \concepttable{
                \conrow{计数器 1}{输入:\red{5 MHz} 系统时钟;初值:$N_1=5000$;输出:\red{1 kHz}。}
                \conrow{计数器 2}{输入:来自计数器 1 的输出 (\red{1 kHz});初值:$N_2=1000$;输出:\red{1 Hz}。}
            }
        }

        \entry{关键连接}{
            计数器 1 的 \red{OUT1} $\to$ 计数器 2 的 \red{CLK2}。
        }

        \entry{编程实现}{
            \concepttable{
                \conrow{计数器 1}{控制字:\red{76H};初值:\red{5000}。}
                \conrow{计数器 2}{控制字:\red{B4H};初值:\red{1000}。}
            }
        }

        \blue{汇编代码}:
        \begin{itemize}
            \item \texttt{MOV AL, 76H}
            \item \texttt{OUT COUNTD, AL}
            \item \texttt{MOV AL, B4H}
            \item \texttt{OUT COUNTD, AL}
            \item \texttt{MOV AX, 5000}
            \item \texttt{OUT COUNTB, AL}
            \item \texttt{MOV AL, AH}
            \item \texttt{OUT COUNTB, AL}
            \item \texttt{MOV AX, 1000}
            \item \texttt{OUT COUNTC, AL}
            \item \texttt{MOV AL, AH}
            \item \texttt{OUT COUNTC, AL}
        \end{itemize}

    }
\end{minipage}%
\hfill
\begin{minipage}[t]{0.33\linewidth}
    \entry{\hlred{级联方式 2:并行输入}}{

        \entry{硬件连接逻辑}{
            \concepttable{
                \conrow{计数器 0}{输入:\red{5 MHz};初值:$N_0=5$;输出:\red{1 MHz}(任务 A)。}
                \conrow{计数器 1}{输入:\red{5 MHz}(并行);初值:$N_1=5000$;输出:\red{1 kHz}。}
                \conrow{计数器 2}{输入:来自计数器 1 (\red{1 kHz});初值:$N_2=1000$;输出:\red{1 Hz}(任务 B)。}
            }
        }

        \entry{关键连接}{
            计数器 0 与 1-2 组并行;计数器 1 的 \red{OUT1} $\to$ 计数器 2 的 \red{CLK2}。
        }

        \entry{编程实现}{
            \concepttable{
                \conrow{计数器 0}{控制字:\red{16H};初值:\red{5}。}
                \conrow{计数器 1}{控制字:\red{76H};初值:\red{5000}。}
                \conrow{计数器 2}{控制字:\red{B4H};初值:\red{1000}。}
            }
        }

        \blue{汇编代码}:
        \begin{itemize}
            \item \texttt{MOV AL, 16H}
            \item \texttt{OUT COUNTD, AL}
            \item \texttt{MOV AL, 76H}
            \item \texttt{OUT COUNTD, AL}
            \item \texttt{MOV AL, B4H}
            \item \texttt{OUT COUNTD, AL}
            \item \texttt{MOV AL, 5}
            \item \texttt{OUT COUNTA, AL}
            \item \texttt{MOV AX, 5000}
            \item \texttt{OUT COUNTB, AL}
            \item \texttt{MOV AL, AH}
            \item \texttt{OUT COUNTB, AL}
            \item \texttt{MOV AX, 1000}
            \item \texttt{OUT COUNTC, AL}
            \item \texttt{MOV AL, AH}
            \item \texttt{OUT COUNTC, AL}
        \end{itemize}

    }
\end{minipage}
\hfill
\begin{minipage}[t]{0.33\linewidth}

\entry{\hlred {级联方式3:完全串联级联}}{
    
    \entry{硬件连接逻辑}{
        充分利用计数器 0 的中间成果,形成 \red{C0 $\to$ C1 $\to$ C2} 的全级联结构。
        \concepttable{
            \conrow{计数器 0}{输入:\red{5 MHz};初值:$N_0=5$;输出:\red{1 MHz}(任务 A)。}
            \conrow{计数器 1}{输入:来自计数器 0 (\red{1 MHz});初值:$N_1=1000$;输出:\red{1 kHz}。}
            \conrow{计数器 2}{输入:来自计数器 1 (\red{1 kHz});初值:$N_2=1000$;输出:\red{1 Hz}(任务 B)。}
        }
    }

    \entry{关键连接}{
        计数器 0 的 \red{OUT0} $\to$ 计数器 1 的 \red{CLK1};计数器 1 的 \red{OUT1} $\to$ 计数器 2 的 \red{CLK2}。总分频比 $= 5 \times 1000 \times 1000 = 5,000,000$。
    }

    \entry{编程实现}{
        \concepttable{
            \conrow{计数器 0}{控制字:\red{16H};初值:\red{5}。}
            \conrow{计数器 1}{控制字:\red{76H};初值:\red{1000}。}
            \conrow{计数器 2}{控制字:\red{B4H};初值:\red{1000}。}
        }
    }

    \blue{汇编代码}:
    \begin{itemize}
        \item \texttt{MOV AL, 16H}
        \item \texttt{OUT COUNTD, AL}
        \item \texttt{MOV AL, 76H}
        \item \texttt{OUT COUNTD, AL}
        \item \texttt{MOV AL, B4H}
        \item \texttt{OUT COUNTD, AL}
        \item \texttt{MOV AL, 5}
        \item \texttt{OUT COUNTA, AL}
        \item \texttt{MOV AX, 1000}
        \item \texttt{OUT COUNTB, AL}
        \item \texttt{MOV AL, AH}
        \item \texttt{OUT COUNTB, AL}
        \item \texttt{MOV AX, 1000}
        \item \texttt{OUT COUNTC, AL}
        \item \texttt{MOV AL, AH}
        \item \texttt{OUT COUNTC, AL}
    \end{itemize}

}

\end{minipage}
}

\entry{方案对比}{
    \concepttable{
        \conrow{方案 1 (串行)}{C1 分频系数大 (5000),C0 与 C1-C2 组独立。}
        \conrow{方案 2 (并行)}{C0 独立完成 1 MHz,C1-C2 组并行接入 5 MHz。}
        \conrow{方案 3 (完全串联)}{充分利用 C0 中间成果,形成完整级联链路。适合需要多级分频的场景。}
    }
}

\entry{\hlblue{8253 控制字与工作方式:共性总结}}{
    \entry{1) 初始化/逻辑复位}{
        向\red{控制寄存器}写入控制字时,芯片内部执行\red{逻辑复位}:清除旧配置,\red{OUT} 引脚回到该方式规定的初始状态(通常为高或低),为新一轮计数做准备。
    }

    \entry{2) GATE 信号的有效形式(按触发源分类)}{
        \concepttable{
            \conrow{电平控制(软件触发)}{\red{方式 0、方式 4}:\red{GATE=1} 允许计数,\red{GATE=0} 暂停计数;计数启动主要依靠\red{软件写入初值}。}
            \conrow{上升沿触发(硬件触发)}{\red{方式 1、方式 5}:必须检测到 \red{GATE} 从 0$\to$1 的\red{上升沿}才启动计数,用于外部硬件触发定时。}
            \conrow{双重触发(分频/发生器)}{\red{方式 2、方式 3}:既可通过\red{软件写初值}启动/更新周期,也可通过 \red{GATE 上升沿}强制\red{重同步}(从头开始当前周期)。}
        }
    }

    \entry{3) CR 初值装入 CE 的时刻(按载入时机分类)}{
        \concepttable{
            \conrow{方式 0、方式 4}{写入初值后,在\red{下一个 CLK}(触发沿)时,\red{CR$\to$CE} 装入并开始计数。}
            \conrow{方式 1、方式 5}{写入初值后\red{不立即装入};需等待 \red{GATE 上升沿}到来后,在\red{下一个 CLK} 才 \red{CR$\to$CE} 装入开始计数。}
            \conrow{方式 2、方式 3}{三种载入来源:\red{写初值后启动装入};每周期结束时\red{自动重装};\red{GATE 上升沿}可强制装入实现同步。}
        }
    }

    \entry{4) 计数到 0 后的行为(是否周期性)}{
        \concepttable{
            \conrow{单次计数(不自动重装)}{\red{方式 0、1、4、5}:到达终值并完成 OUT 动作后,\red{内部计数并不停止},仍会继续减 1(例如 $0\to FFFFH \to FFFEH \to \cdots$),但 OUT 通常保持在该方式规定的终态/后续状态。}
            \conrow{循环计数(自动重装)}{\red{方式 2、3}:当 CE 到达终值后,芯片自动将 CR 中初值\red{再次装入} CE,进入下一周期,是形成连续波形的基础。}
        }
    }
}

% --- 手动换栏命令(如果需要强制换列)---
% \columnbreak 

\end{multicols*}

\end{document}
