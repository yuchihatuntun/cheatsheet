\documentclass[10pt, a4paper, landscape]{article}

% -------------------------------------------------
% 宏包引入
% -------------------------------------------------
\usepackage[fontset=mac]{ctex}       % 中文支持
\usepackage{multicol}   % 多分栏
\usepackage{calc}
\usepackage{ifthen}
\usepackage[landscape]{geometry} % 页面设置
\usepackage{amsmath,amsthm,amsfonts,amssymb} % 数学公式
\usepackage{color,graphicx,overpic} % 颜色与图片
\usepackage{hyperref}   % 超链接
\usepackage{enumitem}   % 列表环境控制
\usepackage{titlesec}   % 标题控制
\usepackage{bm}         % 加粗数学符号
\usepackage{xcolor}
\usepackage{tikz}       % 绘图
\usetikzlibrary{decorations.pathreplacing, positioning} % 加载brace装饰库
\setCJKmainfont{PingFang SC}
\setCJKsansfont{PingFang SC}
\setCJKmonofont{PingFang SC}

% -------------------------------------------------
% 自定义颜色
% -------------------------------------------------

\definecolor{myblue}{HTML}{003153} % 蓝色字
\definecolor{myred}{HTML}{85120F}  % 红色字
\definecolor{hlblue}{HTML}{7090D3} % 蓝色高亮
\definecolor{hlred}{HTML}{D66A83}  % 红色高亮
\definecolor{hlyellow}{HTML}{E3C79F}  % 奢金高亮
\definecolor{hlgreen}{HTML}{BED49D}  % 抹茶绿高亮

% -------------------------------------------------
% 极限空间压缩设置 (核心部分)
% -------------------------------------------------

% 1. 页边距设置为极小 (0.5cm)
\geometry{top=0.5cm,left=0.5cm,right=0.5cm,bottom=0.5cm}

% 2. 去掉段落首行缩进,改为段落间略微留空(可选,这里为了紧凑设为0)
\setlength{\parindent}{0pt}
\setlength{\parskip}{0pt}

% 3. 设置正文基础字体大小为 scriptsize (约8pt),如果还觉得大,可以改为 \tiny
\renewcommand{\baselinestretch}{0.9} % 压缩行间距
\let\oldfootnotesize\footnotesize
\renewcommand{\footnotesize}{\fontsize{7pt}{8pt}\selectfont}

% 4. 压缩列表环境 (Itemize/Enumerate) 的间距
\setlist{nolistsep} 
\setlist[itemize]{leftmargin=*}
\setlist[enumerate]{leftmargin=*}

% 5. 压缩标题间距
\titleformat{\section}{\bfseries\scriptsize\color{myblue}}{}{0em}{}[\hrule] % 标题带下划线,蓝色,省空间
\titlespacing*{\section}{0pt}{2pt}{1pt} % 上方留2pt,下方留1pt
\titleformat{\subsection}
    [runin] % 不换行
    {\bfseries\scriptsize} % 粗体、scriptsize,黑色字体
    {} % 不显示编号
    {0pt} % 标题与正文间距
    {\hlyellow} % 用hlyellow高亮命令包裹标题
    [] % 标题内容后无内容
\titleformat{\subsubsection}
    [runin] % 不换行
    {\bfseries\tiny} % 粗体、scriptsize,黑色字体
    {} % 不显示编号
    {0pt} % 标题与正文间距
    {\hlgreen} % 用hlgreen高亮命令包裹标题
    [] % 标题内容后无内容
\titlespacing*{\subsection}{0pt}{1pt}{0.5em} % 上方1pt,下方0.5em(水平间距)
\titlespacing*{\subsubsection}{0pt}{1pt}{0.5em} % 上方1pt,下方0.5em(水平间距)

% -------------------------------------------------
% 自定义命令
% -------------------------------------------------
% 颜色字
\newcommand{\red}[1]{\textbf{\textcolor{myred}{#1}}}  
\newcommand{\blue}[1]{\textbf{\textcolor{myblue}{#1}}} 
\newcommand{\entry}[2]{$\bullet$ \textbf{#1}: #2\par\vspace{0.5pt}}
% 高亮
\newcommand{\cbox}[2][yellow]{\begingroup\setlength{\fboxsep}{1pt}\colorbox{#1}{\strut#2}\endgroup}
\newcommand{\hlblue}[1]{\cbox[hlblue]{#1}}
\newcommand{\hlred}[1]{\cbox[hlred]{#1}}
\newcommand{\hlyellow}[1]{\cbox[hlyellow]{#1}}
\newcommand{\hlgreen}[1]{\cbox[hlgreen]{#1}} 
% 图片插入
\newcommand{\img}[2][0.9\linewidth]{%
    {\par\vspace{1pt}\centering\includegraphics[width=#1]{#2}\par\vspace{1pt}}%
}
% 左图右文 (参数: [图片宽度比例]{图片路径}{右侧文字内容})
\newcommand{\imgleft}[3][0.3]{%
    \noindent\begin{minipage}[t]{#1\linewidth}%
        \vspace{0pt}%
        \includegraphics[width=\linewidth]{#2}%
    \end{minipage}%
    \hfill%
    \begin{minipage}[t]{0.98\linewidth - #1\linewidth}%
        \vspace{0pt}%
        #3%
    \end{minipage}\par\vspace{2pt}%
}
% 左文右图 (参数: [图片宽度比例]{图片路径}{左侧文字内容})
\newcommand{\imgright}[3][0.3]{%
    \noindent\begin{minipage}[t]{0.98\linewidth - #1\linewidth}%
        \vspace{0pt}%
        #3%
    \end{minipage}%
    \hfill%
    \begin{minipage}[t]{#1\linewidth}%
        \vspace{0pt}%
        \includegraphics[width=\linewidth]{#2}%
    \end{minipage}\par\vspace{2pt}%
}
% -------------------------------------------------
% 新增:概念速查表专用命令
% -------------------------------------------------
% 表格容器
\newcommand{\concepttable}[1]{%
    {\setlength{\tabcolsep}{1.5pt}% 局部减小列间距
     \renewcommand{\arraystretch}{0.92}% 局部紧缩行距
     \par\vspace{2pt}{\color{myblue}\hrule height 0.6pt}\vspace{1pt}% 上边框(蓝色,0.6pt粗)
     \noindent\begin{tabular}{@{}p{0.22\linewidth}p{0.76\linewidth}@{}}%
     #1%
     \end{tabular}%
     \vspace{1pt}{\color{myblue}\hrule height 0.6pt}\par\vspace{2pt}}% 下边框(蓝色,0.6pt粗)
}
% 表格行 (参数: {概念名}{解释})
\newcommand{\conrow}[2]{\blue{#1} & #2 \\}




% -------------------------------------------------
% 正文
% -------------------------------------------------
\begin{document}

\tiny

% 三栏布局
\begin{multicols*}{3}

\section{量子物理基础}



\section{第二章\ PN结}

\section{第三章\ MOSFET}

\section{第四章\ 双极型晶体管}



\section{第五章\ 金半接触}

\subsection{肖特基接触}

\subsubsection{基本概念}

\entry{同质/异质结}{同种/不同材料形成的PN结。}
\entry{金半结}{金属与半导体的接触。}

\noindent
\begin{minipage}[t]{0.49\linewidth}
    \entry{整流接触}{
    \concepttable{
        \conrow{定义}{在半导体表面形成了表面势垒,也称为阻挡层。}
        \conrow{特性}{类似于PN结,具有单向导电性(整流作用)。}
        \conrow{命名}{这就是我们通常所说的\red{肖特基接触}。}
    }   
}
\end{minipage}
\hfill
\begin{minipage}[t]{0.49\linewidth}
\entry{非整流接触}{
    \concepttable{
        \conrow{定义}{在界面处形成了\red{反阻挡层},即高电导区。}
        \conrow{特性}{没有整流作用,电流可以双向自由流动。}
        \conrow{命名}{这就是我们通常所说的\red{欧姆接触}。}
    }
}
\end{minipage}

\red{电子将从功函数小的地方跑到功函数大的地方,空穴则相反。}

\imgleft[0.4]{images/image-2025-12-30-12-07-58.png}{
    \concepttable{
        \conrow{功函数 $\phi$}{电子从 $E_F$ 逸出到真空能级 $E_0$ 所需最小能量。}
        \conrow{金属功函数}{$e\phi_m = E_0 - (E_F)_m$}
        \conrow{半导体功函数}{$e\phi_s = E_0 - (E_F)_s$}
        \conrow{电子亲和能 $\chi$}{导带底 $E_c$ 电子逸出到真空能级 $E_0$ 所需能量。}
        \conrow{\red{能量关系}}{\red{$e\phi_s = \chi + \phi_n$},其中 $\phi_n = E_c - E_F$。$\chi$ 是固有属性,$\phi_s$ 会随掺杂改变(因为费米能级会因掺杂浓度变化)}
    }
}

\noindent
\begin{minipage}[t]{0.49\linewidth}

\img{images/image-2025-12-30-12-30-52.png}

\blue{N型肖特基接触 ($\phi_m > \phi_s$)}
\concepttable{
    \conrow{初始条件}{$\phi_m > \phi_s \implies E_{Fm} < E_{FN}$}
    \conrow{物理过程}{电子自发从 $N$ 型半导体流向金属}
    \conrow{电荷分布}{半导体侧施主失去电子带正电,形成\red{耗尽层};金属侧带负电}
    \conrow{能带弯曲}{表面 $n_s$ 降低,由 $n = N_c \exp[-(E_c - E_F)/kT]$ 知 $E_c$ \red{向上}弯曲}
    \conrow{势垒形成}{形成表面势垒 $e\phi_{B0}$(阻挡层),阻碍电子进入金属}
    \conrow{平衡状态}{热平衡建立,系统费米能级 $E_F$ 处处拉平}
}

\end{minipage}
\hfill
\begin{minipage}[t]{0.49\linewidth}

\img{images/image-2025-12-30-12-30-21.png}
\vspace{2pt}
\blue{P型肖特基接触 ($\phi_s > \phi_m$)}
\concepttable{
    \conrow{初始条件}{$\phi_s > \phi_m \implies E_{Fm} > E_{FP}$}
    \conrow{物理过程}{电子从金属流向半导体 (空穴从 $P$ 型流向金属)}
    \conrow{电荷分布}{半导体侧受主得到电子带负电,形成\red{耗尽层};金属侧带正电}
    \conrow{能带弯曲}{表面 $p_s$ 降低,能带 ($E_c, E_v$) \red{向下}弯曲}
    \conrow{势垒形成}{形成表面势垒 $e\phi_{B0}$(阻挡层),阻碍空穴进入金属}
    \conrow{平衡状态}{热平衡建立,系统费米能级 $E_F$ 处处拉平}
}

\end{minipage}

\textbf{施加偏压(以 N 型接触为例):}

\noindent
\begin{minipage}[t]{0.49\linewidth}
\img[0.5\linewidth]{images/image-2025-12-30-12-36-22.png}
\blue{正向偏压 (Metal +, Semi -)}
\concepttable{
    \conrow{势垒变化}{外加电压 $U$ 抵消内建电势,势垒降低为 $e(V_{bi} - U)$}
    \conrow{物理过程}{电子易于越过势垒从半导体流向金属}
    \conrow{电流特性}{产生巨大的正向电流}
}
\end{minipage}
\hfill
\begin{minipage}[t]{0.49\linewidth}
\img[0.6\linewidth]{images/image-2025-12-30-12-46-10.png}
\blue{反向偏压 (Metal -, Semi +)}
\concepttable{
    \conrow{势垒变化}{外加电压 $U$ 叠加在内建电势上,势垒增加为 $e(V_{bi} + U)$}
    \conrow{物理过程}{半导体侧电子无法越过更高的势垒}
    \conrow{电流特性}{金属侧电子受限于固定势垒 $e\phi_{B0}$,电流极小,反向截止}
}
\end{minipage}

\noindent
\begin{minipage}[t]{0.49\linewidth}
    \blue{N 型计算}
    \concepttable{
        \conrow{费米势}{$\phi_n = V_t \ln(\frac{N_c}{N_d})$}
        \conrow{肖特基势垒}{$e\phi_{\mathrm{B0}} = e\phi_{\mathrm{m}} - \chi$}
        \conrow{内建电势}{$V_{bi} = \phi_{\mathrm{m}} - \phi_{\mathrm{s}} = \phi_{\mathrm{B0}} - \phi_n$}
    }
\end{minipage}
\hfill
\begin{minipage}[t]{0.49\linewidth}
    \blue{P 型计算}
    \concepttable{
        \conrow{费米势}{$\phi_p = V_t \ln(\frac{N_v}{N_a})$}
        \conrow{肖特基势垒}{$e\phi_{\mathrm{B0}} = E_g - (e\phi_{\mathrm{m}} - \chi)$}
        \conrow{内建电势}{$V_{bi} = \phi_{\mathrm{s}} - \phi_{\mathrm{m}} = \phi_{\mathrm{B0}} - \phi_p$}
    }
\end{minipage}

\blue{通用特性 ($N$ 代表 $N_d$ 或 $N_a$)}
\concepttable{
    \conrow{参数说明}{$\phi_{\mathrm{B0}}$ 一般题干\blue{直接给值},否则按上表计算。}
    \conrow{耗尽层宽度}{$W(x_n)=\left[\frac{2\varepsilon\left(V_{bi}+V_{\mathrm{R}}\right)}{e N}\right]^{1 /2}$ (与单边突变结一致),其中 $V_R$ 为外加反向偏压。}
    \conrow{最大电场}{$E_{max} = \frac{e N W}{\varepsilon}$}
    \conrow{势垒电容}{$C= A \frac{\varepsilon}{W}=A\left[\frac{e \varepsilon N}{2\left(V_{bi}+V_{R}\right)}\right]^{1 / 2}$}
    \conrow{$C-V$ 特性}{$\frac{1}{C^{2}}=\frac{2}{e \varepsilon N A^{2}}\left(V_{R}+V_{bi}\right)$,可由曲线斜率求 $N$,截距求 $V_{bi}$}
}

\entry{整流特性}{正偏时半导体侧势垒降低,电流大;反偏时势垒升高,电流极小。由于金属电子浓度极高,金属侧势垒 $q\phi_{\mathrm{b}}$ 随偏压几乎不变。}

\subsubsection{非理想因素}

从这里开始讨论非理想因素,即为什么实际势垒不完全等于 $\phi_m - \chi$。

\noindent
\begin{minipage}[t]{0.49\linewidth}
    \entry{肖特基效应 (\red{镜像力降低})}{}
    \concepttable{
        \conrow{势能修正}{和大物一样,靠近金属的电荷会感应出镜像电荷,引入负电势能项 $-\frac{e}{16\pi\epsilon_s x}$,与电场叠加。}
        \conrow{\red{势垒降低}}{$\Delta \phi = \sqrt{\frac{eE}{4\pi\epsilon_s}}$}
        \conrow{总势能最高点}{$x_m = \sqrt{\frac{e}{16\pi \epsilon_s E}}$}
    }
\end{minipage}
\hfill
\begin{minipage}[t]{0.49\linewidth}
    \entry{界面态 (\red{费米能级钉扎})}{}
    \concepttable{
        \conrow{表面态}{禁带中由缺陷等引起的能级。施主型(失电子正电)、受主型(得电子负电)。}
        \conrow{中性能级}{$E_F < \phi_0$ 呈正电,$E_F > \phi_0$ 呈负电。}
        \conrow{物理机制}{若 $D_{it}$ 很大,表面态储存大量电荷,使 $E_F$ 被“钉扎”在 $\phi_0$ 附近,势垒高度几乎与 $\phi_m$ 无关。就是一个经验值了。}
    }
\end{minipage}

\subsubsection{电流-电压关系}

\entry{热电子发射理论}{}
\concepttable{
    \conrow{适用范围}{描述肖特基接触电流传输的主流模型(适用于 Si, GaAs 等高迁移率半导体)。}
    \conrow{核心假设}{只有\red{能量足够高}($E > E_F + e\phi_{Bn}$)($\phi_{Bn} = \phi_{B0} - \Delta \phi$,是修正后的肖特基势垒)的“热电子”才能从半导体进入金属,电流的大小取决于单位时间内能够“跳过”势垒高度的电子数量。}
}

\entry{电流分量分析}{
    \concepttable{
        \conrow{$J_{s \to m}$}{半导体 $\to$ 金属:电子需克服势垒 $e(V_{bi} - V_a)$。正偏时势垒降低,电流\red{指数级增加}。\par
        $J_{s \to m} = A^* T^2 \exp\left(\frac{-e\phi_{Bn}}{kT}\right) \exp\left(\frac{eV_a}{kT}\right)$}
        \conrow{$J_{m \to s}$}{金属 $\to$ 半导体:电子需克服势垒 $e\phi_{B0}$。势垒固定,此分量视为\red{常数}(反向饱和电流)。\par
        $J_{m \to s} = -A^* T^2 \exp\left(\frac{-e\phi_{Bn}}{kT}\right)$}
    }
}

\entry{有效理查德森常数 $A^*$}{
    \concepttable{
        \conrow{表达式}{$A^* = \frac{4\pi e m_n^* k^2}{h^3}$}
        \conrow{物理意义}{在理查德森常数中用有效质量 $m^*$ 代替 $m_0$,反映了晶格势场对电子运动的影响。}
    }
}

\imgleft[0.25]{images/image-2025-12-30-14-08-58.png}{
    \entry{肖特基二极管方程}{
        \concepttable{
            \conrow{总电流密度}{$J = J_{s \to m} + J_{m \to s} = J_{ST} \left[ \exp\left(\frac{eV_a}{kT}\right) - 1 \right]$}
            \conrow{饱和电流密度}{$J_{ST} = A^* T^2 \exp\left(\frac{-e\phi_{Bn}}{kT}\right)$\par
            $= A^* T^2 \exp\left(\frac{-e\phi_{B0}}{kT}\right)\exp\left(\frac{e\Delta \phi}{kT}\right)$\par
            因此,$J_{ST} \propto \exp\left(\frac{e\Delta \phi}{kT}\right)$}
        }
    }
}

\subsubsection{肖特基二极管与 PN 结对比}

从电流输运机制和数量级两个维度,对比了两种二极管的特性

\noindent
\begin{minipage}[t]{0.49\linewidth}
    \blue{肖特基二极管 (SBD)}
    \concepttable{
        \conrow{载流子类型}{\red{多子器件}}
        \conrow{电流机制}{热电子发射理论}
        \conrow{反向电流}{$J_{ST}$ 较大,随电压增加而增加 (非饱和)}
        \conrow{导通电压}{低 (约 0.3 V)}
        \conrow{开关速度}{\red{极快},无少子存储效应,仅受 $RC$ 限制}
        \conrow{应用}{高频检波、高速开关、肖特基箝位}
    }
\end{minipage}
\hfill
\begin{minipage}[t]{0.49\linewidth}
    \blue{PN 结二极管}
    \concepttable{
        \conrow{载流子类型}{少子器件}
        \conrow{电流机制}{少子扩散理论}
        \conrow{反向电流}{$J_S$ 极小,具有良好的饱和特性}
        \conrow{导通电压}{高 (约 0.7 V)}
        \conrow{开关速度}{较慢,存在\red{电荷存储效应}和反向恢复时间}
        \conrow{应用}{整流、稳压、一般逻辑电路}
    }
\end{minipage}

\subsection{欧姆接触}

由于表面态的存在,欧姆接触只是一个\red{理想化模型}。

\entry{反阻挡层}{通常Schottky接触形成耗尽层起阻挡作用,而此处形成积累层,电导率极高,不仅不阻挡电流反而比体内更利于导电,故称“反”阻挡层。}

\noindent
\begin{minipage}[t]{0.49\linewidth}
    \img{images/image-2025-12-29-23-38-28.png}
    \blue{N型 ($\phi_m < \phi_s$)}
    \concepttable{
        \conrow{形成条件}{$E_{Fm} > E_{FN}$}
        \conrow{载流子输运}{电子 $M \to S$}
        \conrow{弯曲}{能带\red{向下}弯曲}
        \conrow{表面}{积累层 ($n_s \gg n_0$)}
    }
\end{minipage}
\hfill
\begin{minipage}[t]{0.49\linewidth}
    \img{images/image-2025-12-29-23-38-59.png}
    \blue{P型 ($\phi_m > \phi_s$)}
    \concepttable{
        \conrow{形成条件}{$E_{Fm} < E_{FP}$}
        \conrow{载流子输运}{空穴 $M \to S$}
        \conrow{弯曲}{能带\red{向上}弯曲}
        \conrow{表面}{积累层 ($p_s \gg p_0$)}
    }
\end{minipage}

\entry{结论}{只要接触使半导体表面的\red{多数载流子浓度增加}(形成积累层),就能实现欧姆接触。}

\entry{施加偏压能带图}{高电势一侧能带\red{向下}弯曲,低电势一侧能带\red{向上}弯曲。}

\noindent
\begin{minipage}[t]{0.49\linewidth}
    \center
    \img[0.5\linewidth]{images/image-2025-12-30-11-53-53.png}
    \textbf{给半导体一侧施加负电压(N 沟道)}
\end{minipage}
\hfill
\begin{minipage}[t]{0.49\linewidth}
    \center
    \img[0.5\linewidth]{images/image-2025-12-30-11-56-22.png}
    \textbf{给半导体一侧施加正电压(N 沟道)}
\end{minipage}

\subsection{问答题整理}

\blue{孟庆巨版本教材:} \red{红色题干}为作业、课件出现过的题目。
\concepttable{
    \conrow{界面态对肖特基势垒高度的影响}{在大多数实用的肖特基势垒中,\red{界面态在决定 $\phi_b$ 数值中处于支配地位},势垒高度基本上与两个功函数差以及半导体中的掺杂度无关。由于表面态密度无法预知,势垒高度通常为经验值。}
    \conrow{加偏压时肖特基势垒能带图中 $q\phi_b$ 几乎不变的原因}{由于金属中电子浓度极高,空间电荷区极薄,电势连续性决定了加偏压时肖特基势垒能带图中 $q\phi_b$ 几乎不变。}
    \conrow{\red{肖特基势垒二极管与 PN 结二极管的区别}}{\textbf{肖特基势垒二极管}是\red{多子器件},\textbf{PN 结二极管}是少子器件。主要区别:\par
    (1) 无少数载流子存储,存储时间可忽略,适合高频和快速开关;\par
    (2) 多数载流子电流远高于少数载流子,饱和电流远高于同面积 PN 结二极管;\par
    (3) 对同样电流,肖特基势垒上的正向电压降远低于 PN 结,适合箝位和限幅应用;\par
    (4) 多子数目起伏小,噪声小;\par
    (5) 温度特性好。}
    \conrow{金属与重掺杂半导体接触为何可形成欧姆接触}{若半导体为重掺杂(如 $10^{19}\,\mathrm{cm}^{-3}$ 或更高),空间电荷层宽度极薄,载流子可\red{隧道穿透}而非越过势垒。两侧电子均可隧穿,正反向偏压下 $I$-$V$ 曲线基本对称,表现为非整流、低电阻的欧姆接触。}
}

\blue{第四次作业相关:}

\concepttable{
    \conrow{在理想情况下,金属和半导体之间形成非整流接触势垒的条件是什么?}{前面有}
    \conrow{画出n型欧姆接触时,零偏、正偏、反偏条件下的能带图}{这三个图前面都有}
    \conrow{根据给出的金属与半导体,画出形成金半接触后的能带图}{原则就是让金属的费米能级不变,然后让半导体的费米能级和金属对齐,画出弯曲的能带图即可。然后根据半导体类型以及载流子的流向标注是阻挡层还是反阻挡层。}
}



\section{第六章\ 结型场效应晶体管}

\subsection{基本概念}



\subsection{器件特性}

\subsection{非理想因素}

\subsection{等效电路和频率限制}
1
% --- 手动换栏命令(如果需要强制换列)---
% \columnbreak 

\end{multicols*}

\end{document}