\documentclass[10pt, a4paper, landscape]{article}

% -------------------------------------------------
% 宏包引入
% -------------------------------------------------
\usepackage[fontset=mac]{ctex}       % 中文支持
\usepackage{multicol}   % 多分栏
\usepackage{calc}
\usepackage{ifthen}
\usepackage[landscape]{geometry} % 页面设置
\usepackage{amsmath,amsthm,amsfonts,amssymb} % 数学公式
\usepackage{color,graphicx,overpic} % 颜色与图片
\usepackage{hyperref}   % 超链接
\usepackage{enumitem}   % 列表环境控制
\usepackage{titlesec}   % 标题控制
\usepackage{bm}         % 加粗数学符号
\usepackage{xcolor}
\usepackage{tikz}       % 绘图
\usetikzlibrary{decorations.pathreplacing, positioning} % 加载brace装饰库
\usetikzlibrary{calc, positioning, arrows.meta, decorations.markings}
\setCJKmainfont{PingFang SC}
\setCJKsansfont{PingFang SC}
\setCJKmonofont{PingFang SC}

% -------------------------------------------------
% 自定义颜色
% -------------------------------------------------

\definecolor{myblue}{HTML}{003153} % 蓝色字
\definecolor{myred}{HTML}{85120F}  % 红色字
\definecolor{hlblue}{HTML}{7090D3} % 蓝色高亮
\definecolor{hlred}{HTML}{D66A83}  % 红色高亮
\definecolor{hlyellow}{HTML}{E3C79F}  % 奢金高亮
\definecolor{hlgreen}{HTML}{BED49D}  % 抹茶绿高亮

% -------------------------------------------------
% 极限空间压缩设置 (核心部分)
% -------------------------------------------------

% 1. 页边距设置为极小 (0.5cm)
\geometry{top=0.5cm,left=0.5cm,right=0.5cm,bottom=0.5cm}

% 2. 去掉段落首行缩进,改为段落间略微留空(可选,这里为了紧凑设为0)
\setlength{\parindent}{0pt}
\setlength{\parskip}{0pt}

% 3. 设置正文基础字体大小为 scriptsize (约8pt),如果还觉得大,可以改为 \tiny
\renewcommand{\baselinestretch}{0.9} % 压缩行间距
\let\oldfootnotesize\footnotesize
\renewcommand{\footnotesize}{\fontsize{7pt}{8pt}\selectfont}

% 4. 压缩列表环境 (Itemize/Enumerate) 的间距
\setlist{nolistsep} 
\setlist[itemize]{leftmargin=*}
\setlist[enumerate]{leftmargin=*}

% 5. 压缩标题间距
\titleformat{\section}{\bfseries\scriptsize\color{myblue}}{}{0em}{}[\hrule] % 标题带下划线,蓝色,省空间
\titlespacing*{\section}{0pt}{2pt}{1pt} % 上方留2pt,下方留1pt
\titleformat{\subsection}
    [runin] % 不换行
    {\bfseries\scriptsize} % 粗体、scriptsize,黑色字体
    {} % 不显示编号
    {0pt} % 标题与正文间距
    {\hlyellow} % 用hlyellow高亮命令包裹标题
    [] % 标题内容后无内容
\titleformat{\subsubsection}
    [runin] % 不换行
    {\bfseries\tiny} % 粗体、scriptsize,黑色字体
    {} % 不显示编号
    {0pt} % 标题与正文间距
    {\hlgreen} % 用hlgreen高亮命令包裹标题
    [] % 标题内容后无内容
\titlespacing*{\subsection}{0pt}{1pt}{0.5em} % 上方1pt,下方0.5em(水平间距)
\titlespacing*{\subsubsection}{0pt}{1pt}{0.5em} % 上方1pt,下方0.5em(水平间距)

% -------------------------------------------------
% 自定义命令
% -------------------------------------------------
% 颜色字
\newcommand{\red}[1]{\textbf{\textcolor{myred}{#1}}}  
\newcommand{\blue}[1]{\textbf{\textcolor{myblue}{#1}}} 
\newcommand{\entry}[2]{$\bullet$ \textbf{#1}: #2\par\vspace{0.5pt}}
% 高亮
\newcommand{\cbox}[2][yellow]{\begingroup\setlength{\fboxsep}{1pt}\colorbox{#1}{\strut#2}\endgroup}
\newcommand{\hlblue}[1]{\cbox[hlblue]{#1}}
\newcommand{\hlred}[1]{\cbox[hlred]{#1}}
\newcommand{\hlyellow}[1]{\cbox[hlyellow]{#1}}
\newcommand{\hlgreen}[1]{\cbox[hlgreen]{#1}} 
% 图片插入
\newcommand{\img}[2][0.9\linewidth]{%
    {\par\vspace{1pt}\centering\includegraphics[width=#1]{#2}\par\vspace{1pt}}%
}
% 左图右文 (参数: [图片宽度比例]{图片路径}{右侧文字内容})
\newcommand{\imgleft}[3][0.3]{%
    \noindent\begin{minipage}[t]{#1\linewidth}%
        \vspace{0pt}%
        \includegraphics[width=\linewidth]{#2}%
    \end{minipage}%
    \hfill%
    \begin{minipage}[t]{0.98\linewidth - #1\linewidth}%
        \vspace{0pt}%
        #3%
    \end{minipage}\par\vspace{2pt}%
}
% 左文右图 (参数: [图片宽度比例]{图片路径}{左侧文字内容})
\newcommand{\imgright}[3][0.3]{%
    \noindent\begin{minipage}[t]{0.98\linewidth - #1\linewidth}%
        \vspace{0pt}%
        #3%
    \end{minipage}%
    \hfill%
    \begin{minipage}[t]{#1\linewidth}%
        \vspace{0pt}%
        \includegraphics[width=\linewidth]{#2}%
    \end{minipage}\par\vspace{2pt}%
}
% -------------------------------------------------
% 新增:概念速查表专用命令
% -------------------------------------------------
% 表格容器
\newcommand{\concepttable}[1]{%
    {\setlength{\tabcolsep}{1.5pt}% 局部减小列间距
     \renewcommand{\arraystretch}{0.92}% 局部紧缩行距
     \par\vspace{2pt}{\color{myblue}\hrule height 0.6pt}\vspace{1pt}% 上边框(蓝色,0.6pt粗)
     \noindent\begin{tabular}{@{}p{0.22\linewidth}p{0.76\linewidth}@{}}%
     #1%
     \end{tabular}%
     \vspace{1pt}{\color{myblue}\hrule height 0.6pt}\par\vspace{2pt}}% 下边框(蓝色,0.6pt粗)
}
% 表格行 (参数: {概念名}{解释})
\newcommand{\conrow}[2]{\blue{#1} & #2 \\}




% -------------------------------------------------
% 正文
% -------------------------------------------------
\begin{document}

\tiny

% 三栏布局
\begin{multicols*}{3}

\section{第一章\ 集成电路器件基础}

\subsection{MOS 管寄生效应}(主要是 MOS 器件电容模型)

\entry{主要耦合电容}{任意两端子间均存在电容耦合:}
    \concepttable{
        \conrow{$C_{GS}, C_{GD}$}{栅-源 / 栅-漏电容}
        \conrow{$C_{GB}$}{栅-衬底电容}
        \conrow{$C_{SB}, C_{DB}$}{源-衬底 / 漏-衬底电容}
}

\imgleft[0.4]{images/image-2025-12-30-18-17-15.png}{

    \entry{电容分解}{分为\red{本征部分}(沟道相关)与\red{寄生部分}(结构决定):}
    \concepttable{
        \conrow{栅源电容}{\red{$C_{GS} = C_{GCS} + C_{GSO}$}\par
        $C_{GCS}$ 为本征栅-沟道-源电容\par
        $C_{GSO}$ 为栅-源\red{覆盖寄生电容}。}
        \conrow{栅漏电容}{\red{$C_{GD} = C_{GCD} + C_{GDO}$}\par
        $C_{GCD}$ 为本征栅-沟道-漏电容\par
        $C_{GDO}$ 为栅-漏\red{覆盖寄生电容}。}
        \conrow{栅衬底电容}{$C_{GB} = C_{GCB}$,主要考虑本征栅-沟道-衬底电容。}
        \conrow{源漏衬底电容}{$C_{SB}, C_{DB}$,主要由源/漏扩散区与衬底形成的 PN 结电容 $C_{Sdiff}, C_{Ddiff}$ 构成。}
    }

}

\subsubsection{覆盖寄生电容}

\entry{定义}{栅极与源/漏扩散区重叠部分形成的电容,称为\red{覆盖寄生电容}。}

\imgleft[0.4]{images/image-2025-12-30-18-31-44.png}{

\entry{物理成因}{由于横向扩散效应,源/漏杂质向栅极下方扩散 $x_d$,导致栅极边缘与源/漏区在垂直方向重叠形成平板电容。}

\concepttable{
    \conrow{线性特性}{大小仅取决于几何尺寸,不随偏置电压变化,属于\red{线性电容}。}
    \conrow{密勒效应}{连接在输入(栅)与输出(漏)间的 $C_{GDO}$ 会受\red{密勒效应}影响被放大。}
    \conrow{计算公式}{$C_{GSO} = C_{GDO} = C_{ox} x_d W = C_o W$}
}

\entry{参数说明}{$C_{ox}$ 为单位面积氧化层电容,$x_d$ 为横向扩散长度,$W$ 为晶体管宽度,$C_o = C_{ox}x_d$ 为单位宽度的覆盖电容。}

}

\subsubsection{PN 结电容}

\entry{定义}{即源/漏区与衬底间的电容($C_{SB}, C_{DB}$),由反偏 PN 结的势垒电容(耗尽电容)构成。}

\imgleft[0.35]{images/image-2025-12-30-18-37-19.png}{
    \concepttable{
        \conrow{物理结构}{源/漏重掺杂区与衬底形成 PN 结。为防止漏电,常通过“沟道阻挡注入”增加侧壁掺杂浓度。}
        \conrow{底板电容}{对应扩散区底面积 $AREA = L_S \times W$\par
        \red{计算项:}$C_{bottom} = C_j \times AREA$,$C_j$ 为单位面积结电容。}
        \conrow{侧壁电容}{对应扩散区侧面周长 $PERIMETER = 2L_S + W$(不含面向沟道的一侧)。\par
        \red{计算项:}$C_{sw} = C_{jsw} \times PERIMETER$。}
        \conrow{总扩散电容}{$C_{diff} = C_j L_S W + C_{jsw}(2L_S + W)$}
    }
}

\subsubsection{栅沟道电容(本征电容)}

由于不同工作区的沟道形态不同,栅沟道电容也不同:

\imgleft[0.5]{images/image-2025-12-30-18-26-22.png}{
    {\setlength{\tabcolsep}{1pt}
    \begin{tabular}{l|ccc}
    \hline
    \blue{工作区} & \blue{$C_{GCB}$} & \blue{$C_{GCS}$} & \blue{$C_{GCD}$} \\ \hline
    \red{截止} & $C_{ox}WL$ & 0 & 0 \\
    \red{线性(电阻区)} & 0 & $\frac{1}{2}C_{ox}WL$ & $\frac{1}{2}C_{ox}WL$ \\
    \red{饱和} & 0 & $\frac{2}{3}C_{ox}WL$ & 0 \\ \hline
    \end{tabular}}
}

\subsection{阈值电压}

(这部分看半导体器件物理 cheatsheet 观感更佳)

\entry{定义}{$V_T$ 是表面载流子浓度等于衬底掺杂浓度(强反型)时的 $V_{GS}$。此时表面电势 $\phi_s$ 达到 $2\phi_F$。}

\concepttable{
    \conrow{功函数差 $\phi_{ms}$}{抵消金属与半导体费米能级不匹配,由材料本征属性决定。}
    \conrow{耗尽电荷 $Q_B$}{补偿耗尽层固定电荷。NMOS 中 $Q_B < 0$,对应项 $-\frac{Q_B}{C_{ox}}$ 为正。}
    \conrow{强反型电势 $2\phi_F$}{产生强反型所需的能带弯曲量。P 衬底 $\phi_F < 0$,项 $-2\phi_F$ 为正。}
    \conrow{表面电荷 $Q_{SS}$}{氧化层界面正电荷,有助于感应电子,从而降低 $V_T$。项 $-\frac{Q_{SS}}{C_{ox}}$ 为负。}
    \conrow{注入电荷 $Q_I$}{工艺调节项,通过离子注入(如 P 注入)精确修正 $V_T$ 数值。}
}

\entry{汇总公式}{$\boxed{V_T = \phi_{ms} - 2\phi_F - \frac{Q_B}{C_{ox}} - \frac{Q_{SS}}{C_{ox}} - \frac{Q_I}{C_{ox}}}$}

\entry{物理本质}{$V_T$ 是克服材料差异、抵消干扰电荷、平衡耗尽层并建立强反型表面电势所需的栅极电压总和。}

\subsection{速度饱和}

这个暂时没在 PPT 上找到,先空着吧。

\subsection{互补网络}

\entry{架构组成}{由两个相互关联的网络组成:}

\imgleft[0.25]{images/image-2025-12-30-18-44-30.png}{
    
    \concepttable{
        \conrow{上拉网络 (PUN)}{连接 $V_{DD}$ 与输出 $F$。满足条件 $G$ 时导通,将输出上拉至 \red{Logic 1}。}
        \conrow{下拉网络 (PDN)}{连接 $GND$ 与输出 $F$。满足条件 $\bar{G}$ 时导通,将输出下拉至 \red{Logic 0}。}
    }
    \entry{布尔表达式}{工作原理遵循 $F = 1 \cdot G + 0 \cdot \bar{G} = G$。}
    \entry{互补性}{PUN 与 PDN 必须\red{互斥}(稳态下不能同时导通或截止),以确保输出电平确定并消除静态功耗。}
}

\subsection{习题解析}

\subsubsection{课堂练习}

主要就是一些判断逻辑以及跟器件强相关的题目,和模电很像。

\imgleft[0.2]{images/image-2025-12-30-18-50-54.png}{
    \entry{例题 1}{考虑图中的静态互补CMOS逻辑门,写出它的布尔表达式(注意化简)。}
    \entry{启发意义}{没啥启发意义。}
}

\entry{解答}{这道题看下拉网络比较好判断逻辑,答案是 $F = \overline{A B + C D}$。}

\imgleft[0.3]{images/image-2025-12-30-18-54-44.png}{
    \entry{例题 2}{计算两个宽长比分别为$W1/L$和$W2/L$的串联NMOS晶体管的等效宽长比
$W/L$。忽略体效应,阈值电压恒定。}
    \entry{启发意义}{单看题目只是一个模集的题,但是为后面计算串联尺寸系数打下基础。}
}

\entry{解答}{
        \entry{推导前提}{利用线性区电流公式 $I_{DS} = k' \frac{W}{L} [(V_{GS} - V_{T0})V_{DS} - \frac{1}{2}V_{DS}^2]$,基于电流连续性($I_{DS1} = I_{DS2}$)进行推导。}

        \concepttable{
            \conrow{M2 (下管)}{$I_{DS} = k' \frac{W_2}{L} [ (V_{GS} - V_{T0})V_{DS2} - \frac{1}{2}V_{DS2}^2 ]$}
            \conrow{M1 (上管)}{$I_{DS} = k' \frac{W_1}{L} [ (V_{GS} - V_{DS2} - V_{T0})(V_{DS} - V_{DS2}) - \frac{1}{2}(V_{DS} - V_{DS2})^2 ]$}
        }

        \entry{代数变换}{展开 M1 方程并提取项,可发现其包含 M2 的电压项。整理得:$I_{DS} = k' \frac{W_1}{L} [ ((V_{GS}-V_{T0})V_{DS} - \frac{1}{2}V_{DS}^2) - \frac{I_{DS}}{k' (W_2/L)} ]$。}

        \entry{等效结果}{将 $I_{DS}$ 项移项合并,对比标准方程可得:$\boxed{\frac{1}{W_{eq}} = \frac{1}{W_1} + \frac{1}{W_2}}$ 或 $\boxed{W_{eq} = \frac{W_1 W_2}{W_1 + W_2}}$。}

        \entry{物理意义与设计指导}{
            \concepttable{
                \conrow{电阻类比}{导通电阻 $R_{on} \propto L/W$。串联电阻 $R_{eq} = R_1 + R_2$ 对应 $\frac{L}{W_{eq}} = \frac{L}{W_1} + \frac{L}{W_2}$。}
                \conrow{尺寸补偿}{串联会降低驱动能力。若要使两个串联管等效于宽度为 $W$ 的单管,则每个管子宽度需设为 \red{$2W$}。}
            }
        }
}

\subsubsection{作业题}

有一些奇奇怪怪的题,不知道牢王都从哪找的。

\imgleft[0.2]{images/image-2025-12-30-19-03-08.png}{
    \entry{作业 1-1}{
        \begin{enumerate}[label=(\arabic*)]
            \item 考虑下图的静态互补 CMOS 逻辑门,写出布尔表达式(注意化简),并画出下拉网络结构;
            \item 画出实现 $Y = AB + C(D + E)$ 的静态互补 CMOS 逻辑门电路的晶体管级电路图。
        \end{enumerate}
    }
    \entry{启发意义}{这道题的上拉电路给的很怪,没有办法直接看出逻辑表达式,需要通过分析通路来推导。}
}

\entry{第一步}{寻找从 $V_{DD}$ 到 Output 所有可能通路(PMOS 导通需低电平):}
\concepttable{
    \conrow{直接路径}{路径 1:$\overline{A}\overline{C}$;路径 2:$\overline{B}\overline{D}$}
    \conrow{跨桥路径}{路径 3:$\overline{A}\overline{E}\overline{D}$;路径 4:$\overline{B}\overline{E}\overline{C}$}
    \conrow{PUN 逻辑}{$F = \overline{A}\overline{C} + \overline{B}\overline{D} + \overline{A}\overline{E}\overline{D} + \overline{B}\overline{E}\overline{C}$}
}

\entry{第二步}{推导下拉网络 (PDN) 的导通条件:}
\concepttable{
    \conrow{设计原理}{CMOS 逻辑具有反相特性。PDN 由 NMOS 构成(高电平导通),需满足输出为低电平 (\red{Logic 0}) 的条件,即求 $\overline{Out}$。}
    \conrow{德·摩根变换}{对 PUN 表达式整体取反:$\overline{Out} = \overline{(\bar{A}\bar{C} + \bar{A}\bar{E}\bar{D} + \bar{B}\bar{D} + \bar{B}\bar{E}\bar{C})}$}
    \conrow{逻辑展开}{根据 $\overline{X+Y} = \bar{X} \cdot \bar{Y}$,得 $\overline{Out} = (\overline{\bar{A}\bar{C}}) \cdot (\overline{\bar{A}\bar{E}\bar{D}}) \cdot (\overline{\bar{B}\bar{D}}) \cdot (\overline{\bar{B}\bar{E}\bar{C}})$}
    \conrow{去反号}{根据 $\overline{XY} = \bar{X} + \bar{Y}$,得 $\overline{Out} = (A+C)(A+E+D)(B+D)(B+E+C)$}
}

\imgleft[0.3]{images/image-2025-12-30-19-16-37.png}{
    \entry{代数化简}{通过提取公因式进行重组:}
    \concepttable{
        \conrow{分组观察}{前两项含 $A$,后两项含 $B$。利用分配律:$(A+C)(A+E+D) = A + C(E+D)$}
        \conrow{最终嵌套形式}{$\overline{Out} = AB+AED+CD+BEC$}
        \conrow{电路对应}{该表达式直接决定了 PDN 的串并联拓扑结构。}
    }
}

\entry{作业 1-2}{请分别解释说明体效应、短沟效应、 DIBL对阈值电压影响及原理。}

\entry{启发意义}{一些牢王自己忘记讲的概念}

\entry{解析}{

\concepttable{
    \conrow{体效应}{当 $V_{SB} > 0$ 时,更多负电荷聚集在栅氧化层下,增加了耗尽层电荷,导致 \red{$V_T$ 增加}。公式:$V_T = V_{T0} + \gamma (\sqrt{V_{SB} + |2\phi_F|} - \sqrt{|2\phi_F|})$。源端势垒上升,\red{需更大栅压克服势垒}。}
    \conrow{短沟道效应}{当 $L$ 减小时,\red{$V_T$ 随之减小}。部分栅下区域空穴被漏-衬底 PN 结电场耗尽,导致 $Q_B$ 下降;同时 MOS 效应影响区域比例变小,导致 $V_{T0}$ 下降。}
    \conrow{DIBL 效应}{漏端感应势垒降低。$V_{DS}$ 增加使漏端耗尽区扩大并接近源端,引起源端势垒降低,使源区注入电子增加,导致 \red{$V_T$ 下降}。}
}

}

\entry{作业 1-3}{请分别解释说明速度饱和对短沟器件和长沟器件的影响及原理。}

\entry{启发意义}{算是弥补上文缺失的速度补偿部分,牢王应该是又忘讲了。}

\entry{解析}{
\entry{载流子速度}{速度 $v$ 与电场 $E$ 的关系近似为:$v = \frac{\mu_n E}{1 + E/E_c}$。}
\concepttable{
    \conrow{临界电场 $E_c$}{速度饱和发生时的电场。连续性要求:当 $E=E_c$ 时,$v_{sat} = \mu_n E_c / 2$。}
    \conrow{短沟道修正}{由于 $L$ 极小,水平电场大,很快达到饱和。修正公式:$I_D = k' [(V_{GS}-V_T)V_{DS} - \frac{1}{2}V_{DS}^2] \kappa(V_{DS})$,其中 $\kappa(V) = \frac{1}{1 + V/(E_c L)} < 1$。}
}

\entry{饱和电压 $V_{DSAT}$}{由连续性要求解得:$V_{DSAT} = \kappa(V_{GT})V_{GT}$。}
\concepttable{
    \conrow{物理结论}{因为 $\kappa(V_{GT}) < 1$,所以 \red{$V_{DSAT} < V_{GT}$}。短沟道器件速度饱和的 $V_{DSAT}$ 小于长沟道沟道夹断的 $V_{DSAT}$。}
    \conrow{长沟道特性}{当 $E_c \gg V_{GT}/L$ 时,$I_{DSAT} = \frac{1}{2} \frac{W}{L} C_{ox} \mu_n V_{GT}^2$,电流电压为\red{二次关系}。}
    \conrow{短沟道特性}{当 $E_c \ll V_{GT}/L$ 时,$I_{DSAT} = v_{sat} C_{ox} W V_{GT}$,电流电压为\red{一次关系}。}
}

}

\imgleft[0.2]{images/image-2025-12-30-19-34-42.png}{
    \entry{作业 1-4}{如图所示电路,已知 M1 参数如下(忽略体效应和沟道长度调制效应):

    \concepttable{
        \conrow{电压参数}{$V_{T0} = 0.43V, V_{DSAT} = 0.63V$}
        \conrow{工艺常数}{$k' = 115 \times 10^{-6} A/V^2, C_{ox} = 6 fF/\mu m^2$}
        \conrow{覆盖电容}{$C_{gso} = C_{gdo} = 0.31 fF/\mu m$}
    }

    \begin{enumerate}[label=(\arabic*)]
        \item 当 $V_{in} = 2.5V$ 时,$V_{out}$ 的稳态电压(记为 $V_{OL}$)是多少?M1 管处在什么工作区?
        \item 如果 $V_{in}$ 从 $0V$ 上升到 $2.5V$,而 $V_{out}$ 的初始电压等于 $2.5V$,那么从 $In$ 端变化开始到 $V_{out}$ 达到稳态这个过程中由 $In$ 端注入的总净电荷量等于多少?
    \end{enumerate}

    }
}

\entry{解答 (1)}{采用假设-验证法确定 $V_{OL}$ 与工作区:}
\concepttable{
    \conrow{假设}{M1 工作在\red{线性区},即满足 $V_{OL} < V_{DSAT} = 0.63V$。}
    \conrow{方程组}{1. 线性区电流公式:$I_{ds} = k' \frac{W}{L} [(V_{in} - V_{T0})V_{OL} - \frac{1}{2}V_{OL}^2]$ \par 
    2. 根据分压定律:$I_{ds} \cdot R + V_{OL} = V_{DD}$ (KVL)}
    \conrow{参数代入}{$V_{DSAT} = V_{in}-V_{T0} = 2.07V$,$k'W/L = 460 \mu A/V^2$,$R = 8k\Omega$。}
    \conrow{计算结果}{解得 $V_{OL} \approx 0.31V$ (另一解 $4.37V$ 舍去)。}
    \conrow{验证}{因 $0.31V < 0.63V$,假设成立,M1 确实处于\red{线性区}。}
}

\entry{\red{解答 (2)}}{计算注入总净电荷 $\Delta Q = Q_{final} - Q_{initial}$:}
\concepttable{
    \conrow{初始态 $Q_1$}{$V_{in}=0V, V_{out}=2.5V$ (\red{截止区})。\par
    沟道未形成,$C_{GCB}$ 无压差。$C_{GCS} = 0$,且 GS 之间也没有压差,$C_{GCD} = 0$,但是 GD 之间存在电压  \par 
    $Q_1 = C_{GD} \times (V_G - V_D) = C_{gd0} \times W \times (V_G - V_D) = (0.31 \times 1) \times (0 - 2.5) = -0.775 fC$(这里完全是 PN 结扩散电容)}
    \conrow{稳态 $Q_2$}{$V_{in}=2.5V, V_{out}=0.31V$ (\red{线性区})。沟道形成。 \par 
    后面的计算留给读者自己完成,\red{步骤就是:查表看电容 $\rightarrow$ 看 GS、GD、GB 之间是否有电压$\rightarrow$有电压有电容的地方计算电荷 $\rightarrow$ 累加。}}
    \conrow{总注入量}{$\Delta Q = 4.97 - (-0.775) = 5.745 fC$。}
}

\section{第二章\ 数字集成电路的速度}

\subsection{MOS 管电容模型}

\subsection{MOS 管的漏电流}

\subsection{逻辑门的静态特性}

\subsection{开关阈值(对称反相器)}

\subsection{CMOS 逻辑门的延时特性}

\subsection{本征延时、努力延时}

\subsection{逻辑努力(含义、计算)}

\subsection{关键路径的计算}

\subsection{逻辑路径的延时模型}

\subsection{尺寸优化问题}

\subsection{分支努力}

\section{第三章\ 数字集成电路的功耗}

\subsection{动态功耗}

CMOS功耗主要分为两大类:

\concepttable{
    \conrow{动态功耗}{电路逻辑状态\red{翻转过程中产生的功耗},分为充放电功耗和短路功耗。}
    \conrow{静态功耗}{\red{电路处于稳态时},由漏电流引起的功耗。}
}

动态功耗又可细分为:

\concepttable{
    \conrow{充放电功耗}{对负载电容进行充电和放电所消耗的能量。}
    \conrow{短路功耗}{输入翻转瞬间 PMOS 和 NMOS 同时导通,形成 $V_{dd}$ 到 GND 的瞬时通路。}
}

\subsubsection{充放电功耗}

CMOS 逻辑门的输出端可以等效为一个一阶 RC 电路。$V_{in}$ 代表逻辑切换,向负载电容 $C_L$ 充电。

\imgleft[0.4]{images/image-2025-12-29-16-42-55.png}{

    \concepttable{
        \conrow{总能耗 $E_{0 \to 1}$}{输出从 0 翻转到 1 时,电源 $V_{dd}$ 提供的总能量为 $C_L V_{dd}^2$。}
        \conrow{存储能量 $E_{cap}$}{电容 $C_L$ 最终存储的电能为 $\frac{1}{2} C_L V_{dd}^2$。}
        \conrow{能量损耗 $E_{res}$}{充电路径电阻上以热能形式耗散 $\frac{1}{2} C_L V_{dd}^2$。}
    }

    \red{结论}:在一个完整的翻转周期 ($0 \to 1 \to 0$) 中,\red{电源消耗}的总能量为 $C_L V_{dd}^2$。
}

\concepttable{
        \conrow{充电 ($0 \to 1$)}{PMOS 导通,电源提供能量。一半存于电容,一半消耗在 PMOS 网络电阻上。}
        \conrow{放电 ($1 \to 0$)}{NMOS 导通,电源不供能。电容存储能量全部在 NMOS 网络电阻上消耗。}
}

\subsubsection{短路功耗}

\entry{定义}{输入翻转瞬间,当 $V_{Tn} < V_{in} < V_{DD} - |V_{Tp}|$ 时,PMOS 和 NMOS 同时导通,形成 $V_{dd}$ 到 GND 的瞬时通路。}

\imgleft[0.5]{images/image-2025-12-29-19-16-01.png}{
    \entry{负载电容 $C_L$ 的影响}{
    \concepttable{
        \conrow{负载较大}{输出 $V_{out}$ 变化慢,管子漏源压差小,短路电流 \red{$I_{sc} \approx 0$}。}
        \conrow{\red{负载较小}}{输出 $V_{out}$ 变化快,管子进入饱和区,短路电流达到最大值 \red{$I_{MAX}$}。}
        }
    }

    \entry{定量计算}{
        \concepttable{
            \conrow{积分公式}{$V_{DD} \cdot \int_{t(V_T)}^{t(V_{DD}-V_T)} I(t)dt \cdot f$}
            \conrow{近似公式}{$P_{sc} = V_{DD} I_{peak} t_{sc} f$,其中 $I_{peak}$ 为峰值,$t_{sc}$ 为等效持续时间。}
        }
    }
}

\entry{核心影响因素}{
        \concepttable{
            \conrow{翻转速率}{与输入/输出 \red{翻转速率} 有关。输入斜率越慢,导通时间 $t_{sc}$ 越长。设计中通常控制短路功耗在总动态功耗的 10\%-15\%。}
            \conrow{器件尺寸}{$I_{peak}$ 与 \red{$W/L$} 成正比。\red{增加逻辑门驱动强度(增大尺寸)会直接导致短路功耗上升。}}
}}

\subsection{逻辑门的能耗}

能耗与功耗:

\noindent\begin{minipage}[t]{0.48\linewidth}
    \concepttable{
        \conrow{能耗}{从电源取得的总能量}
        \conrow{全摆幅}{$E_{0 \to 1} = C_L V_{dd}^2$}
        \conrow{周期}{一个翻转周期 $0\to1\to0$}
    }
\end{minipage}
\hfill
\begin{minipage}[t]{0.48\linewidth}
    \concepttable{
        \conrow{功耗}{单位时间内的能量消耗}
        \conrow{公式}{$P = \frac{E}{\Delta T} = C_L V_{dd}^2 f_{0 \to 1}$}
        \conrow{频率}{$f_{0 \to 1}$ 为开关频率}
    }
\end{minipage}


\subsection{开关活动性}

动态功耗的核心统计学指标:

\concepttable{
    \conrow{定义}{开关活动性 $\alpha_{0 \to 1}$(翻转概率)指逻辑门在一个时钟周期内发生 $0 \to 1$ \red{翻转的概率}。}
    \conrow{计算公式}{$\alpha_{0 \to 1} = p_0 \cdot p_1$,其中 $p_0, p_1$ 分别为输出为 0 和 1 的概率。}
    \conrow{静态门分布}{对于 $N$ 输入逻辑门,$\alpha_{0 \to 1} = \frac{N_0}{2^N} \cdot \frac{N_1}{2^N}$,其中 $N_0, N_1$ 为输出 0 和 1 的状态数。}
}

\entry{等效电容}{$\boxed{C_{EFF} = \alpha_{0 \to 1} C_L}$,则动态功耗 $\boxed{P = C_{EFF} V_{dd}^2 f_{0 \to 1}}$。}

\entry{实例计算}{计算 $\alpha_{0 \to 1}$(假设输入信号为 1 的概率均为 $1/2$):}

\noindent\begin{minipage}[t]{0.48\linewidth}
    \entry{示例 1-2输入 NOR}{\concepttable{
        \conrow{真值表}{仅 $A=B=0$ 时输出为 1}
        \conrow{概率}{$p_1 = 1/4, p_0 = 3/4$}
        \conrow{翻转率}{$\alpha_{0 \to 1} = 3/4 \times 1/4 = 3/16$}
    }}
\end{minipage}
\hfill
\begin{minipage}[t]{0.48\linewidth}
    \entry{示例 2-2输入 XOR}{\concepttable{
        \conrow{真值表}{$A \neq B$ 时输出为 1}
        \conrow{概率}{$p_1 = 1/2, p_0 = 1/2$}
        \conrow{翻转率}{$\alpha_{0 \to 1} = 1/2 \times 1/2 = 1/4$}
    }}
\end{minipage}

\entry{常用逻辑门翻转概率}{
    \concepttable{
        \conrow{AND 门}{输出为 1 的概率 $p_1 = p_A p_B$,则 \red{$\alpha_{0 \to 1} = (1 - p_A p_B) p_A p_B$}}
        \conrow{OR 门}{输出为 1 的概率 $p_1 = 1 - (1-p_A)(1-p_B)$,则 \red{$\alpha_{0 \to 1} = (1-p_1)p_1$}}
        \conrow{XOR 门}{输出为 1 的概率 $p_1 = p_A(1-p_B) + p_B(1-p_A)$,则 \red{$\alpha_{0 \to 1} = (1-p_1)p_1$}}
    }
}

\subsection{信号间的相关性}

当电路中存在\red{重聚扇出}结构时,逻辑门输入端信号不再相互独立。

\imgleft[0.4]{images/image-2025-12-29-17-05-17.png}{
    \concepttable{
        \conrow{无重聚扇出}{信号 $B, C$ 独立,$P(B=1, C=1) = P(B=1)P(C=1)$,概率可直接相乘。}
        \conrow{有重聚扇出}{信号 $B, C$ 源自同一信号 $A$,受逻辑约束(如 $P(B=1, C=1) = 0$)。}
    }
    \entry{结论}{简单概率乘法不再适用,需利用\red{条件概率}建模或使用 CAD 工具进行仿真分析。}
}


\subsection{影响静态功耗的因素}

\entry{定义}{电路在稳态(无信号翻转)下,由\red{静态漏电流} 引起的功耗。}

\imgleft[0.15]{images/image-2025-12-29-19-29-40.png}{
    \concepttable{
        \conrow{亚阈值漏电 $I_{SUB}$}{\red{占比最大}。即使 $V_{GS} < V_T$,源漏间仍存在由载流子扩散运动形成的微弱电流。}
        \conrow{栅极漏电 $I_{GS}, I_{GD}$}{栅氧化层极薄,载流子通过\red{量子隧穿效应}穿过绝缘层形成的电流。}
        \conrow{结泄漏电流 $I_{LEAK}$}{源/漏扩散区与衬底间 PN 结反偏电流,包含\red{带间隧穿 (BTBT) }成分。}
    }
    \entry{核心影响因素}{\red{阈值电压 $V_T$}(指数级影响,后面的效应也都是影响这个因素)、\red{温度}、\red{电源电压 $V_{DD}$}。}
}

\subsubsection{体效应}

\entry{定义}{又称\red{衬偏调制效应}。当源极电位高于衬底电位($V_{SB}$ 上升)时,耗尽层电荷增加,导致开启晶体管所需的阈值电压 $V_T$ 上升。}

\concepttable{
    \conrow{计算公式}{$V_T = V_{T0} + \gamma (\sqrt{V_{SB} + |2\phi_F|} - \sqrt{|2\phi_F|})$}
    \conrow{参数说明}{$\gamma$ 为衬偏效应系数,与氧化层电容 $C_{ox}$ 和衬底掺杂浓度 $N_A$ 有关。}
    \conrow{结论}{在堆叠结构中,由于 $V_x > 0$ 导致 $V_{SB} > 0$,使 \red{$V_T$ 增大},从而有效减小漏电流。}
}

\subsubsection{DIBL 效应}

\entry{定义}{\red{漏端感应势垒降低效应}。当 $V_{DS}$ 较高时,漏端电场渗透至源端降低势垒高度,导致电子易注入沟道,从而降低阈值电压。}

\concepttable{
    \conrow{公式}{$V_{T0}' = V_{T0} - \eta V_{DS}$。即 $V_{DS}$ 越大,有效 $V_T$ 越低,漏电流越大。}
    \conrow{堆叠关联}{堆叠结构中上方管 $V_{DS}$ 减小($V_{DD} \to V_{DD}-V_x$),使 $V_T$ 保持较高,抑制漏电。}
    \conrow{严重后果}{若 $V_{DS}$ 过大可能发生\red{源漏穿通 (Punch-through)},电流不再受栅压控制。}
}

\subsubsection{高阈值器件位置}

\red{阈值电压高的晶体管放在外层(远离输出端)},其它堆叠管的源端电位更高。

\subsubsection{堆叠效应(降低静态功耗的手段)}

\entry{定义}{当两个或多个晶体管串联(堆叠)且同时截止时,其总漏电流显著小于单管截止时的漏电流。}

\entry{物理模型}{
    在堆叠结构中,中间节点 $x$ 会由微小漏电流充电达到稳态电压 $V_x > 0$。
    \red{亚阈值电流公式:}
    \vspace{-0.3cm}
    $$I = I_{ds0} e^{\frac{V_{gs}-V_{th}}{nkT/q}}(1 - e^{\frac{-V_{ds}}{kT/q}})$$
}

\vspace{-0.3cm}

\imgleft[0.2]{images/image-2025-12-29-19-35-10.png}{

    \entry{核心物理机制}{
        \concepttable{
            \conrow{负 $V_{GS}$}{对于上管 M1,$V_{GS1} = 0 - V_x = -V_x$。由于亚阈值电流随 $V_{GS}$ 呈指数下降,负的栅源电压极大地抑制了漏电流。}
            \conrow{体效应}{源极电位 $V_x$ 升高导致 $V_{SB} > 0$。根据体效应,M1 的阈值电压 $V_T$ 升高,进一步减小漏电流。}
            \conrow{DIBL 减弱}{漏源电压 $V_{DS1} = V_{DD} - V_x$ 减小,削弱了漏致势垒降低效应(DIBL),使 $V_T$ 相对保持在较高水平。}
        }
    }

}

\subsection{习题解析}

\subsubsection{课堂练习}

\entry{例题 1}{当逻辑门电容以外部负载电容为主时,尺寸放大 2 倍以减小延时,其平均功耗变为 \red{2} 倍,一次翻转能耗变为 \red{1} 倍。}

\concepttable{
    \conrow{能耗分析}{一次翻转能耗 $E = \frac{1}{2} C_{total} V_{DD}^2$。因 $C_{load}$ 占据主导且保持不变,$C_{total} \approx C_{load}$,故\red{单次翻转做功总量不变}。}
    \conrow{功耗分析}{平均功耗 $P \propto I$。尺寸放大 2 倍使驱动电流 $I$ 变为 2 倍,单位时间内从电源汲取的能量(功率)随之变为 \red{2} 倍。}
    \conrow{延时分析}{传播延时 $t_p \propto \frac{C_{total} V_{DD}}{I}$。由于 $C_{total}$ 近似不变而 $I$ 翻倍,延时 $t_p$ 缩小为原来的 1/2。}
    \conrow{PDP 验证}{$PDP = P_{avg} \times t_p = (2P) \times (0.5t_p) = E$。功耗延时积(即能耗)在尺寸放大后保持一致。}
}
\entry{例题 2}{已知 $C_{ext}/C_{g1}=4$,调节 $V_{DD}$ 使总延时 $D \le D(f=1, V_{DD\_nom})$,求 $f=1.2$ 与 $f=1.4$ 谁的最优能耗更低?}

\entry{延时计算分析}{
    \concepttable{
        \conrow{延时模型}{单级延时 $d = p + g \cdot h$,反相器 $p=1, g=1$}
        \conrow{第一级}{负载 $f C_{g1}$,输入 $C_{g1} \implies h_1 = f, d_1 = 1 + f$}
        \conrow{第二级}{负载 $4 C_{g1}$,输入 $f C_{g1} \implies h_2 = 4/f, d_2 = 1 + 4/f$}
    }
}

\concepttable{
        \conrow{延时模型}{总延时 $D = d_1 + d_2 = (1+f) + (1+4/f) = 2+f+4/f$ (单位 $t_p$)}
        \conrow{基准约束}{$f=1$ 时,$D_0 = 2+1+4 = 7t_p$。此为设计必须满足的延时上限}
        \conrow{延时计算}{$D(1.2) = 2+1.2+3.33 = 6.53t_p$;$D(1.4) = 2+1.4+2.86 = 6.26t_p$}
}

\entry{电压缩放原理}{延时 $t_p \propto \frac{V_{DD}}{(V_{DD}-V_t)^\alpha}$,降低 $V_{DD}$ 会使电路变慢;能耗 $E \propto C V_{DD}^2$,降低 $V_{DD}$ 显著降低能耗。}

\entry{优化策略}{若电路速度快于设计要求,可\red{通过降低 $V_{DD}$ 牺牲速度冗余来换取能耗下降}。}


\entry{分析}{
    由于 $D(1.4) = 6.26$ 且 $D(1.2) = 6.53$,两者均小于基准 $D_0 = 7$。
    $f=1.4$ 方案的速度冗余更大,意味着其电源电压 $V_{DD}$ 具有更大的下降空间。
}

\entry{结论}{\red{$f=1.4$ 的最优能耗更低}。根据 $E \propto V_{DD}^2$,允许电压降幅越大,最终能耗越小。}

\subsubsection{作业题}




\section{第四章\ 数字集成电路的鲁棒性}

\subsection{信号完整性}

\entry{定义}{信号在传播过程中保持原始形状的能力。本质是研究\red{数字信号的模拟特性}(即电压或电流随时间变化的波形物理退化)。}

\concepttable{
    \conrow{幅度}{电压\red{电平是否足够高}(逻辑 $1$)或足够低(逻辑 $0$),防止因衰减导致识别错误。}
    \conrow{时序}{信号边缘跳变是否在预定窗口内到达,防止延迟或抖动导致时序违例。}
}

\entry{判定标准}{信号必须以\red{要求的时序}和\red{要求的电压幅度}到达接收端。}

\entry{工程目标}{研究电路在存在噪声的情况下如何保持正确的功能。}

\subsubsection{噪声模型}

\entry{数学模型}{$\overline{V_{no}^2} = f(\overline{V_{ni}^2}) + g(\overline{V_{ngate}^2})$。}

\imgleft[0.3]{images/image-2025-12-29-20-58-11.png}{
    \concepttable{
        \conrow{输入噪声}{前一级输出噪声成为当前级输入。}
        \conrow{电路噪声}{包含电源噪声和耦合噪声(如串扰等)。}
        \conrow{输出噪声}{上述所有噪声在输出端的总和。}
    }
    \entry{信号再生}{为防止噪声在多级传输中累积淹没信号,数字电路必须具备将受扰动的信号恢复至标准电平的能力。}
}

\subsubsection{信号再生}

 \entry{信号再生性}{当第一级输入偏离额定电平时,后面各级仍能恢复其正确值。}

\imgleft[0.4]{images/image-2025-12-29-21-02-55.png}{
\entry{反相器链}{
    观察 $V_0 \to V_1 \to V_2 \dots$ 的瞬态响应:
    \concepttable{
        \conrow{输入 $V_0$}{波形质量差,边缘缓慢且电平有偏差。}
        \conrow{中间 $V_1$}{波形反相,但仍存在畸变。}
        \conrow{恢复 $V_2$}{波形变得陡峭,电平接近理想值。}
    }
}

\entry{再生性 (Regenerative)}{
    \concepttable{
        \conrow{定义}{当第一级输入偏离额定电平时,后面各级仍能恢复其正确值。}
        \conrow{本质}{具备\red{去噪}与\red{阈值判决}能力,将输出拉回标准 $V_{DD}$ 或 $GND$。}
    }
}
}

\subsection{噪声容限}

\subsubsection{开关阈值与单级噪声容限}

\entry{物理意义}{若噪声使输入进入高增益过渡区,输出将产生逻辑错误。$V_{IH}$ 需高于 $V_M$ 一定程度,$V_{IL}$ 需低于 $V_M$ 一定程度。}

\imgleft[0.3]{images/image-2025-12-29-21-07-35.png}{
\entry{VTC 曲线与再生性}{
    \concepttable{
        \conrow{再生性}{过渡区陡峭($|Gain| > 1$),输入偏差经 $f(v)$ 映射后收敛至稳态(Rail-to-Rail)。}
        \conrow{不可再生}{过渡区平缓,输入偏差无法修正,甚至因多次迭代发散或停留在中间态。}
    }
}
\entry{关键概念}{
    \concepttable{
        \conrow{开关阈值 $V_M$}{$V_{in} = V_{out}$ 的交点。逻辑判断的中间分界点,电路处于亚稳态。}
        \conrow{斜率 $= -1$}{噪声容限的边界判定标准。界定逻辑稳态区(增益 $< 1$)与过渡区。}
        \conrow{单级噪声容限}{输入最大偏差不能超过 $(V_{OH} - V_M)$ 和 $(V_M - V_{OL})$。}
    }
}
}

\section{第五章\ 互联线与互联技术}

\subsection{互联线模型}

互连线的电气特性建模,重点在于从物理结构到电路模型的抽象,以及对\red{寄生电容}特别是\red{对地电容}的精确计算。

\imgleft[0.3]{images/image-2025-12-31-11-05-22.png}{

\entry{分布参数模型}{基于物理结构抽象出的电路模型。}

\concepttable{
    \conrow{寄生电阻 $R$}{代表\red{导线自身电阻}。由于有限电导率和横截面积,电流流过时产生压降和功耗。}
    \conrow{寄生电感 $L$}{代表\red{导线自感}。\red{高频下}变化的电流产生磁场并感应出电动势,影响信号完整性(低频常忽略)。}
    \conrow{寄生电容 $C$}{分为两类:\par
    1. \red{对地/衬底电容}:导线与底部半导体衬底 (GND) 间形成的电容。\par
    2. \red{线间电容} :相邻导线间因电位差形成的耦合电容 (Coupling)。}
}

}

\subsection{互联线的寄生效应}

下图展示了互连线在物理上最完整的电气模型,称为\red{传输线模型}。

\img[0.8\linewidth]{images/image-2025-12-31-11-55-37.png}

\subsubsection{寄生电容}

寄生电容主要包括对地电容和线间电容:

\noindent\begin{minipage}[t]{0.48\linewidth}
    \img[0.9\linewidth]{images/image-2025-12-31-11-22-18.png}
    \entry{对地电容 }{单根导线与下方衬底(交流地)间的电容。}
    \concepttable{
        \conrow{平行板模型}{假设仅存在垂直均匀电场。\par
        $C_{pp} = (\varepsilon_{di} / t_{di}) WL$\par
        其中 $L, W$ 为长宽,$t_{di}$ 为绝缘层厚度。}
        \conrow{边缘电容}{考虑导线厚度 $H$,侧壁发出的边缘场形成的附加电容 $C_{fringe}$。}
        \conrow{总对地电容}{$C_{wire} = C_{pp} + C_{fringe}$。}
    }
    \entry{几何比例影响}{大 $W/H$ 时 $C_{pp}$ 主导;小 $W/H$(窄高导线)时边缘场效应显著,$C_{fringe}$ 占主要部分。}
\end{minipage}
\hfill
\begin{minipage}[t]{0.48\linewidth}
    \img[0.5\linewidth]{images/image-2025-12-31-11-25-02.png}
    \entry{线间电容}{密集布线环境下相邻导线间的相互作用。}
    \concepttable{
        \conrow{耦合电容}{\red{随导线间距减小而显著增加},存在于左右相邻或上下相邻的导线之间。}
        \conrow{多导体系统}{由平板电容、边缘电容及线间耦合电容共同构成的复杂电容网络。}
        \conrow{物理意义}{导线寄生电容不仅取决于对衬底电容,更\red{受周围邻近导线电位的相互影响}。}
        \conrow{模型演变}{在密集布线中,单纯的平行板模型因忽略了复杂的侧向耦合和边缘效应而失效。}
    }
\end{minipage}

\entry{“三明治”结构模型}{关注 Layer $n$ 的中间导线(受害线)及其周围物理环境:}

\concepttable{
        \conrow{垂直环境}{Layer $n+1$(上层金属板/布线层)与 Layer $n-1$(下层金属板/衬底)。}
        \conrow{水平环境}{Layer $n$ 同层左右两侧的相邻导线。}
}

\imgleft[0.3]{images/image-2025-12-31-11-12-43.png}{

    \entry{电容分量分解}{受害导线(Victim Wire)的寄生电容可分解为:}
    \concepttable{
        \conrow{底板电容 $C_{bot}$}{导线底面与下层间的平行板电容,主要由介质厚度 $t_1$ 决定。}
        \conrow{顶板电容 $C_{top}$}{导线顶面与上层间的平行板电容,主要由介质厚度 $t_2$ 决定。}
        \conrow{侧壁电容 $C_{adj}$}{与同层左右相邻导线间的耦合电容,由间距 $S$ 和导线厚度 $h$ 决定。}
    }

}

\entry{总对地电容计算}{前提:若周围导线均连接至固定电位(即视为交流地)。}
\concepttable{
        \conrow{计算公式}{$\boxed{C_{total} = C_{top} + C_{bot} + 2 \times C_{adj}}$}
        \conrow{设计意义}{定义了该节点驱动的总容性负载,对计算互连线延时和功耗至关重要。}
}

\subsubsection{寄生电阻}

\entry{物理电阻公式}{基于电阻定律:$R = \frac{\rho L}{A} = \frac{\rho}{H} \frac{L}{W}$。其中 $\rho$ 为电阻率,$H$ 为导线厚度,$W$ 为宽度,$L$ 为长度。}

\entry{\red{方块电阻}}{在集成电路中,厚度 $H$ 由工艺固定。定义 \red{$R_{\square} = \rho / H$}(单位:$\Omega/\square$)。}

\imgleft[0.3]{images/image-2025-12-31-11-38-34.png}{
    \concepttable{
        \conrow{计算公式}{$\boxed{R = R_{\square} \frac{L}{W}}$,其中 $L/W$ 代表\red{导线包含的正方形(方块)数量}。}
        \conrow{几何意义}{电阻值仅取决于长宽比。只要 $L=W$,无论绝对尺寸大小,电阻均等于 $R_{\square}$。}
        \conrow{计算规则}{导线总电阻 = 串联的方块数 $\times$ 材料方块电阻值。}
    }
}

\entry{典型材料特性}{反映了不同层导电性能的巨大差异:}
\concepttable{
    \conrow{金属 (Al, Cu)}{$0.05 \Omega/\square$。极低,适合长距离信号传输。}
    \conrow{多晶硅 (Poly)}{$10 \sim 15 \Omega/\square$。较高,常用于短距离连接或栅极。}
    \conrow{扩散区 (Diff)}{$20 \sim 30 \Omega/\square$。最高,主要存在于有源区。}
}

\subsubsection{接触电阻 (了解即可)}

\entry{定义}{层与层之间的电气连接(接触孔和通孔)引入的额外电阻。}

\concepttable{
    \conrow{物理结构}{多层金属或金属与半导体间的垂直连接。}
    \conrow{设计原则}{\red{应尽可能减少接触孔和通孔的数量},以降低压降和信号延迟。}
}

\entry{尺寸与电阻关系}{
    \concepttable{
        \conrow{基本规律}{接触电阻与接触面积成反比。理论上增大尺寸可减小电阻。}
        \conrow{物理局限}{受限于\red{电流集聚效应},单纯增大单个孔径效率较低。}
    }
}

\entry{电流集聚效应}{
    \concepttable{
        \conrow{现象描述}{电流从高导电率层流入高电阻率层时,流线分布不均匀。}
        \conrow{分布特征}{电流集中在接触孔的\red{周边},中心区域电流密度极小。}
        \conrow{后果}{中心面积未被充分利用,大尺寸接触孔的电阻效率很低。}
    }
}

\entry{工程解决方案}{
    \concepttable{
        \conrow{核心结论}{\red{采用多个小尺寸接触孔并联}以减小接触电阻。}
        \conrow{原理解析}{利用并联电阻公式 $R_{total} = R/n$ 降低总阻值,并通过增加总周边长度减轻集聚效应,使电流分布更均匀。}
    }
}

\subsubsection{寄生电感}

\entry{计算原理}{利用电磁场理论中电感与电容的内在物理联系,通过已知的电容参数来推导电感参数。}

\concepttable{
    \conrow{L-C 耦合关系}{对于处于均匀介质中的传输线,单位长度电容 $c$ 与电感 $l$ 满足:\par
    \red{$cl = \varepsilon \mu$}}
    \conrow{参数解析}{$\varepsilon$ 为介电常数,$\mu$ 为磁导率。}
    \conrow{物理意义}{在均匀介质中,\red{一旦计算出寄生电容 $c$,即可直接推导出寄生电感 $l$},无需进行复杂的磁场分布计算。}
}

\entry{电磁波传播速度}{信号(电磁波)在互连线中的传播速度 $v$ 由 $l$ 和 $c$ 决定:}

\concepttable{
    \conrow{基础公式}{$v = 1/\sqrt{lc}$}
    \conrow{介质参数表示}{$v = 1/\sqrt{\varepsilon \mu}$}
    \conrow{相对参数表示}{$v = c_0 / \sqrt{\varepsilon_r \mu_r}$,其中 $c_0$ 为真空光速,$\varepsilon_r$ 为相对介电常数,$\mu_r$ 为相对磁导率。}
    \conrow{核心结论}{互连线中的信号传播速度主要受\red{周围绝缘介质的材料特性($\varepsilon_r$)}限制。}
}

\entry{忽略电感的前提}{集成电路芯片内部导线寄生电感通常极小,可忽略。}
\concepttable{
    \conrow{物理机制}{在低速或高阻抗电路中,$R$ 和 $C$ 效应占主导,感抗 $j\omega L$ 远小于电阻 $R$。}
    \conrow{建模简化}{\red{通常将导线建模为 $RC$ 电路而不是 $RLC$ 电路。}}
}

\entry{\red{考虑电感的必要条件}}{必须引入电感建模的两个关键条件:}
\concepttable{
    \conrow{低电阻}{使用铜 (Cu) 等低电阻率金属且导线截面较大,使电阻 $R$ 降低,感抗占比提升。}
    \conrow{高开关频率}{感抗 $X_L = 2\pi f L$ 与频率成正比。高频下即使 $L$ 很小,感抗也会显著影响性能。}
}

\entry{寄生电感引发的电路现象}{高频电路中由电感主导的负面效应:}
\concepttable{
    \conrow{振荡}{RLC 电路二阶阶跃响应特征。若阻尼不足,电压会在稳定值附近谐振摆动。}
    \conrow{过冲}{伴随振荡产生,电压瞬间超过 $V_{DD}$ 或低于 $GND$,可能导致击穿或误触发。}
    \conrow{信号反射}{高频下视为传输线。若特性阻抗 $Z_0 = \sqrt{L/C}$ 与端点不匹配,信号在终端反射。}
    \conrow{线间互感}{相邻导线间的磁场耦合产生感应电动势(互感),是产生串扰的磁性分量(对应容性串扰)。}
}


\noindent
\begin{minipage}[t]{0.48\linewidth}

\subsection{集总模型}

\entry{适用条件}{\red{导线很短},其自身寄生电阻 $r$ 远小于驱动门的输出电阻,且高频电感 $l$ 在高频下的阻抗可忽略时。}

\img[0.9\linewidth]{images/image-2025-12-31-12-02-32.png}

\concepttable{
    \conrow{物理特性}{导线可被视为一个\red{等电位体}(理想导线)。}
    \conrow{\red{延时计算}}{$\boxed{\tau = RC = rcL^2}$。}
    \conrow{参数说明}{$r, c$:单位长度的电阻和电容(Unit length parameters)。}
    \conrow{模型简化}{整根导线被简化为一个\red{单一的集总电容},作为驱动门的负载负载。}
}

\end{minipage}
\hfill
\begin{minipage}[t]{0.48\linewidth}
\subsection{分布模型}
111
\end{minipage}

\subsection{长导线的延时优化}

\section{第六章\ 组合逻辑}

\subsection{静态互补 CMOS 逻辑}

\subsection{有比逻辑}

\subsection{传输门逻辑}

\subsection{CPL 逻辑}

\subsection{动态逻辑}

\subsection{串联动态门}

\subsection{动态逻辑的速度和功耗}

\section{第七章\ 时序逻辑}

\subsection{双稳态原理}

\subsection{锁存器}

\subsection{主从边沿触发寄存器}

\subsection{时钟偏差和时钟抖动}

\section{第八章\ 加法器}

\subsection{二进制加法器}

\subsection{进位选择加法器}

\subsection{进位旁路加法器}

\subsection{超前进位加法器}

\section{第九章\ 乘法器}

\subsection{二进制乘法器(有、无符号}

\subsection{部分即产生(部分积压缩、Booth 编码)}

\subsection{部分积累加}

\subsection{逐位进位阵列乘法器}

\subsection{进位保留乘法器}

\subsection{最终相加}

\section{第十章\ 移位器}

\subsection{移位器设计}

\entry{移位种类}{三种常见的数字逻辑移位操作:}

\concepttable{
    \conrow{逻辑移位}{空位统一填充 \red{$0$}。常用于无符号数乘除(左移 $\times 2$,右移 $\div 2$)或位操作。}
    \conrow{算术移位}{针对带符号数。右移需进行\red{符号扩展}(MSB 保持不变,空位补符号位);左移通常低位补 $0$,需注意溢出。}
    \conrow{循环移位}{数据视为环形,移出的位重新填入另一端空位,不丢失信息。}
}

\subsection{漏斗型移位器}

只介绍了移位的实现方法,并未介绍其具体电路结构。

\imgleft[0.35]{images/image-2025-12-30-20-09-04.png}{

\entry{核心架构}{通过构建宽窗口统一处理逻辑、算术和循环移位。}

\concepttable{
    \conrow{基本原理}{构建一个 $2N-1$ 到 $0$ 的宽数据输入域(总位宽 $2N$),输出 $Y$ 为其中的 $N$ 位滑动窗口。}
    \conrow{输入构成}{由两个 $N$ 位向量 $B$(高位 $2N-1 \dots N$)和 $C$(低位 $N-1 \dots 0$)拼接而成。}
    \conrow{偏移量}{决定输出窗口起始位置,输出 $Y$ 对应输入域的 $[offset + N - 1 : offset]$ 区间。}
}

}

\entry{操作映射表}{设定原始 $N$ 位数据为 $A$,移位量为 $k$:}

\img[0.6\linewidth]{images/image-2025-12-30-20-14-01.png}

\concepttable{
        \conrow{逻辑右移}{配置 $B=0, C=A$,Offset=$k$。高位滑入 $B$ 区的 $0$,实现高位补 $0$。}
        \conrow{逻辑左移}{配置 $B=A, C=0$,Offset=$N-k$。从高位“回退”截取,低位引入 $C$ 中的 $0$。}
        \conrow{算术右移}{配置 $B=A_{N-1}\dots A_{N-1}$(符号位), $C=A$,Offset=$k$。高位填充符号位。}
        \conrow{算术左移}{配置 $B=A, C=0$,Offset=$N-k$。通路同逻辑左移,区别在于溢出判断。}
        \conrow{循环右移}{配置 $B=A, C=A$,Offset=$k$。$[A, A]$ 结构使移出的低位在高位出现。}
        \conrow{循环左移}{配置 $B=A, C=A$,Offset=$N-k$。利用偏移量实现反向的循环移动。}
}
\entry{核心逻辑}{漏斗移位器本质是“截取”结构,通过调整 $B, C$ 内容与 $N-k$ 偏移量将左移转化为右移截取。}

\subsubsection{一位左右移位器}

这个感觉像是学生根据自己学习传输管逻辑的理解设计出来的,从设计的角度来看稍显稚嫩,了解即可。

\imgleft[0.4]{images/image-2025-12-30-20-20-34.png}{

\concepttable{
    \conrow{输入/输出}{$A_i, A_{i-1}$ 为输入,$B_i, B_{i-1}$ 为输出。}
    \conrow{控制信号}{\blue{Right}, \blue{nop}, \blue{Left}。三者在同一时刻应当\red{互斥},以避免信号竞争。}
    \conrow{Bit-Slice}{模块化设计,通过重复堆叠 $N$ 次该模块实现 $N$ 位移位,本质是相邻位间的数据路由。}
}

\entry{工作模式}{
    \concepttable{
        \conrow{右移 (Right)}{Right=1,$A_i \to B_{i-1}$。高位数据流向低位。}
        \conrow{左移 (Left)}{Left=1,$A_{i-1} \to B_i$。低位数据流向高位。}
        \conrow{保持 (nop)}{nop=1,$A_i \to B_i$。数据位置不发生改变。}
    }
}

}

\entry{输出缓冲}{ NMOS 传输管逻辑存在高电平阈值损失(输出最高为 $V_{DD} - V_{th}$),需接缓冲器进行\red{电平恢复}并提供驱动能力。}

\subsection{桶型移位器}

\entry{系统架构}{定义输入 $A$ ($A_3 \dots A_0$),输出 $B$ ($B_3 \dots B_0$),控制信号 $Sh$ ($Sh0 \dots Sh3$)。信号最多通过一个传输门,理论上移位延时不依赖于移位器的大小和移位位数。}

\imgleft[0.4]{images/image-2025-12-30-20-28-49.png}{
    MOS 管阵列的\red{行数等于数据的字长},\red{列数等于可支持的移位数}。

    \concepttable{
        \conrow{独热编码}{控制线采用 \red{One-hot} 编码,仅一根为高电平,决定移位位数。}
        \conrow{物理结构}{由 NMOS \red{传输管} 构成的 $4 \times 4$ \red{交叉开关阵列}。}
    }
    \entry{算术右移实现}{核心要求是空出的高位用\red{符号位 $A_3$} 填充(符号扩展)。}
    $\boxed{B_3B_2B_1B_0 = A_3A_2A_1A_0 \gg Sh_3Sh_2Sh_1Sh_0}$
}

\concepttable{
        \conrow{$Sh0$ (不移位)}{$A_3 \to B_3, A_2 \to B_2, A_1 \to B_1, A_0 \to B_0$。}
        \conrow{$Sh1$ (右移 1 位)}{$A_3 \to B_2, A_2 \to B_1, A_1 \to B_0$,且 $B_3 = A_3$。}
        \conrow{$Sh2$ (右移 2 位)}{$A_3 \to B_1, A_2 \to B_0$,且 $B_3 = B_2 = A_3$。}
        \conrow{$Sh3$ (右移 3 位)}{$A_3 \to B_0$,且 $B_3 = B_2 = B_1 = A_3$。}
}

\subsection{对数移位器}

\entry{核心设计理念}{采用\red{分级控制},将总移位值分解为二进制权重的组合。}

\imgleft[0.4]{images/image-2025-12-30-21-05-25.png}{

\concepttable{
    \conrow{二进制分解}{移位量 $K$ 基于\red{“2 的指数位”} ($2^k$) 进行分解。每一级硬件只负责移动固定的 $2^k$ 位。}
    \conrow{级联结构}{级联多个移位级,每一级由 $Sh_k$ 控制。若 $Sh_k$ 有效则移位,否则信号直通。}
}

\entry{电路结构分析}{以 0-7 位对数移位器为例:}
\concepttable{
    \conrow{第一级}{控制信号 $Sh1$,负责右移 $2^0 = 1$ 位。}
    \conrow{第二级}{控制信号 $Sh2$,负责右移 $2^1 = 2$ 位。}
    \conrow{第三级}{控制信号 $Sh4$,负责右移 $2^2 = 4$ 位。}
    \conrow{输出缓冲}{最后一级经过 Buffer 输出最终结果 $B_3, B_2, B_1, B_0$。}
}

}

\entry{移位范围}{对于 $N$ 位数据,通常级联 $\log_2 N$ 级。图中三级结构支持最大 $1+2+4=7$ 位的移位量,适用于大范围移位场景。}
\subsubsection{对数移位器原理与特性}

\entry{单元逻辑}{本质为并行的 \red{2 选 1 多路复用器 (2-to-1 MUX)}。}
\concepttable{
    \conrow{结构}{每一级节点由两个 NMOS 传输管组成:一个负责直通路径,一个负责移位路径。}
    \conrow{控制逻辑}{每一级根据二进制控制位决定是“直传”还是“跳跃” $2^k$ 位。}
}

\entry{性能分析与设计权衡}{
    \concepttable{
        \conrow{速度特性}{延迟与移位宽度 $M$ 呈\red{对数关系},即具有 $\log_2 M$ 级延迟。}
        \conrow{RC 延时}{信号必须穿过每一级传输管,串联电阻导致 $RC$ 延时随级数累积。}
        \conrow{优化手段}{在级与级之间插入\red{中间缓冲器 (Buffer)},以打断长 $RC$ 链并恢复信号驱动能力。}
        \conrow{控制编码}{无需译码器。控制信号直接对应二进制权值(如移位 3 位即 $Sh1=1, Sh2=1, Sh4=0$)。}
    }
}

\entry{适用场景对比}{
    \concepttable{
        \conrow{桶型移位器}{适用于\red{较小位宽}。仅一级传输管延迟,速度极快,但面积代价随位宽增加迅速上升。}
        \conrow{对数移位器}{适用于\red{大位宽}(如 64 位以上)。结构易于参数化和 EDA 自动生成,面积效率更高。}
    }
}

\subsection{习题解析}

\subsubsection{作业题}

\entry{作业6-2}{画出一个支持循环右移的 4-bit 桶型移位器的电路结构图}

这里牢王应该是让这个移位器只支持循环移位,桶型移位器就像 Mask ROM 一样,设好就改不了了。

\section{第十一章\ MOS 存储器}

\subsection{MOS 存储器分类}

\subsection{存储器结构}

\subsection{存储体}

\subsection{地址译码器}

\subsection{读写控制及输入输出电路}

\subsection{Mask ROM}

\subsection{SRAM}

\subsection{DRAM}

\section{第十二章\ Verilog 硬件设计}

\subsection{组合逻辑}

\subsection{时序逻辑}

\subsection{状态机}
\entry{基本组成}{状态机由三个核心模块构成:}

\concepttable{
    \conrow{次态逻辑}{\red{组合逻辑}模块。负责根据“输入”和“现态”计算出下一个时钟周期应该跳转到的“次态”。}
    \conrow{时序逻辑}{\red{存储模块}(通常由 $DFF$ 构成)。负责存储“现态”,在时钟边沿到来时,将“次态”更新为新的“现态”。}
    \conrow{输出逻辑}{\red{组合逻辑}模块。负责产生系统的最终“输出”。}
}

\noindent
\begin{minipage}[t]{0.48\linewidth}
  \img[0.9\linewidth]{images/image-2025-12-30-21-26-07.png}
  \vspace{2pt}
  \entry{摩尔型 (Moore) 状态机}{输出仅取决于\red{现态}。}

    \concepttable{
        \conrow{信号流向}{输入信号与反馈的“现态”共同决定“次态”;“时序逻辑”在时钟驱动下更新“现态”。}
        \conrow{核心特征}{\red{输出逻辑的输入}仅来自于寄存器的输出 $Q$ 端(即现态)。}
        \conrow{同步特性}{输出严格同步于状态变化。输入信号的改变必须先触发状态翻转,才能传递至输出。}
        \conrow{工程意义}{输出通常比输入滞后一个周期,但具有良好的\red{抗毛刺 (Glitch)} 能力。}
    }
\end{minipage}
\hfill
\begin{minipage}[t]{0.48\linewidth}

\img[0.9\linewidth]{images/image-2025-12-30-21-32-44.png}

\entry{米里型 (Mealy) 状态机}{输出由\red{现态与输入}共同决定。}

\concepttable{
    \conrow{信号流向}{次态逻辑与时序逻辑部分与 Moore 型基本一致。}
    \conrow{关键特征}{输出逻辑模块同时接收来自寄存器的“现态”与直接来自外部的“输入”信号。}
    \conrow{响应特性}{输出可对输入变化做出\red{异步响应}。在同一时钟周期内,输入改变输出即可随之改变,无需等待时钟边沿。}
    \conrow{工程对比}{实现相同功能时状态数通常更少,响应速度更快;但输入端的噪声或毛刺容易直接传递至输出端。}
}
\end{minipage}

\subsubsection{状态机设计描述方式}

\entry{核心概念}{根据现态 (CS)、次态 (NS) 和输出逻辑 (OL) 在 \blue{always} 块中的分配,分为三种经典描述方式:}

\entry{1. 三段式 (Three-process)}{将三个逻辑环节完全解耦,与硬件框图一一对应。}
\concepttable{
    \conrow{现态 (CS)}{时序逻辑 \red{always} 块,由时钟驱动更新状态寄存器。}
    \conrow{次态 (NS)}{组合逻辑 \red{always} 块,根据输入和 $CS$ 计算 $NS$。}
    \conrow{输出逻辑 (OL)}{独立组合逻辑 \red{always} 块,描述输出信号。}
    \conrow{优点}{结构最清晰,可读性最高,利于综合工具进行时序分析和优化。}
}

\entry{2. 两段式 (Two-process)}{将逻辑环节进行合并,是工业界常用的写法。}
\concepttable{
    \conrow{策略一:(CS+NS) + (OL)}{第一个过程块描述现态更新与次态计算;第二个描述输出。}
    \conrow{策略二:(CS) + (NS+OL)}{第一个过程块描述寄存器更新;第二个合并次态与输出逻辑。}
    \conrow{特性}{明确分离了时序逻辑与组合逻辑,但需注意复杂 OL 可能导致关键路径过长。}
}

\entry{3. 一段式 (Single-process)}{将 $CS, NS, OL$ 全部放入同一个时序 \blue{always} 块中。}
\concepttable{
    \conrow{实现方式}{通常基于时钟边沿触发,代码量最少,结构简洁。}
    \conrow{输出特性}{输出信号被综合为\red{寄存器输出} (Registered Output),天然无毛刺。}
    \conrow{局限性}{输出信号会比状态变化\red{滞后一个时钟周期};复杂 Mealy 机描述臃肿。}
}

\subsubsection{三段式状态机}

这里放一个基于 三段式描述的有限状态机(FSM)设计实例,具体功能为 “101序列检测器”。

\imgleft[0.3]{images/image-2025-12-30-21-51-35.png}{
\entry{状态转移图分析}{该设计为 \red{Moore 型}状态机,输出 $z$ 仅取决于当前状态(节点内标记为 $S_n/z$)。}

\concepttable{
    \conrow{$S0/0$}{初始状态 (Idle),表示未检测到有效序列。}
    \conrow{$S1/0$}{检测到了“1”,即序列的第一位。}
    \conrow{$S2/0$}{检测到了“10”,即序列的前两位。}
    \conrow{$S3/1$}{检测到了“101”,序列匹配成功,输出 \red{$z=1$}。}
}

\entry{转移逻辑 (重叠检测)}{
    \concepttable{
        \conrow{从 $S0$}{输入 $1 \to S1$;输入 $0 \to S0$。}
        \conrow{从 $S1$}{输入 $0 \to S2$;输入 $1 \to S1$(最新的“1”可作为新序列开头)。}
        \conrow{从 $S2$}{输入 $1 \to S3$;输入 $0 \to S0$(序列中断,需重新检测)。}
        \conrow{从 $S3$}{输入 $1 \to S1$;输入 $0 \to S2$(最后的“1”或“10”作为下一序列前缀)。}
    }
}
}

% --- 手动换栏命令(如果需要强制换列)---
% \columnbreak 

\end{multicols*}

\end{document}